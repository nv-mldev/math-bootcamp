\chapter{Introduction to Functions}
\author{Nithin}

\section{Domain,Range and Equality}

\subsection{What is a Function ?}
A function associates every number in some set of real numbers, called the domain of the function, with exactly one real number.

\subsection{Domain}
If a function is defined by a formula, with no domain specified, then the domain is assumed to be the set of all real numbers for which the formula makes sense and produces a real number.

\subsubsection{Example 3}
Find the domain of the function \(f\)  defined by \[f (x) = (3x-1)^2 \]

\subsubsection{Example 4}
Find the domain of the function \(f\) defined by \[h (t) = \frac{t^2 + 3t + 7}{t-4} \]

\subsubsection{Example 6}
Find the domain of the function g defined by \[g(x) = \sqrt{|x|-5}\]

\subsection{Range}
The range of a function \(f\) is the set of all numbers \(y\) such that \(f (x) = y\) for at least one \(x\) in the domain of \(f\).

\subsubsection{Example 4}
The domain of \(f\) is the interval \( [2, 5] \), with \(f\) defined on this interval by the equation \(f (x) = 3x + 1\).
Solution:
\[ y = f(x) = 3x+1 \]
\[ 2 \leq \frac{y-1}{3} \leq 5. \]
\[ 7 \leq y \leq 16. \]

\subsubsection{Example 5}
The domain of \(g\) is the interval \([1,20]\), with \(g\) defined on this interval by the equation
\[ g(x) = |x - 5|. \]
Is \(2\) in the range of \(g\)?
Solution:
\[ y =  |x-5| \]
for \(x-5>0 \),\(y = x-5 \implies x = y+5 \)
\[ 5 < y+5 \leq 20  \implies  0 < y \leq 15 \]
for \(x-5<0 \), \(y = -(x-5) \implies y = -x+5 \implies 5-y = x \)
\[ 1 \leq 5-y \leq 5  \implies -4 \leq -y \leq 0 \implies  4 \geq y \geq 0 \]

\subsection{Equality of Functions}
Two functions are equal if and only if they have the same domain and the same value at every number in that domain.

\subsubsection{Example}
Suppose \(f\) is the function whose domain is the set of real numbers, with \(f\) defined on this domain by
\[f (x) = x^2\]
Suppose \(g\) is the function whose domain is the set of positive numbers, with \(g\) defined on this domain by
\[g(x) = x^2\]
Are \(f\) and \(g\) equal functions?

\subsubsection{Example 2}
Suppose \(f\) and \(g\) are functions whose domain is the set consisting of the two numbers \(\{1, 2\}\) with \(f\) and \(g\) defined on this domain by the formulas
\[f (x) = x^2\] and \(g(x) = 3x-2\).
Are \(f\) and \(g\) equal functions?

\section{Analytical Geometry}
\subsection{What is Analytic Geometry?}
\begin{itemize}
    \item \textbf{Analytic Geometry} (also called \textit{coordinate geometry} or \textit{Cartesian geometry}) bridges algebra and geometry.
    \item It uses a coordinate system to study geometric shapes and properties.
    \item Geometric objects are represented as algebraic equations.
\end{itemize}

\subsection{Co-ordinate Plane}
\begin{figure}[h]
    \centering
    \includegraphics[scale=0.23]{cartesian.png}
\end{figure}
The plane with this system of labeling is often called the \textbf{Cartesian plane} in honor of the French mathematician Rene Descartes(1596-1650), who described this technique in his 1637 book Discourse on Method.

\subsection{Graph Functions}
The graph of a function \(f\) is the set of points of the form \(x, f (x)\) as x varies over the domain of \(f\).

\subsubsection{Graphs}
\begin{figure}[h]
  \centering
  \includegraphics[scale=0.3]{graph.png}
\end{figure}
\begin{figure}[h]
  \centering
  \includegraphics[scale=0.3]{graph2.png}
\end{figure}

\subsection{Checking for a function: Vertical line test}
The line \(x=1\) intersects the curve at two points. That is that for each \(x\) value there are multiple \(y\) values which is contradicting to definition of a function.
\subsubsection{Vertical Line Test}
A set of points in the coordinate plane is the graph of some function if and only if every vertical line intersects the set in at most one point.

\section{Average Rate of change}
The rate of change describes how an output quantity changes relative to the change in the input quantity. The unit on a rate of change.
The \textbf{average rate of change} of a function \(f\) over an interval \([x_{1}, x_{2}]\) is given by
\[ =\frac{\Delta y}{\Delta x} =\frac{f(x_{2}) - f(x_{1})}{x_{2} - x_{1}} \]

\subsection{Increasing, Decreasing and Constant Functions}
\begin{figure}
  \centering
  \includegraphics[scale=0.3]{inc-dec.png}
\end{figure}

\subsubsection{Increasing Function}
A function \(f\) is said to be increasing on an interval \((a, b)\) if for any \(x_{1}, x_{2}\) in \((a, b)\) with \(x_{1} < x_{2}\), we have \(f(x_{1}) < f(x_{2})\) or it has a positive rate of change over the interval.

\subsubsection{Decreasing Function}
A function \(f\) is said to be decreasing on an interval \((a, b)\) if for any \(x_{1}, x_{2}\) in \((a, b)\) with \(x_{1} < x_{2}\), we have \(f(x_{1}) > f(x_{2})\) or it has a negative rate of change over the interval.

\subsubsection{Constant Function}
A function \(f\) is said to be constant on an interval \((a, b)\) if for any \(x_{1}, x_{2}\) in \((a, b)\) with \(x_{1} < x_{2}\), we have \(f(x_{1}) = f(x_{2})\) or it has a zero rate of change over the interval.

\subsection{Local Maxima and Local Minima}
\begin{figure}
  \centering
  \includegraphics[scale=0.5]{max-min.png}
\end{figure}

\subsubsection{Local Maximum}
A function \(f\) has a \textbf{local maximum} at \(x = c\) if there exists an interval \((a, b)\) containing \(c\) such that
\[ f(c) \geq f(x) \quad \text{for all } x \in (a, b) \]

\subsubsection{Local Minimum}
A function \(f\) has a \textbf{local minimum} at \(x = c\) if there exists an interval \((a, b)\) containing \(c\) such that
\[ f(c) \leq f(x) \quad \text{for all } x \in (a, b) \]

\section{Function Transformation}
\subsection{Vertical Transformations}
\subsubsection{Shifting a graph up or down}
Suppose \(f \) is a function and \(a > 0\). Upshift \(g\) and  Downshift \(h\) by
\[g(x) = f (x) + a \;\; h(x) = f (x) - a \]
\begin{figure}[h]
  \centering
  \includegraphics[scale=0.4]{vertical_shift.png}
\end{figure}

\subsubsection{Vertical Stretch}
Suppose \(f\) is a function and \(c > 0\). Define a function \(g\) by
\[g(x) = c f (x)\]
\begin{figure}[h]
  \centering
  \includegraphics[scale=0.5]{vertical_stretch.png}
\end{figure}

\subsubsection{Flipping along the Vertical Axis}
Vertical fliiping of \(f(x) \) is
\[g(x) = - f (x) \]
\begin{figure}[h]
  \centering
  \includegraphics[scale=0.5]{vertical_flip.png}
\end{figure}

\subsection{Horizontal Transformation}
\subsubsection{Horizontal Shift}
\[g(x) = f(x+a), h(x) = f(x-a)\]
\begin{figure}[h]
  \centering
  \includegraphics[scale=0.5]{horizontal-shift.png}
\end{figure}

\subsubsection{Horizontal Stretching}
\[g(x) = f(cx)\]
\begin{figure}[h]
  \centering
  \includegraphics[scale=0.55]{horizontal-stretch.png}
\end{figure}

\subsection{Flipping ac the Vertical Axis}
\subsubsection{Horizontal Stretching}
\[g(x) = f(-x)\]
\begin{figure}[h]
  \centering
  \includegraphics[scale=0.55]{vertical-axis-flip.png}
\end{figure}

\subsection{Combinations of vertical Transformation}
\begin{figure}[h]
  \centering
  \includegraphics[scale=0.55]{vertical_combination.png}
\end{figure}

\subsection{Even Functions}
\[ f(-x) = f(x) \quad \text{for all } x \text{ in the domain} \]
Example: \( f(x) = x^2, \quad f(x) = \cos x \)
The graph of an even function is symmetric across the vertical axis.
\begin{figure}
\centering
\includegraphics[width=0.45\linewidth]{even.png}
\includegraphics[width=0.45\linewidth]{even2.png}
\end{figure}

\subsection{Odd Function}
\[ f(-x) = -f(x) \quad \text{for all } x \text{ in the domain} \]
Example: \( f(x) = x^3, \quad f(x) = \sin x \)
The graph of an even function is symmetric if flipped or rotated 18 across the origin.
\begin{figure}
\centering
\includegraphics[width=0.45\linewidth]{odd.png}
\includegraphics[width=0.45\linewidth]{odd2.png}
\end{figure}

\section{Function Composition}
\subsection{Algebra of Functions}
Suppose \(f\) and \(g\) are functions. We can define new functions from \(f\) and \(g\) as follows: \\
\textbf{Sum:}
\[ (f+g)(x) = f(x) + g(x) \]
\textbf{Difference:}
\[ (f-g)(x) = f(x) - g(x) \]
\textbf{Product:}
\[ (f\cdot g)(x) = f(x) \cdot g(x) \]
\textbf{Quotient:}
\[ \left(\frac{f}{g}\right)(x) = \frac{f(x)}{g(x)} \quad \text{provided } g(x) \neq 0. \]
\textbf{Note:} If \(f\) and \(g\) have domains \(D_f\) and \(D_g\), then these operations are defined on the intersection \(D_f \cap D_g\). In the case of the quotient, it is defined on
\[ \{x \in D_f \cap D_g : g(x) \neq 0\}. \]

\subsection{Exercise}
\( f(x)  = \sqrt{x-3}\) and \(g(x) = \sqrt{8 - x } \)
Evaluate
\begin{enumerate}
  \item[a.] \((f+g)(x)\)
  \item[b.]  \((fg)(x)\)
  \item[c.] Find the domain of above
\end{enumerate}
Sol:
\begin{enumerate}
  \item[a.] \(\sqrt{x-3} + \sqrt{8-x} \)
  \item[b.] \(\sqrt{(x-3)(8-x)} \)
  \item[c.]  Domian of \begin{enumerate}
    \item[a.] \(x\geq 3 \)
    \item[b.] \(x \leq 8 \)
    \item[c.] \( 3 \leq x \leq 8 \)
  \end{enumerate}
\end{enumerate}

\subsection{Function Composition}
\subsubsection{Definition:}
If \( f(x) \) and \( g(x) \) are functions, then the composition of \( f \) and \( g \), denoted by \( f \circ g \), is defined by
\[ (f \circ g)(x) = f(g(x)). \]
\textbf{Example:}
Consider the function
\[ h(x) = \sqrt{x+3}. \]
We can express \( h(x) \) as a composition of two functions \( f \) and \( g \) where:
\[ f(x) = \sqrt{x} \quad \text{and} \quad g(x) = x+3. \]
Then,
\[ h(x) = f(g(x)) = f(x+3) = \sqrt{x+3}. \]

\subsection{Exercise}
\( f(x) = \frac{1}{x-4} \) and \( g(x) = x^{2} \)
\begin{enumerate}
  \item \(f \circ g \)
  \item \( g \circ f \)
  \item domian of \(f \circ g \)
  \item domain of \( g \circ f \)
\end{enumerate}
Sol:
\begin{enumerate}
  \item  \(f(g(x)) = \frac{1}{x^{2} - 4}\)
  \item \(g(f(x)) = (\frac{1}{x-4})^{2}\)
  \item \(R - \{-2,2\} \)
  \item \(R - \{4\} \)
\end{enumerate}

\subsection{Composition Machine}
\begin{figure}
  \centering
  \includegraphics[scale=0.3]{composition.png}
\end{figure}

\subsection{Excersise}
\textbf{Problem:} Suppose your cell phone company charges \(\$0.05\) per minute plus \(\$0.47\) for each call to China.
\begin{enumerate}
  \item[(a)] Find a function \(p\) that gives the amount charged by your cell phone company for a call to China as a function of the number of minutes \(m\).
  \item[(b)] Suppose the tax on cell phone bills is 6\% plus \(\$0.01\) for each call. Find a function \(t\) that gives your total cost, including tax, for a call to China as a function of the amount charged by your cell phone company.
  \item[(c)] Explain why the composition \(t \circ p\) gives your total cost, including tax, of making a cell phone call to China as a function of the number of minutes.
  \item[(d)] Compute a formula for \(t \circ p\).
  \item[(e)] What is your total cost for a ten-minute call to China?
\end{enumerate}

\subsection{Solution}
\textbf{(a)} The company charges \(\$0.05\) per minute plus a fixed charge of \(\$0.47\) per call. Hence, the pre-tax charge function is
\[ p(m)=0.05m+0.47. \]
\textbf{(b)} The tax on the cell phone bill is 6\% of the pre-tax amount plus an additional \(\$0.01\) per call. Thus, if the pre-tax charge is \(x\), the total cost function (including tax) is
\[ t(x)=1.06x+0.01. \]
\textbf{(c)} The composition \(t \circ p\) means we first compute the pre-tax charge \(p(m)\) for a call of \(m\) minutes, and then we apply the tax function \(t\) to this amount. In other words, \(t(p(m))\) gives the total cost, including tax, as a function of the number of minutes.
\textbf{(d)} To compute the composition, substitute \(p(m)\) into \(t\):
\[ (t \circ p)(m)=t(p(m))=1.06\bigl(0.05m+0.47\bigr)+0.01. \]
Distribute \(1.06\):
\[ 1.06(0.05m)=0.053m \quad \text{and} \quad 1.06(0.47)=0.4982. \]
Thus,
\[ (t \circ p)(m)=0.053m+0.4982+0.01=0.053m+0.5082. \]
\textbf{(e)} For a ten-minute call (\(m=10\)):
\[ (t \circ p)(10)=0.053(10)+0.5082=0.53+0.5082=1.0382. \]
Rounded to the nearest cent, the total cost is approximately \(\$1.04\).

\subsection{Identity Function}
The identity function is defined by
\[ I(x) = x \quad \text{for every number } x. \]
The function \(I\) is the identity for composition. If \(f\) is any function, then
\[ f \circ I = I \circ f = f. \]

\subsection{Decomposing the Functions}
\subsubsection{Function Decomposition}
\[ T(y) = \frac{|y^{2} -3|}{|y^{2} - 7|} \]
Sol:
\[f(y) = |y|, g(y) = \frac{y^{2}-3}{y^{2}-7} \]
\[ f(y) = \frac{|y-3|}{|y-7|}, g(y) = y^{2}\]
Composition is associative if \(f,g, h\) are functions then
\[(f \circ g) \circ h  = f \circ (g \circ h) \]

\subsection{Example}
\subsubsection{Composition of three functions}
\[T(x) = \left| \frac{x^{2}-3}{x^{2} - 7}  \right| \]
Sol:
\[f(x) = |x|, g(x) = \frac{x-3}{x-7}, h(x) = x^{2}\]

\subsection{Linear Functions}
A linear function is a function \(h\) of the form
\[h(x) = mx + b\]
where \(m\) and \(b\) are numbers.

\subsection{Linear Functions as Composition}
\subsubsection{Vertical Transformations as Compositions}
A funtion \(g(x)\) is defined by
\[g(x)= -2f(x)+1\]
Write \(g(x)\) as a the composition of a linear function with \(f(x)\)
\[h(x) = -2x + 1 \]
\[\implies g(x) = h(f(x)) \implies g = h \circ f \]

\subsubsection{Horizontal Transformations as Compositions}
A funtion \(g(x)\) is defined by
\[g(x)= f(2x)+1\]
Write \(g(x)\) as a the composition of a linear function,  \(f(x)\) and other linear function
\[h(x) = x + 1, p(x) = 2x \]
\[\implies g(x) = h(f(p(x))) \implies g = h \circ f \circ p \]

\section{Inverse Functions}
\subsection{Inverse Function: Example}
Consider the function \( f: \mathbb{R} \to \mathbb{R} \) defined by
\[ y = \frac{9}{5}x + 32, \]
which converts a temperature \( x \) in Celsius to Fahrenheit \( y \). 
The inverse function \( f^{-1} \) converts Fahrenheit back to Celsius:
\[ f^{-1}(y) = \frac{5}{9}(y - 32). \]
Verifying that these functions are inverses:
\[ f^{-1}(f(x)) = \frac{5}{9}\Bigl(\frac{9}{5}x + 32 - 32\Bigr) = x, \]
\[ f(f^{-1}(f)) = \frac{9}{5}\Bigl(\frac{5}{9}(y - 32)\Bigr) + 32 = y. \]

\subsection{One-to-One Function}
A function \(f\) is called one-to-one if for each number \(y\) in the range of \(f\) there is
exactly one number \(x\) in the domain of \(f\) such that \(f (x) = y\).

\subsection{Inverse Function}
\subsubsection{Definition}
Suppose \( f \) is a one-to-one function.
\begin{itemize}
  \item If \( y \) is in the range of \( f \), then \( f^{-1}(y) \) is defined to be the number \( x \) such that \( f(x) = y \).
  \item The function \( f^{-1} \) is called the \emph{inverse function} of \( f \).
\end{itemize}
\textbf{Short version:}
\begin{itemize}
  \item \( f^{-1}(y) = x \) means \( f(x) = y \).
\end{itemize}

\begin{figure}
  \centering
  \includegraphics[scale=0.4]{inverse.jpeg}
\end{figure}

\subsection{Domain and Range of an Inverse Function}
If \( f \) is a one-to-one function, then:
\begin{itemize}
  \item The domain of \( f^{-1} \) equals the range of \( f \).
  \item The range of \( f^{-1} \) equals the domain of \( f \).
\end{itemize}

\subsection{Increasing and Decreasing Function}
\subsubsection{Increasing}
A function \(f\) is called increasing if \(f (a) < f (b)\) whenever \(a < b\) and \(a, b \) are in
the domain of \(f\).

\subsubsection{Decreasing}
A function \(f\) is called decreasing if \(f (a) >  f (b)\) whenever \(a < b\) and \(a, b \) are in
the domain of \(f\).

\subsubsection{Increasing and decreasing functions are one-to-one}
\begin{itemize}
  \item Every increasing function is one-to-one
  \item Every decreasing function is one-to-one.
\end{itemize}

\subsection{Exercise}
\begin{figure}
  \centering
  \includegraphics[scale=0.2]{increase.jpeg}
\end{figure}
\begin{enumerate}
  \item[a.] Decreasing
  \item[b.] Increasing
  \item[c.] Neither
\end{enumerate}

\subsection{Do all one-to-one maps are increasing or decreasing ?}
% \begin{figure}
%   \centering
%   \includegraphics[scale=0.6]{non-increse-decrese.png}
% \end{figure}

\subsection{Increasing and Decreasing Functions}
\subsubsection{Inverses of increasing and decreasing functions}
\begin{itemize}
  \item The inverse of an increasing function is increasing.
  \item The inverse of a decreasing function is decreasing.
\end{itemize}
