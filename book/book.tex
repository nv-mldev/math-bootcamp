\documentclass{book}

% Required for \includegraphics and \graphicspath
\usepackage{graphicx}
\usepackage{geometry}
\usepackage{hyperref}
\usepackage{subcaption} % For subfigure environments

% Mathematical packages for symbols and equations
\usepackage{amsmath}
\usepackage{amssymb}
\usepackage{amsthm}
\usepackage{amsmath,amsthm,amssymb,mathtools}
\usepackage{lmodern}
\usepackage{microtype}
\usepackage{enumitem}

% For dialogue environment - using the proper package
\usepackage{dialogue}

% For colored boxes and enhanced environments
\usepackage{xcolor}
\usepackage[skins,breakable]{tcolorbox}
\usepackage{tikz}
\usepackage{titlesec} % For custom heading styles
% Utility for checking empty arguments when supplying optional titles
\usepackage{etoolbox}
% xparse for defining wrapper environments that forward to breakable boxes
\usepackage{xparse}
% Provide the H float specifier used in some chapter figures
\usepackage{float}

% Custom heading levels
\newcommand{\headingA}[1]{%
  \vspace{0.7em}\noindent{\fontsize{13}{15}\selectfont\bfseries #1}\vspace{0.4em}\par\noindent
}

% Level 5 heading (smaller than headingA but still distinctive)
\newcommand{\headingB}[1]{%
  \vspace{0.5em}\noindent{\fontsize{12}{14}\selectfont\bfseries #1}\vspace{0.3em}\par\noindent
}

% Informal heading (for less important divisions)
\newcommand{\infoheading}[1]{%
  \vspace{0.4em}\noindent\textit{\textbf{#1}}\vspace{0.2em}\par\noindent
}

% Set up page geometry
\geometry{
    a4paper,
    left=2.5cm,
    right=2.5cm,
    top=3cm,
    bottom=3cm
}

% All figures are consolidated in figures/ directory as per the plan
\graphicspath{{figures/}}

% Custom theorem environments
\newtheorem{theorem}{Theorem}[chapter]
\newtheorem{definition}{Definition}[chapter]
\newtheorem{example}{Example}[chapter]

% Define beautiful colored block environments
\tcbset{
    base/.style={
        arc=3mm,
        outer arc=3mm,
        boxrule=0.5pt,
        left=5mm,
        right=5mm,
        top=3mm,
        bottom=3mm,
        fonttitle=\bfseries\sffamily,
        coltitle=white,
        enhanced,
        drop shadow,
        before skip=10pt,
        after skip=10pt
    },
    breakable-base/.style={
        base,
        breakable,
        pad at break=3mm
    }
}

% Breakable Definition block (Blue theme) — accepts optional title as a second argument
\newtcolorbox{definitionboxbreak}[2][]{
    breakable-base,
    colback=blue!8,
    colframe=blue!50!black,
    colbacktitle=blue!70!black,
    title={\ifstrempty{#2}{Definition}{#2}},
    #1
}

% Breakable Theorem block (Green theme) — accepts optional title as a second argument
\newtcolorbox{theoremboxbreak}[2][]{
    breakable-base,
    colback=green!8,
    colframe=green!50!black,
    colbacktitle=green!70!black,
    title={\ifstrempty{#2}{Theorem}{#2}},
    #1
}

% Breakable Example block (Orange theme) — accepts optional title as a second argument
\newtcolorbox{exampleboxbreak}[2][]{
    breakable-base,
    colback=orange!8,
    colframe=orange!50!black,
    colbacktitle=orange!70!black,
    title={\ifstrempty{#2}{Example}{#2}},
    halign title=left,
    #1
}

% Breakable Important block (Red theme)
\newtcolorbox{importantboxbreak}[2][]{
    breakable-base,
    colback=red!8,
    colframe=red!50!black,
    colbacktitle=red!70!black,
    title={\ifstrempty{#2}{Important}{#2}},
    #1
}

% Breakable Exercise block (Teal theme)
\newtcolorbox{exerciseboxbreak}[2][]{
    breakable-base,
    colback=teal!8,
    colframe=teal!50!black,
    colbacktitle=teal!70!black,
    title={\ifstrempty{#2}{Exercise}{#2}},
    #1
}

% Breakable Proof block (Gray theme)
\newtcolorbox{proofboxbreak}[2][]{
    breakable-base,
    colback=gray!8,
    colframe=gray!50!black,
    colbacktitle=gray!70!black,
    title={\ifstrempty{#2}{Proof}{#2}},
    #1
}

% Breakable Solution block (Purple theme) — accepts optional title as a second argument
\newtcolorbox{solutionboxbreak}[2][]{
    breakable-base,
    colback=purple!8,
    colframe=purple!50!black,
    colbacktitle=purple!70!black,
    title={\ifstrempty{#2}{Solution}{#2}},
    #1
}

% Breakable Key concept block (Cyan theme) — accepts optional title as a second argument
\newtcolorbox{keyconceptboxbreak}[2][]{
    breakable-base,
    colback=cyan!8,
    colframe=cyan!50!black,
    colbacktitle=cyan!70!black,
    title={\ifstrempty{#2}{Key Concept}{Key Concept: #2}},
    #1
}

% Breakable Fun Facts block (Magenta theme) — accepts optional title as a second argument
\newtcolorbox{funfactsbreak}[2][]0{
    breakable-base,
    colback=magenta!8,
    colframe=magenta!50!black,
    colbacktitle=magenta!70!black,
    title={\ifstrempty{#2}{Fun Facts}{Fun Facts: #2}},
    #1
}

% Breakable Identities block (Gold theme) — accepts optional title as a second argument
\newtcolorbox{identitiesboxbreak}[2][]{
    breakable-base,
    colback=yellow!15,
    colframe=yellow!50!orange,
    colbacktitle=yellow!50!orange!80!black,
    title={\ifstrempty{#2}{Mathematical Identities}{Mathematical Identities: #2}},
    #1
}

% Breakable Axioms block (Lime/Olive theme) — accepts optional title as a second argument
\newtcolorbox{axiomsboxbreak}[2][]{
    breakable-base,
    colback=lime!8,
    colframe=olive!50!black,
    colbacktitle=lime!50!olive!80!black,
    title={\ifstrempty{#2}{Axioms}{Axioms: #2}},
    #1
}

% Code block environment for Python/programming examples
\usepackage{listings}
\usepackage{xcolor}

% Define colors for syntax highlighting
\definecolor{codegreen}{rgb}{0,0.6,0}
\definecolor{codegray}{rgb}{0.5,0.5,0.5}
\definecolor{codepurple}{rgb}{0.58,0,0.82}
\definecolor{backcolour}{rgb}{0.95,0.95,0.92}

% Configure listings style
\lstdefinestyle{pythonstyle}{
    language=Python,
    backgroundcolor=\color{backcolour},
    commentstyle=\color{codegreen},
    keywordstyle=\color{magenta},
    numberstyle=\tiny\color{codegray},
    stringstyle=\color{codepurple},
    basicstyle=\ttfamily\small,
    breakatwhitespace=false,
    breaklines=true,
    captionpos=b,
    keepspaces=true,
    numbers=left,
    numbersep=5pt,
    showspaces=false,
    showstringspaces=false,
    showtabs=false,
    tabsize=4,
    frame=single,
    rulecolor=\color{black!30}
}

\lstset{style=pythonstyle}

% Breakable code block environment (Light gray theme with syntax highlighting)
\newtcolorbox{codeblock}[2][]{
    breakable-base,
    colback=backcolour,
    colframe=black!50,
    colbacktitle=black!70,
    fonttitle=\ttfamily\bfseries,
    title={\ifstrempty{#2}{Code}{#2}},
    before upper={\lstset{style=pythonstyle}},
    #1
}

\title{Mathematics Bootcamp\\
       \large A Comprehensive Guide to Pre-Calculus, Geometry, and Calculus}
\author{Mathematics Learning Team}
\date{\today}

\begin{document}

\maketitle
\tableofcontents

% Part I: Pre-Calculus Foundations
\part{Pre-Calculus Foundations}

\include{chapters/chapter_introduction}
\include{chapters/chapter_real_numbers}
\chapter{Introduction to Functions}
\author{Nithin}

\section{Domain,Range and Equality}

\subsection{What is a Function ?}
A function associates every number in some set of real numbers, called the domain of the function, with exactly one real number.

\subsection{Domain}
If a function is defined by a formula, with no domain specified, then the domain is assumed to be the set of all real numbers for which the formula makes sense and produces a real number.

\subsubsection{Example 3}
Find the domain of the function \(f\)  defined by \[f (x) = (3x-1)^2 \]

\subsubsection{Example 4}
Find the domain of the function \(f\) defined by \[h (t) = \frac{t^2 + 3t + 7}{t-4} \]

\subsubsection{Example 6}
Find the domain of the function g defined by \[g(x) = \sqrt{|x|-5}\]

\subsection{Range}
The range of a function \(f\) is the set of all numbers \(y\) such that \(f (x) = y\) for at least one \(x\) in the domain of \(f\).

\subsubsection{Example 4}
The domain of \(f\) is the interval \( [2, 5] \), with \(f\) defined on this interval by the equation \(f (x) = 3x + 1\).
Solution:
\[ y = f(x) = 3x+1 \]
\[ 2 \leq \frac{y-1}{3} \leq 5. \]
\[ 7 \leq y \leq 16. \]

\subsubsection{Example 5}
The domain of \(g\) is the interval \([1,20]\), with \(g\) defined on this interval by the equation
\[ g(x) = |x - 5|. \]
Is \(2\) in the range of \(g\)?
Solution:
\[ y =  |x-5| \]
for \(x-5>0 \),\(y = x-5 \implies x = y+5 \)
\[ 5 < y+5 \leq 20  \implies  0 < y \leq 15 \]
for \(x-5<0 \), \(y = -(x-5) \implies y = -x+5 \implies 5-y = x \)
\[ 1 \leq 5-y \leq 5  \implies -4 \leq -y \leq 0 \implies  4 \geq y \geq 0 \]

\subsection{Equality of Functions}
Two functions are equal if and only if they have the same domain and the same value at every number in that domain.

\subsubsection{Example}
Suppose \(f\) is the function whose domain is the set of real numbers, with \(f\) defined on this domain by
\[f (x) = x^2\]
Suppose \(g\) is the function whose domain is the set of positive numbers, with \(g\) defined on this domain by
\[g(x) = x^2\]
Are \(f\) and \(g\) equal functions?

\subsubsection{Example 2}
Suppose \(f\) and \(g\) are functions whose domain is the set consisting of the two numbers \(\{1, 2\}\) with \(f\) and \(g\) defined on this domain by the formulas
\[f (x) = x^2\] and \(g(x) = 3x-2\).
Are \(f\) and \(g\) equal functions?

\section{Analytical Geometry}
\subsection{What is Analytic Geometry?}
\begin{itemize}
    \item \textbf{Analytic Geometry} (also called \textit{coordinate geometry} or \textit{Cartesian geometry}) bridges algebra and geometry.
    \item It uses a coordinate system to study geometric shapes and properties.
    \item Geometric objects are represented as algebraic equations.
\end{itemize}

\subsection{Co-ordinate Plane}
\begin{figure}[h]
    \centering
    \includegraphics[scale=0.23]{cartesian.png}
\end{figure}
The plane with this system of labeling is often called the \textbf{Cartesian plane} in honor of the French mathematician Rene Descartes(1596-1650), who described this technique in his 1637 book Discourse on Method.

\subsection{Graph Functions}
The graph of a function \(f\) is the set of points of the form \(x, f (x)\) as x varies over the domain of \(f\).

\subsubsection{Graphs}
\begin{figure}[h]
  \centering
  \includegraphics[scale=0.3]{graph.png}
\end{figure}
\begin{figure}[h]
  \centering
  \includegraphics[scale=0.3]{graph2.png}
\end{figure}

\subsection{Checking for a function: Vertical line test}
The line \(x=1\) intersects the curve at two points. That is that for each \(x\) value there are multiple \(y\) values which is contradicting to definition of a function.
\subsubsection{Vertical Line Test}
A set of points in the coordinate plane is the graph of some function if and only if every vertical line intersects the set in at most one point.

\section{Average Rate of change}
The rate of change describes how an output quantity changes relative to the change in the input quantity. The unit on a rate of change.
The \textbf{average rate of change} of a function \(f\) over an interval \([x_{1}, x_{2}]\) is given by
\[ =\frac{\Delta y}{\Delta x} =\frac{f(x_{2}) - f(x_{1})}{x_{2} - x_{1}} \]

\subsection{Increasing, Decreasing and Constant Functions}
\begin{figure}
  \centering
  \includegraphics[scale=0.3]{inc-dec.png}
\end{figure}

\subsubsection{Increasing Function}
A function \(f\) is said to be increasing on an interval \((a, b)\) if for any \(x_{1}, x_{2}\) in \((a, b)\) with \(x_{1} < x_{2}\), we have \(f(x_{1}) < f(x_{2})\) or it has a positive rate of change over the interval.

\subsubsection{Decreasing Function}
A function \(f\) is said to be decreasing on an interval \((a, b)\) if for any \(x_{1}, x_{2}\) in \((a, b)\) with \(x_{1} < x_{2}\), we have \(f(x_{1}) > f(x_{2})\) or it has a negative rate of change over the interval.

\subsubsection{Constant Function}
A function \(f\) is said to be constant on an interval \((a, b)\) if for any \(x_{1}, x_{2}\) in \((a, b)\) with \(x_{1} < x_{2}\), we have \(f(x_{1}) = f(x_{2})\) or it has a zero rate of change over the interval.

\subsection{Local Maxima and Local Minima}
\begin{figure}
  \centering
  \includegraphics[scale=0.5]{max-min.png}
\end{figure}

\subsubsection{Local Maximum}
A function \(f\) has a \textbf{local maximum} at \(x = c\) if there exists an interval \((a, b)\) containing \(c\) such that
\[ f(c) \geq f(x) \quad \text{for all } x \in (a, b) \]

\subsubsection{Local Minimum}
A function \(f\) has a \textbf{local minimum} at \(x = c\) if there exists an interval \((a, b)\) containing \(c\) such that
\[ f(c) \leq f(x) \quad \text{for all } x \in (a, b) \]

\section{Function Transformation}
\subsection{Vertical Transformations}
\subsubsection{Shifting a graph up or down}
Suppose \(f \) is a function and \(a > 0\). Upshift \(g\) and  Downshift \(h\) by
\[g(x) = f (x) + a \;\; h(x) = f (x) - a \]
\begin{figure}[h]
  \centering
  \includegraphics[scale=0.4]{vertical_shift.png}
\end{figure}

\subsubsection{Vertical Stretch}
Suppose \(f\) is a function and \(c > 0\). Define a function \(g\) by
\[g(x) = c f (x)\]
\begin{figure}[h]
  \centering
  \includegraphics[scale=0.5]{vertical_stretch.png}
\end{figure}

\subsubsection{Flipping along the Vertical Axis}
Vertical fliiping of \(f(x) \) is
\[g(x) = - f (x) \]
\begin{figure}[h]
  \centering
  \includegraphics[scale=0.5]{vertical_flip.png}
\end{figure}

\subsection{Horizontal Transformation}
\subsubsection{Horizontal Shift}
\[g(x) = f(x+a), h(x) = f(x-a)\]
\begin{figure}[h]
  \centering
  \includegraphics[scale=0.5]{horizontal-shift.png}
\end{figure}

\subsubsection{Horizontal Stretching}
\[g(x) = f(cx)\]
\begin{figure}[h]
  \centering
  \includegraphics[scale=0.55]{horizontal-stretch.png}
\end{figure}

\subsection{Flipping ac the Vertical Axis}
\subsubsection{Horizontal Stretching}
\[g(x) = f(-x)\]
\begin{figure}[h]
  \centering
  \includegraphics[scale=0.55]{vertical-axis-flip.png}
\end{figure}

\subsection{Combinations of vertical Transformation}
\begin{figure}[h]
  \centering
  \includegraphics[scale=0.55]{vertical_combination.png}
\end{figure}

\subsection{Even Functions}
\[ f(-x) = f(x) \quad \text{for all } x \text{ in the domain} \]
Example: \( f(x) = x^2, \quad f(x) = \cos x \)
The graph of an even function is symmetric across the vertical axis.
\begin{figure}
\centering
\includegraphics[width=0.45\linewidth]{even.png}
\includegraphics[width=0.45\linewidth]{even2.png}
\end{figure}

\subsection{Odd Function}
\[ f(-x) = -f(x) \quad \text{for all } x \text{ in the domain} \]
Example: \( f(x) = x^3, \quad f(x) = \sin x \)
The graph of an even function is symmetric if flipped or rotated 18 across the origin.
\begin{figure}
\centering
\includegraphics[width=0.45\linewidth]{odd.png}
\includegraphics[width=0.45\linewidth]{odd2.png}
\end{figure}

\section{Function Composition}
\subsection{Algebra of Functions}
Suppose \(f\) and \(g\) are functions. We can define new functions from \(f\) and \(g\) as follows: \\
\textbf{Sum:}
\[ (f+g)(x) = f(x) + g(x) \]
\textbf{Difference:}
\[ (f-g)(x) = f(x) - g(x) \]
\textbf{Product:}
\[ (f\cdot g)(x) = f(x) \cdot g(x) \]
\textbf{Quotient:}
\[ \left(\frac{f}{g}\right)(x) = \frac{f(x)}{g(x)} \quad \text{provided } g(x) \neq 0. \]
\textbf{Note:} If \(f\) and \(g\) have domains \(D_f\) and \(D_g\), then these operations are defined on the intersection \(D_f \cap D_g\). In the case of the quotient, it is defined on
\[ \{x \in D_f \cap D_g : g(x) \neq 0\}. \]

\subsection{Exercise}
\( f(x)  = \sqrt{x-3}\) and \(g(x) = \sqrt{8 - x } \)
Evaluate
\begin{enumerate}
  \item[a.] \((f+g)(x)\)
  \item[b.]  \((fg)(x)\)
  \item[c.] Find the domain of above
\end{enumerate}
Sol:
\begin{enumerate}
  \item[a.] \(\sqrt{x-3} + \sqrt{8-x} \)
  \item[b.] \(\sqrt{(x-3)(8-x)} \)
  \item[c.]  Domian of \begin{enumerate}
    \item[a.] \(x\geq 3 \)
    \item[b.] \(x \leq 8 \)
    \item[c.] \( 3 \leq x \leq 8 \)
  \end{enumerate}
\end{enumerate}

\subsection{Function Composition}
\subsubsection{Definition:}
If \( f(x) \) and \( g(x) \) are functions, then the composition of \( f \) and \( g \), denoted by \( f \circ g \), is defined by
\[ (f \circ g)(x) = f(g(x)). \]
\textbf{Example:}
Consider the function
\[ h(x) = \sqrt{x+3}. \]
We can express \( h(x) \) as a composition of two functions \( f \) and \( g \) where:
\[ f(x) = \sqrt{x} \quad \text{and} \quad g(x) = x+3. \]
Then,
\[ h(x) = f(g(x)) = f(x+3) = \sqrt{x+3}. \]

\subsection{Exercise}
\( f(x) = \frac{1}{x-4} \) and \( g(x) = x^{2} \)
\begin{enumerate}
  \item \(f \circ g \)
  \item \( g \circ f \)
  \item domian of \(f \circ g \)
  \item domain of \( g \circ f \)
\end{enumerate}
Sol:
\begin{enumerate}
  \item  \(f(g(x)) = \frac{1}{x^{2} - 4}\)
  \item \(g(f(x)) = (\frac{1}{x-4})^{2}\)
  \item \(R - \{-2,2\} \)
  \item \(R - \{4\} \)
\end{enumerate}

\subsection{Composition Machine}
\begin{figure}
  \centering
  \includegraphics[scale=0.3]{composition.png}
\end{figure}

\subsection{Excersise}
\textbf{Problem:} Suppose your cell phone company charges \(\$0.05\) per minute plus \(\$0.47\) for each call to China.
\begin{enumerate}
  \item[(a)] Find a function \(p\) that gives the amount charged by your cell phone company for a call to China as a function of the number of minutes \(m\).
  \item[(b)] Suppose the tax on cell phone bills is 6\% plus \(\$0.01\) for each call. Find a function \(t\) that gives your total cost, including tax, for a call to China as a function of the amount charged by your cell phone company.
  \item[(c)] Explain why the composition \(t \circ p\) gives your total cost, including tax, of making a cell phone call to China as a function of the number of minutes.
  \item[(d)] Compute a formula for \(t \circ p\).
  \item[(e)] What is your total cost for a ten-minute call to China?
\end{enumerate}

\subsection{Solution}
\textbf{(a)} The company charges \(\$0.05\) per minute plus a fixed charge of \(\$0.47\) per call. Hence, the pre-tax charge function is
\[ p(m)=0.05m+0.47. \]
\textbf{(b)} The tax on the cell phone bill is 6\% of the pre-tax amount plus an additional \(\$0.01\) per call. Thus, if the pre-tax charge is \(x\), the total cost function (including tax) is
\[ t(x)=1.06x+0.01. \]
\textbf{(c)} The composition \(t \circ p\) means we first compute the pre-tax charge \(p(m)\) for a call of \(m\) minutes, and then we apply the tax function \(t\) to this amount. In other words, \(t(p(m))\) gives the total cost, including tax, as a function of the number of minutes.
\textbf{(d)} To compute the composition, substitute \(p(m)\) into \(t\):
\[ (t \circ p)(m)=t(p(m))=1.06\bigl(0.05m+0.47\bigr)+0.01. \]
Distribute \(1.06\):
\[ 1.06(0.05m)=0.053m \quad \text{and} \quad 1.06(0.47)=0.4982. \]
Thus,
\[ (t \circ p)(m)=0.053m+0.4982+0.01=0.053m+0.5082. \]
\textbf{(e)} For a ten-minute call (\(m=10\)):
\[ (t \circ p)(10)=0.053(10)+0.5082=0.53+0.5082=1.0382. \]
Rounded to the nearest cent, the total cost is approximately \(\$1.04\).

\subsection{Identity Function}
The identity function is defined by
\[ I(x) = x \quad \text{for every number } x. \]
The function \(I\) is the identity for composition. If \(f\) is any function, then
\[ f \circ I = I \circ f = f. \]

\subsection{Decomposing the Functions}
\subsubsection{Function Decomposition}
\[ T(y) = \frac{|y^{2} -3|}{|y^{2} - 7|} \]
Sol:
\[f(y) = |y|, g(y) = \frac{y^{2}-3}{y^{2}-7} \]
\[ f(y) = \frac{|y-3|}{|y-7|}, g(y) = y^{2}\]
Composition is associative if \(f,g, h\) are functions then
\[(f \circ g) \circ h  = f \circ (g \circ h) \]

\subsection{Example}
\subsubsection{Composition of three functions}
\[T(x) = \left| \frac{x^{2}-3}{x^{2} - 7}  \right| \]
Sol:
\[f(x) = |x|, g(x) = \frac{x-3}{x-7}, h(x) = x^{2}\]

\subsection{Linear Functions}
A linear function is a function \(h\) of the form
\[h(x) = mx + b\]
where \(m\) and \(b\) are numbers.

\subsection{Linear Functions as Composition}
\subsubsection{Vertical Transformations as Compositions}
A funtion \(g(x)\) is defined by
\[g(x)= -2f(x)+1\]
Write \(g(x)\) as a the composition of a linear function with \(f(x)\)
\[h(x) = -2x + 1 \]
\[\implies g(x) = h(f(x)) \implies g = h \circ f \]

\subsubsection{Horizontal Transformations as Compositions}
A funtion \(g(x)\) is defined by
\[g(x)= f(2x)+1\]
Write \(g(x)\) as a the composition of a linear function,  \(f(x)\) and other linear function
\[h(x) = x + 1, p(x) = 2x \]
\[\implies g(x) = h(f(p(x))) \implies g = h \circ f \circ p \]

\section{Inverse Functions}
\subsection{Inverse Function: Example}
Consider the function \( f: \mathbb{R} \to \mathbb{R} \) defined by
\[ y = \frac{9}{5}x + 32, \]
which converts a temperature \( x \) in Celsius to Fahrenheit \( y \). 
The inverse function \( f^{-1} \) converts Fahrenheit back to Celsius:
\[ f^{-1}(y) = \frac{5}{9}(y - 32). \]
Verifying that these functions are inverses:
\[ f^{-1}(f(x)) = \frac{5}{9}\Bigl(\frac{9}{5}x + 32 - 32\Bigr) = x, \]
\[ f(f^{-1}(f)) = \frac{9}{5}\Bigl(\frac{5}{9}(y - 32)\Bigr) + 32 = y. \]

\subsection{One-to-One Function}
A function \(f\) is called one-to-one if for each number \(y\) in the range of \(f\) there is
exactly one number \(x\) in the domain of \(f\) such that \(f (x) = y\).

\subsection{Inverse Function}
\subsubsection{Definition}
Suppose \( f \) is a one-to-one function.
\begin{itemize}
  \item If \( y \) is in the range of \( f \), then \( f^{-1}(y) \) is defined to be the number \( x \) such that \( f(x) = y \).
  \item The function \( f^{-1} \) is called the \emph{inverse function} of \( f \).
\end{itemize}
\textbf{Short version:}
\begin{itemize}
  \item \( f^{-1}(y) = x \) means \( f(x) = y \).
\end{itemize}

\begin{figure}
  \centering
  \includegraphics[scale=0.4]{inverse.jpeg}
\end{figure}

\subsection{Domain and Range of an Inverse Function}
If \( f \) is a one-to-one function, then:
\begin{itemize}
  \item The domain of \( f^{-1} \) equals the range of \( f \).
  \item The range of \( f^{-1} \) equals the domain of \( f \).
\end{itemize}

\subsection{Increasing and Decreasing Function}
\subsubsection{Increasing}
A function \(f\) is called increasing if \(f (a) < f (b)\) whenever \(a < b\) and \(a, b \) are in
the domain of \(f\).

\subsubsection{Decreasing}
A function \(f\) is called decreasing if \(f (a) >  f (b)\) whenever \(a < b\) and \(a, b \) are in
the domain of \(f\).

\subsubsection{Increasing and decreasing functions are one-to-one}
\begin{itemize}
  \item Every increasing function is one-to-one
  \item Every decreasing function is one-to-one.
\end{itemize}

\subsection{Exercise}
\begin{figure}
  \centering
  \includegraphics[scale=0.2]{increase.jpeg}
\end{figure}
\begin{enumerate}
  \item[a.] Decreasing
  \item[b.] Increasing
  \item[c.] Neither
\end{enumerate}

\subsection{Do all one-to-one maps are increasing or decreasing ?}
% \begin{figure}
%   \centering
%   \includegraphics[scale=0.6]{non-increse-decrese.png}
% \end{figure}

\subsection{Increasing and Decreasing Functions}
\subsubsection{Inverses of increasing and decreasing functions}
\begin{itemize}
  \item The inverse of an increasing function is increasing.
  \item The inverse of a decreasing function is decreasing.
\end{itemize}

\chapter{Linear,Quadratic, Polynomial and Rational Functions}
\author{Nithin}

\section{Lines and Linear Function}
\subsection{Slope}
\begin{figure}
  \centering
  \includegraphics[scale=0.25]{slope.png}
\end{figure}
\[\frac{y_{2} - y_{1}}{x_{2} - x_{1}} = \frac{y_{4} - y_{3}}{x_{4} - x_{3}}\]

If \(x_{1},y_{1}\) and \(x_{2},y_{2}\) are any two points on a line with \(x_{1} \neq x_{2}\), then the \textbf{slope} of the line is
\[\frac{y_{2} - y_{1}}{x_{2} - x_{1}}\]

\begin{figure}
  \centering
  \includegraphics[width=0.5\linewidth]{slope2.png}
\end{figure}
\textbf{Key Points:}
\begin{itemize}
    \item Positive slope slands up from left to right
    \item Negative slope slands down from left to right
    \item Horizontal line has slope = 0
    \item Vertical line has no slope
\end{itemize}

\subsection{Line Equation}
\subsubsection{Slope and one point on it}
The line in the xy-plane that has slope \(m\) and contains the point \((x1, y1)\) is given by the equation
\[y - y_{1} = m(x  -  x_{1})\]
\begin{figure}
  \centering
  \includegraphics[width=0.5\linewidth]{line1.png}
\end{figure}

\subsubsection{Slope and \(y\) intercept}
The line in the xy-plane with slope \(m\) that intersects the \(y\) axis at \(0,b\) is given by the equation
\[y = mx+b\]
\begin{figure}
  \centering
  \includegraphics[width=0.5\linewidth]{line2.png}
\end{figure}

The line in the xy-plane that contains the points \(x_{1},y_{1}\) and \(x_{2}, y_{2}\) where \(x_{1} \neq x_{2}\), is
\[y-y_{1} = \left( \frac{y_{2} - y_{1}}{x_{2} - x_{1}} \right) (x-x_{1})\]

\subsection{Linear Function}
A \textbf{linear function} is a function \(f\) of the form
\[f(x) = mx + b\]
where \(m\) and \(b\) are numbers.

\subsubsection{Linear Functions: Origin vs Y-Intercept}
\textbf{Example 1: Temperature Conversion}
\begin{itemize}
    \item Correct formula: \( F = 1.8C + 32 \) (Starts at 32°F)
    \item Incorrect direct proportion: \( F' = 1.8C \) (Wrong assumption)
\end{itemize}
\textbf{Example 2: Weight Conversion}
\begin{itemize}
    \item True conversion: \( lb = 2.205 \times kg \) (Passes through origin)
    \item Shipping charge model: \( lb' = 5 + 2.205 \times kg \) (Has minimum billable weight or fixed cost markup)
\end{itemize}
\begin{figure}
\centering
\includegraphics[width=0.45\linewidth]{Celsius to Fahrenheit Comparison.png}
\includegraphics[width=0.45\linewidth]{Kilograms to Pounds Comparison.png}
\end{figure}

\subsection{Constant Function}
A constant function is a function \(f\) of the form \(f (x) = b\),  where \(b\) is a number.
\begin{figure}
\centering
\includegraphics[width=0.6\linewidth]{constant.png}
\end{figure}

\subsection{Parallel Lines}
\begin{figure}
  \centering
  \includegraphics[scale=0.2]{parallel.png}
\end{figure}
As two lines are parallel, the corresponding angles are concurent and so two triangles are similar so
\[\frac{a}{c} = \frac{b}{d} \implies \frac{b}{a} = \frac{d}{c}  \]
it has same slope.

\subsection{Negative Slope}
\begin{figure}
  \centering
  \includegraphics[scale=0.3]{negative-slope.png}
\end{figure}
As lengths are positive \(a = x_{2} - x_{1}\) and \(c = y_{1} - y_{2} \)
Slope  = \(\frac{y_{2} - y_{1}}{x_{2} - x_{1}} = -\frac{c}{a} \

\subsection{Perpendicular Lines}
\begin{figure}
  \centering
  \includegraphics[width=0.5\linewidth]{perpendicular.png}
\end{figure}
\(\triangle PSQ \) and \( \triangle TSP \) are similar
\(\frac{QS}{SP} = \frac{PS}{ST} \implies \frac{b}{a} = \frac{a}{c} \
Multiplying by \(-\frac{c}{a} \implies \frac{b}{a} \cdot \left( - \frac{c}{a} \right) = -1  \

\subsection{Unequal Scales}
Angles are distorted by unequal scales on coordinate axes. In graphs with unequal scales on the two coordinate axes, angles are not accurately represented.
\begin{figure}
  \centering
  \includegraphics[scale=0.25]{unequal-scale.png}
\end{figure}

\section{Quadratic Functions and Conics}
\subsection{Conics}
\begin{figure}
  \begin{center}
    \includegraphics[scale = 0.3]{conic-section.png}
  \end{center}
\end{figure}

\subsection{Quadratic Function}
The function of the form
\[ax^{2} + bx + c = 0\]
where \(a,b,c\) are real numbers with \(a \neq 0\)
\begin{itemize}
    \item if \(b^{2} - 4ac < 0\), then equation have no real solutions
    \item if \(b^{2} - 4ac = 0\), then equation has one solution, \(x = -\frac{b}{2a}\)
    \item if \(b^{2} - 4ac  > 0\), then equation has two solutions \(x = \frac{-b \underset{-}{+}\sqrt{b^{2} - 4ac}}{2a}
\end{itemize}

\subsection{Parabola}
A \textbf{parabola} is the graph of a quadratic function. The \textbf{vertex} of the parabola is the where the line of symmetry of the parabola, intersects the parabola.
Suppose \(f\) is a quadratic function. Complete the square to write \(f\) in the form
\[ f(x) = a(x-h)^{2} + k \
\begin{itemize}
    \item If \(a > 0\) then \(f(x)\) attains its minimum value \(k\) when \(x=h\) and the graph of \(f\) is a parabola that opens upward.
    \item If \(a < 0 \) then \(f(x) \) its maximum value \(k\) when \(x=h\) and the graph of \(f\) is a parabola that opens downward
    \item The vertex of the graph is \(h,k\)
\end{itemize}

\subsubsection{Example}
\[f(x) = -3x^{2} + 12x - 8 \
\begin{enumerate}
  \item For what value of \(x\) does \(f (x) \) attain its maximum value?
  \item What is the maximum value of \(f (x)\)?
  \item Find the vertex
\end{enumerate}
Sol:
\[f(x) = -3x^{2} + 12x - 8 \implies -3(x^{2} - 4x + 4) + 4 \implies -3(x-2)^{2} + 4 \
\begin{enumerate}
\item \(x = 2\
\item \(f(x=2) = 4\
\item \( (2,4) \
\end{enumerate}

\begin{figure}
\centering
\includegraphics[scale=0.6]{parabola.png}
\end{figure}

\subsection{Distance Between Points}
\begin{figure}
\centering
\includegraphics[scale=0.3]{distance-between-points.png}
\end{figure}
The distance between points \(x_{1}, y_{1}\) and \(x_{2}, y_{2}\) is given by
\[\sqrt{ \left(x_{2} - x_{1} \right)^{2}  + \left( y_{2} - y_{1} \right)^{2} }\]

\subsection{Circle}
The circle with center \(h,k\) and radius \(r\) is the set of the points \(x,y\) that satisfy the equation
\[\left(x-h\right)^{2} + \left(y-k\right)^{2} = r^{2}\]

\subsection{Ellipse}
\begin{figure}
\centering
\includegraphics[scale=0.38]{ellipse0.png}
\end{figure}

Stretching the circle horizontally and/or vertically produces a curve called an \textbf{ellipse}.
\begin{figure}
\centering
\includegraphics[scale=0.4]{ellipse1.png}
\end{figure}

\subsubsection{Example}
Equation of the circle is given by \(u^{2} + v^{2} = 1\).
By stretching \(x = 3u, y = 5v\),
Substituting for \(u,v\)
\[ \left( \frac{x}{3} \right)^{2} + \left(\frac{y}{5}\right)^{2} = 1 \

\subsubsection{Ellipse Equation}
\[\frac{x{2}}{a^{2}} + \frac{y^{2}}{b^{2}} = 1 \
The \textbf{foci} of an ellispe are two points with the property that the sum of the distances from the \textbf{foci} to any point on the ellipse is a constant independent of the point on the ellispe.

\begin{figure}
\centering
\includegraphics[scale=0.5]{ellipse1.png}
\end{figure}

\begin{figure}
\centering
\includegraphics[scale=0.4]{foci.png}
\end{figure}

\subsection{Eccentricity of an Ellipse}
The \textbf{eccentricity} (\(e\)) of an ellipse is a measure of how much the ellipse deviates from being a circle. It is defined as
\[ e = \frac{c}{a} .
\qquad c^2 = a^2 - b^2, 
\qquad e = \sqrt{1 - \frac{b^2}{a^2}}\]
where:
\begin{itemize}
    \item \(c\) is the distance from the center to a focus.
    \item \(a\) is the length of the semi-major axis.
\end{itemize}
Additionally, the semi-minor axis \(b\) is related to \(a\) and \(c\) by:
\textbf{Key Points:}
\begin{itemize}
    \item If \(e = 0\), the ellipse is a circle.
    \item If \(0 < e < 1\), the ellipse is elongated, with greater elongation as \(e\) increases.
\end{itemize}

\subsection{Hyperbola}
The graph of the equation of the form
\[ \frac{y^2}{b^2} - \frac{x^2}{a^2} = 1 \
where \(a,b\) are non-zero numbers.
\begin{figure}
\centering
\includegraphics[scale=0.5]{hyperbola2.png}
\end{figure}

The foci of a hyperbola are two points with the property that the difference of the distances from the foci to a point on the hyperbola is a constant independent of the point on the hyperbola.
\begin{figure}
\centering
\includegraphics[scale=0.4]{hyperbola3.png}
\end{figure}
\begin{figure}
\centering
\includegraphics[scale=0.5]{hyperbola1.png}
\end{figure}

\section{Exponents}
\subsection{Positive Integer Exponent}
If \(x\) is a real number and \(m\) is a positive integer, then \(x^{m}\) is defined to be the product with \(x\) appearing \(m\) times
\[x^{m} = \underset{x \;	ext{appears}\;	ext{m} \;	ext{times}}{\underbrace{x \cdot x \cdots x }}\]

\subsubsection{Properties}
Suppose \(x\) and \(y\) are numbers and \(m\) and \(n\) are positive integers. Then
\[x^{m}x^{n} = x^{m+n} \
\left(x^{m}\right)^{n} = x^{mn} \
x^{m}y^{m} = \left(xy\right)^{m}\]

\subsection{\(x^{0}\)}
If  \(x^{m}x^{n} = x^{m+n}\) then we can write \(x^{0}x^{n} = x^{0+n} = x^{n} \implies x^{0} = 1\;for\;	ext{x} \neq 0\]

\subsubsection{What is \(0^{0}\)}
\begin{itemize}
  \item The rule \( x^0 = 1 \) (for \( x \neq 0 \)) suggests that \( 0^0 \) should be \(1\).
  \item The rule \( 0^m = 0 \) (for \( m > 0 \)) suggests that \( 0^0 \) should be \(0\).
  \item Since these two rules contradict each other, \( 0^0 \) is left undefined in general mathematics.
  \item However, in combinatorics and programming, \( 0^0 \) is often defined as \(1\) for convenience.
\end{itemize}

\subsection{Negative Integer Exponents}
If \(x^{m}x^{n} = x^{m+n}\), if we take \(m = -n \), then
\[x^{m}x^{-m} = x^{0} = 1 \implies x^{m}x^{-m} = 1\]
We have to define \(x^{-m}\) to equal the multiplicative inverse of \(x^{m}\).
If \(x \neq 0\) and \(m\) is a positive integer, then \(x^{-m}\) is defined to multiplicative inverse of \(x^{m}\)
\[x^{-m} = \frac{1}{x^{m}}\]

\subsubsection{Exponents: Some Graphs}
\begin{figure}
\centering
\includegraphics[scale=0.4]{exponent-graph1.png}
\caption{graph of \(\frac{1}{x}\)}
\includegraphics[scale=0.4]{exponent-graph2.png}
\caption{graph of \(\frac{1}{x^{2}}\)}
\end{figure}

\subsubsection{Graph of Negative Integer Exponents}
if \(m\) is  a positive integer then
\begin{itemize}
  \item \(\frac{1}{x^{m}}\) behaves like \(\frac{1}{x}\) if \(m\) is odd
  \item \(\frac{1}{x^{m}}\) behaves like \(\frac{1}{x^{2}}\) if \(m\) is even
  \item Larger values of \(m\) correspond to functions whose graphs get closer to the x-axis more rapidly for large values of \(x\) and closer to the vertical axis more rapidly for values of \(x\) near 0
\end{itemize}

\subsection{Roots}
\subsubsection{\(m^{th}\) root}
If \(m\) is a positive integer and \(x\) is a real number, then \(x^{m}\) is defined to be the real number satisfying the equation
\[\left(x^{m}\right)^{m} = x \
subject to the following conditions:
\begin{itemize}
\item  If \(x < 0\) and \(m\) is an even integer, then \(x^{m} \)is undefined
\item If \(x > 0\) and \(m\) is an even integer, then \(x^{m} \)is chosen to be the \textit{\textcolor{red}{positive number}} satisfying the equation above
\end{itemize}
The number \(x^{m}\) is called the \textbf{\(m^{th}\)} root of \(x\).

\subsubsection{Example}
\begin{itemize}
  \item \(8^{3}\) and \(-8^{3}\)
  \item \(9^{2}\) and \(-9^{2}\)
\end{itemize}
Solution:
\begin{enumerate}
  \item \( \left(8^{3}\right)^{3} = 8 \implies 2 \
  \item \( \left(-8^{3}\right)^{3} = -8 \implies -2 \). There is no other number other than \(-2\)
  \item \( \left( 9^{2} \right)^{2} = -3 \; or \;3\). But as per the rule, we have to choose positive possibility, that is \(3\
  \item \( \left( -9^{2} \right)^{2} \). No number real number exists so no solution
\end{enumerate}

\subsection{Rational Exponents}
Suppose \(\frac{n}{m}\) is a fraction in reduced form, where \(n\) and \(m\) are integers and \(m > 0\). Then, whenever it makes sense,
\[ x^{\frac{n}{m}} = \Bigl(x^{\frac{1}{m}}\Bigr)^n. \
\textbf{Note:} For the expression \(x^{\frac{1}{m}}\) to be defined, additional conditions on \(x\) may be required (for example, if \(m\) is even, then typically \(x \ge 0\)).

\subsection{Algebra of Exponents}
Let \(p, q\) be rational numbers and \(x, y\) be positive numbers. Then the following rules hold:
\begin{itemize}
  \item \(x^p \cdot x^q = x^{\,p+q}
  \item \(x^p \cdot y^p = (xy)^p
  \item \((x^p)^q = x^{\,pq}
  \item \(x^0 = 1
  \item \(x^{-p} = \dfrac{1}{x^p}
  \item \(\dfrac{x^p}{x^q} = x^{\,p-q}
  \item \(\left(\dfrac{x}{y}\right)^p = \dfrac{x^p}{y^p}
\end{itemize}

\section{Polynomials}
\subsection{Polynomial Definition}
A polynomial is a function \( p \) such that
\[ p(x) = a_0 + a_1 x + a_2 x^2 + \cdots + a_n x^n, \
where \( n \) is a nonnegative integer and \( a_0, a_1, a_2, \dots, a_n \) are numbers.

\subsection{Degree of a Polynomial}
Suppose \( p \) is a polynomial defined by
\[ p(x) = a_0 + a_1 x + a_2 x^2 + \cdots + a_n x^n. \
If \( a_n \neq 0 \), then we say that \( p \) has degree \( n \). The degree of \( p \) is denoted by \(\deg p\).

\subsection{Polynomial Graphs}
\begin{figure}
\centering
\includegraphics[width=0.45\linewidth]{polynomial_degree_0.png}
\includegraphics[width=0.45\linewidth]{polynomial_degree_1.png}
\includegraphics[width=0.45\linewidth]{polynomial_degree_2.png}
\includegraphics[width=0.45\linewidth]{polynomial_degree_4.png}
\includegraphics[width=0.45\linewidth]{polynomial_degree_7.png}
\end{figure}

\subsection{The Algebra of Polynomials}
Two functions can be added, subtracted, or multiplied, producing another function. Specifically, if \(p\) and \(q\) are functions, then the functions
\[ p+q,\quad p-q,\quad \text{and} \quad pq \
are defined by
\[ (p+q)(x) = p(x) + q(x), \
\[ (p-q)(x) = p(x) - q(x), \
\[ (pq)(x) = p(x) \, q(x). \

\subsection{Degree of the Sum and Difference of Two Polynomials}
If \(p\) and \(q\) are nonzero polynomials, then
\[ \deg(p+q) \leq \max\{\deg p,\, \deg q\}, \
and
\[ \deg(p-q) \leq \max\{\deg p,\, \deg q\}. \

\subsection{Degree of the Product of Two Polynomials}
If \(p\) and \(q\) are nonzero polynomials, then
\[ \deg(pq) = \deg p + \deg q. \

\subsection{Example: Polynomials \(p\) and \(q\)}
Suppose \(p\) and \(q\) are polynomials defined by
\[ p(x) = 2 - 3x^2 \quad \text{and} \quad q(x) = 4x + 7x^5. \
Answer the following:
\begin{enumerate}
    \item What is \(\deg p\)?
    \item What is \(\deg q\)?
    \item Find a formula for \(pq\).
    \item What is \(\deg(pq)\)?
\end{enumerate}
Solution:
\begin{enumerate}
    \item Since \(p(x) = 2 - 3x^2\) has the highest power \(x^2\), we have \(\deg p = 2\).
    \item For \(q(x) = 4x + 7x^5\), the highest power is \(x^5\), so \(\deg q = 5\).
    \item The product \(pq\) is computed as follows:
    \[ pq = (2-3x^2)(4x+7x^5) = 2\cdot 4x + 2\cdot 7x^5 - 3x^2\cdot 4x - 3x^2\cdot 7x^5, \
    which simplifies to:
    \[ pq = 8x - 12x^3 + 14x^5 - 21x^7. \
    \item The highest power in \(pq\) is \(x^7\), so \(\deg(pq) = 7\).
\end{enumerate}

\subsection{Roots of a Function}
A number \(t\) is called a zero of a function \(p\) if
\[ p(t) = 0. \

\subsection{Closed-Form Expression}
A closed-form expression is an explicit formula that can be written using a finite number of standard operations and functions (e.g., addition, multiplication, exponentiation, logarithms, trigonometric functions). It does not involve infinite series, integrals, or iterative processes.
\textbf{Example:} The quadratic formula,
\[ x = \frac{-b \pm \sqrt{b^2-4ac}}{2a}, \
is a closed-form expression.

\subsection{Zeros of Higher–Degree Polynomials}
\begin{itemize}
  \item The quadratic formula gives exact zeros for degree–2 polynomials.
  \item Although formulas exist for cubics and quartics, they are rarely used.
  \item No closed-form formula exists for polynomials of degree 5 or higher.
  \item Numerical methods can approximate zeros for any polynomial.
  \item \textbf{Example:} For
  \[ p(x)=x^5-5x^4-6x^3+17x^2+4x-7, \
  approximate zeros are:
  \[ -1.80,\; -0.73,\; 0.63,\; 1.48,\; 5.56. \
\end{itemize}

\subsubsection{Zeros on graph}
\begin{figure}
\centering
\includegraphics[scale=0.23]{zeros.png}
\end{figure}
\begin{figure}
\centering
\includegraphics[scale=0.3]{no-real-zeros.png}
\caption{The function \(x^{2}+1\) has no real zeros}
\end{figure}

\subsection{Factor of a Polynomial}
Suppose \(p\) is a polynomial and \(t\) is a real number. Then \(x-t\) is called a \emph{factor} of \(p(x)\) if there exists a polynomial \(G(x)\) such that
\[ p(x) = (x-t)\,G(x) \
for every real number \(x\).

\subsection{Example: Factors and Zeros of a Polynomial}
Let
\[ p(x) = (x-2)(x-5)(x^2+1). \
\begin{enumerate}[(a)]
  \item Explain why \(x-2\) is a factor of \(p(x)\).
  \item Explain why \(x-5\) is a factor of \(p(x)\).
  \item Show that \(2\) and \(5\) are zeros of \(p\).
  \item Show that \(p\) has no (real) zeros except \(2\) and \(5\).
\end{enumerate}
Solution:
\begin{enumerate}[(a)]
  \item The polynomial is given in factored form as \((x-2)(x-5)(x^2+1)\). Since \((x-2)\) appears as one of the factors, it is a factor of \(p(x)\).
  \item Similarly, \((x-5)\) appears explicitly in the factorization, so it is a factor of \(p(x)\).
  \item To show that \(2\) and \(5\) are zeros, substitute:
    \[ p(2) = (2-2)(2-5)(2^2+1) = 0\cdot(-3)\cdot5 = 0, \
    \[ p(5) = (5-2)(5-5)(5^2+1) = 3\cdot0\cdot26 = 0. \
    Thus, \(p(2)=0\) and \(p(5)=0\).
  \item The factor \(x^2+1\) yields \(x^2=-1\), which has no real solutions. Hence, aside from the zeros from \((x-2)\) and \((x-5)\), there are no other real zeros.
\end{enumerate}

\subsection{Zeros and Factors of a Polynomial}
Suppose \(p\) is a polynomial and \(t\) is a real number. Then \(t\) is a zero of \(p\) if and only if \(x-t\) is a factor of \(p(x)\).

\subsection{Number of Zeros of a Polynomial}
A nonzero polynomial \(p(x)\) of degree \(n\) has at most \(n\) zeros.
A polynomial of degree 15 has at most 15 zeros. This is because each (real) zero \(t_j\) of a polynomial \(p\) corresponds to a factor \((x-t_j)\) in a factorization of the form
\[ p(x) = (x-t_1)(x-t_2)\cdots(x-t_m) \, G(x), \
where \(G(x)\) is a polynomial with no (real) zeros. If \(p(x)\) had more than 15 zeros, then the right-hand side would represent a polynomial of degree higher than 15, contradicting the fact that \(p\) is of degree 15.

\subsection{Behavior of a Polynomial Near \(\pm\infty\)}
\begin{itemize}
  \item \textbf{\(x\) near \(+\infty\):} \(x\) is very large.
  \item \textbf{\(x\) near \(-\infty\):} \(x\) is very negative (i.e., \(|x|\) is very large).
  \item Our goal is to determine whether a polynomial \(p(x)\) is positive or negative in these extremes.
\end{itemize}
\begin{figure}
\centering
\includegraphics[scale=0.2]{at-infty.png}
\end{figure}
To analyze the behavior as \(x\to\pm\infty\), factor out the highest degree term.
If \(c\,x^n\) is the highest degree term in \(p(x)\), then for very large \(|x|\), \(p(x)\) behaves like \(c\,x^n\).

\subsection{Zero in an Interval}
\subsubsection{Intermediate Value Theorem}
Suppose \(p\) is a polynomial and \(a, b \in \mathbb{R}\) with \(a < b\). If \(p(a)\) and \(p(b)\) have opposite signs, then there exists a number \(c \in (a, b)\) such that \(p(c)=0\).

\subsection{Example 7: Zero in an Interval}
Let
\[ p(x) = x^5 + x^2 - 1. \
Explain why \(p\) has a zero in the interval \((0,1)\).
Solution:
Evaluate the polynomial at the endpoints:
\[ p(0) = 0^5 + 0^2 - 1 = -1 \quad \text{and} \quad p(1) = 1^5 + 1^2 - 1 = 1. \
Since \(p(0) < 0\) and \(p(1) > 0\), by the Intermediate Value Theorem, there exists a \(c \in (0,1)\) such that \(p(c)=0\).

\subsection{Zeros for Polynomials with Odd Degree}
Every polynomial with odd degree has at least one real zero.
For a polynomial \(p(x)\) of odd degree, as \(x \to -\infty\), \(p(x)\) tends to \(-\infty\) (or \(+\infty\)) and as \(x \to \infty\), \(p(x)\) tends to \(+\infty\) (or \(-\infty\)).
By the Intermediate Value Theorem, since \(p(x)\) changes sign, there exists at least one real number \(c\) such that \(p(c)=0\).

\section{Rational Function}
\subsection{Definition}
A \emph{rational function} is a function \(r\) defined by
\[ r(x) = \frac{p(x)}{q(x)}, \
where \(p(x)\) and \(q(x)\) are polynomials, with \(q(x) \neq 0\).

\subsection{Domain of a Rational Function & Example for \(r(x)\)}
The domain of a rational function
\[ \frac{p(x)}{q(x)} \
is the set of all real numbers where the expression makes sense. Since division by 0 is undefined, we exclude all zeros of \(q(x)\).

\subsubsection{Example}
\[r(x)=\frac{3x^5+x^4-6x^3-2}{x^2-9}\]
Factor the denominator:
\[ x^2-9=(x-3)(x+3). \
Thus, \(r(x)\) is undefined at \(x=3\) and \(x=-3\). Hence, its domain is
\[ \{x\in\mathbb{R}: x\neq 3 \text{ and } x\neq -3\}. \

\subsubsection{Example}
\[s(x)=\frac{x^3-6x+5}{x^4+9}\]
The denominator \(x^4+9\) is always positive (since \(x^4\ge 0\) and \(9>0\)). Therefore, \(s(x)\) is defined for every real number.
Domain of \(s(x)\) :
\( \mathbb{R} \

\subsection{Advantages of Mixed Representation}
Expressing a rational function as a polynomial plus a proper rational function (one where the numerator's degree is less than the denominator's) is analogous to writing an improper fraction as an integer plus a proper fraction.
\begin{itemize}
  \item \textbf{Simplification:} It makes further operations (such as integration, differentiation, and partial fraction decomposition) easier.
  \item \textbf{Asymptotic Insight:} The polynomial part reveals the behavior of the function as \(x\to\pm\infty\), while the proper fraction (the remainder) tends to zero for large \(|x|\).
  \item \textbf{Clarity:} It separates the "whole" part from the "fractional" part, making the function's structure more transparent.
\end{itemize}

\subsection{Mixed Rational Function Representation}
Express
\[ \frac{x^5 + 6x^3 + 11x + 7}{x^2+4} \
in the form
\[ G(x) + \frac{ax+b}{x^2+4}, \
where \(G(x)\) is a polynomial and \(a, b\) are constants.

\subsection{Procedure for Dividing Polynomials}
\begin{enumerate}
  \item \textbf{Rewrite:} Express the highest-degree term in the numerator as a single term times the denominator, plus the necessary adjustment.
  \item \textbf{Simplify:} Use the rewritten numerator to simplify the quotient.
  \item \textbf{Iterate:} Repeat steps (1) and (2) on the remaining rational function until the degree of the numerator is less than the degree of the denominator or the numerator becomes 0.
\end{enumerate}

\subsection{Mixed Representation Example}
Express
\[ \frac{x^5+6x^3+11x+7}{x^2+4} \
in the form
\[ G(x)+\frac{ax+b}{x^2+4}, \
where \(G(x)\) is a polynomial and \(a, b\) are constants.
\subsubsection{Step 1: Eliminate the \(x^5\) Term}
Notice that
\[ x^5=x^3(x^2+4)-4x^3. \
Therefore,
\[
\begin{aligned}
  x^5+6x^3 &= x^3(x^2+4)-4x^3+6x^3\\[1mm]
           &= x^3(x^2+4)+2x^3.
\end{aligned}
\]
So we can write:
\[ \frac{x^5+6x^3+11x+7}{x^2+4}=x^3+\frac{2x^3+11x+7}{x^2+4}. \

\subsubsection{Step 2: Eliminate the \(2x^3\) Term}
Write
\[ 2x^3=2x(x^2+4)-8x. \
Then,
\[
\begin{aligned}
  2x^3+11x+7 &= 2x(x^2+4)-8x+11x+7\\[1mm]
              &= 2x(x^2+4)+(3x+7).
\end{aligned}
\]
Thus,
\[ \frac{2x^3+11x+7}{x^2+4}=2x+\frac{3x+7}{x^2+4}. \

\subsubsection{Final Mixed Representation}
Combining the steps, we have:
\[ \frac{x^5+6x^3+11x+7}{x^2+4}=x^3+2x+\frac{3x+7}{x^2+4}. \

\subsection{Division Algorithm for Polynomials}
If \(p(x)\) and \(q(x)\) are polynomials with \(q(x) \neq 0\), then there exist unique polynomials \(G(x)\) and \(R(x)\) such that
\[ \frac{p(x)}{q(x)} = G(x) + \frac{R(x)}{q(x)}, \
where either \(R(x)=0\) or \(\deg R < \deg q\). Equivalently, we can write
\[ p(x)=q(x)G(x)+R(x). \

\subsection{Division by \(x-t\) and Zeros of a Polynomial}
Fix a real number \(t\) and let \(q(x)=x-t\). Since \(\deg q=1\), the remainder \(R(x)\) is a constant \(c\)(Because degree \(R < \)degree \(q\)). Thus, there exist a polynomial \(G(x)\) and a constant \(c\) such that
\[ p(x) = (x-t)G(x) + c. \
Evaluating at \(x=t\) yields \(p(t)=c\), so we can rewrite this as
\[ p(x) = (x-t)G(x) + p(t). \
\(t\) is a zero of \(p\) if and only if \(p(t)=0\), which happens precisely when
\[ p(x)=(x-t)G(x). \

\subsection{Behavior of a Rational Function Near \(\pm\infty\)}
To determine the behavior of a rational function as \(x\to\pm\infty\), factor out the highest power of \(x\) in the numerator and the denominator.
Consider
\[ r(x)=\frac{9x^5-2x^3+1}{x^8+x+1}. \
The highest degree in the numerator is \(x^5\) and in the denominator is \(x^8\). Factoring these out yields:
\[ r(x) \sim \frac{9x^5}{x^8}=\frac{9}{x^3}. \
As \(|x|\) becomes very large:
\begin{itemize}
  \item \(r(x)\to 0\) as \(x\to\infty\) (approaching \(0^+\)).
  \item \(r(x)\to 0\) as \(x\to-\infty\) (approaching \(0^-\)).
\end{itemize}

\subsection{Asymptote of a Rational Function}
A line is an asymptote of a graph if the graph becomes and stays arbitrarily close to the line as \(x\) tends to \(\pm\infty\).
Consider
\[ r(x)=\frac{3x^6-9x^4+6}{2x^6+4x+3}. \
Both the numerator and the denominator are degree 6. Therefore, the horizontal asymptote is the ratio of the leading coefficients:
\[ y=\frac{3}{2}. \
\begin{figure}
\centering
\includegraphics[scale=0.2]{asymptote1.png}
\end{figure}

\subsubsection{Example}
\[ r(x) = \frac{4x^{10}-2x^3+3x+15}{2x^6+x^5+1} \
Solution:
\[ \frac{4x^{10}}{2x^6} = 2x^4. \
Thus, as \(|x|\to\infty\), \(r(x)\) behaves like \(x^{4}\). That is, \(r(x)\) is very large and positive when \( x \to \infty\) and \( x \to -\infty\).

\include{chapters/chapter_exponentials_logarithms}
\chapter{Trignometry}
\author{Nithin}

\section{The Unit Circle}
The \textbf{unit circle} is the circle in the Cartesian plane with center at the origin and radius 1, defined by the equation:
\[ x^2 + y^2 = 1 \]

\subsection{Radius corresponding to a positive angle}
\begin{figure}
    \centering
    \includegraphics[scale=0.5]{1.png}
\end{figure}

\subsection{Radius corresponding to a negative angle}
\begin{figure}
    \centering
    \includegraphics[scale=0.5]{2.png}
\end{figure}

\subsection{Positive and Negative Angles}
\begin{itemize}
    \item Angle measurements for a radius on the unit circle are made from the positive horizontal axis.
    \item Positive angles correspond to moving counterclockwise from the positive horizontal axis.
    \item Negative angles correspond to moving clockwise from the positive horizontal axis.
\end{itemize}

\subsection{Angles more than 360 degrees}
A radius of the unit circle corresponding to $\theta$ degrees also corresponds to $\theta + 360n$ degrees for every integer n.
\begin{figure}[h]
    \centering
    \includegraphics[scale=0.4]{3.png}
\end{figure}

\subsection{Length of a Circular Arc}
\begin{figure}
    \centering
    \includegraphics[scale=0.5]{5.png}
\end{figure}
\[ 360^{\circ}   \rightarrow  2 \pi r \implies \theta^{\circ}  \rightarrow \frac{\theta}{360}.2\pi r  = \frac{\theta \pi r}{180}\]

\subsection{Radians}
For example an ant moving around a unit circle would travel a distance of $2\pi$ radians when it completes one full rotation.
Radians are a unit of measurement for angles such that $2\pi$ radians correspond to a rotation through an entire circle.
\[ 360^{\circ} = 2 \pi \text{ radians} \]
\[ \theta ^{\circ}  = \frac{\theta \pi}{180} \text{ radians} \]

\subsection{Arc Length}
If $0 < \theta \leq 2\pi$ , then a circular arc on the unit circle corresponding to $\theta$ radians has length $\theta$.

\subsection{Area of a Sector}
A sector/slice with angle $\theta$ radians inside a circle with radius $r$ has area $\frac{1}{2} \theta r^{2}$.
\begin{figure}[h]
    \centering
    \includegraphics[scale=0.25]{6.png}
\end{figure}
If no unit is specified, angles are assumed to be in radians.

\section{Cosine and Sine}
\begin{itemize}
    \item The \textbf{cosine} of an angle $\theta$ is the x-coordinate of the point on the unit circle corresponding to that angle.
    \item The \textbf{sine} of an angle $\theta$ is the y-coordinate of the point on the unit circle corresponding to that angle.
\end{itemize}
\begin{figure}[h]
    \centering
    \includegraphics[scale=0.22]{7.png}
    \caption{sine and cosine}
\end{figure}

\subsection{The Signs of Sine and Cosine}
\begin{figure}
    \centering
    \includegraphics[scale=0.2]{8.png}
    \caption{Signs of sine and cosine in different quadrants}
\end{figure}

\subsection{Key Equation Connecting Sine and Cosine}
\begin{itemize}
    \item By definition cosine and sine are the x and y coordinates of a point on the unit circle.
    \item The equation of the unit circle is $x^2 + y^2 = 1$.
    \item Therefore, for any angle $\theta$,
\end{itemize}
\[\cos^2(\theta) + \sin^2(\theta) = 1\]

\subsection{The limits of Sine and Cosine}
\begin{itemize}
    \item For each real number $\theta$, there is a radius of the unit circle corresponding to that angle.
    \item The co-ordinates of the end point of the radius are $(\cos(\theta), \sin(\theta))$.
    \item That is this function is defined for all real numbers because theta can take any real value.
    \item The domain of sine and cosine is all real numbers. $\(\mathbb{R}\)$
    \item For unit circle $ \cos \theta^{2} + \sin \theta^{2} = 1 $
    \item Because $ \cos \theta^{2} + \sin \theta^{2} = 1 $ for all $\theta$, the range of both sine and cosine is limited to $[-1, 1]$.
\end{itemize}

\subsection{Domain and Range of Sine and Cosine}
\[ -1 \leq \cos(\theta) \leq 1 \quad \text{and} \quad -1 \leq \sin(\theta) \leq 1 \] for all $\theta \in \mathbb{R}$
\begin{itemize}
    \item Domain of sine and cosine: $\(\mathbb{R}\)$
    \item Range of sine and cosine: $[-1, 1]$
\end{itemize}

\section{Tangent}
The \textbf{tangent} of an angle $\theta$ is defined as the ratio of the sine to the cosine of that angle:
\[\tan(\theta) = \frac{\sin(\theta)}{\cos(\theta)}\]
provided that $\(\cos(\theta) \neq 0\)$.

\subsection{Tangent as Slope}
\begin{figure}
    \centering
    \includegraphics[scale=0.2]{9.png}
    \caption{Tangent as slope of the radius}
\end{figure}
\[ \text{Slope} = \frac{y_{2} - y_{1}}{x_{2} - x_{1}} = \frac{\sin(\theta) - 0}{\cos(\theta) - 0} = \tan(\theta) \]
$\tan \theta$ is the slope of the radius corresponding to angle $\theta$ in the \textbf{unit circle}.
\textbf{Note:} The slope of the radius applies to any circle, not just the unit circle.

\subsection{Sign of Tangent}
\begin{figure}
    \centering
    \includegraphics[scale=0.2]{10.png}
    \caption{Signs of tangent in different quadrants}
\end{figure}

\subsection{Radius of unit circle corresponding to a positive angle}
\begin{figure}
    \centering
    \includegraphics[scale=0.25]{11.png}
    \caption{Radius of unit circle corresponding to angle $\theta$ such that $\tan(\theta) = \frac{1}{2}$}
\end{figure}

\subsection{Domain and Range of Tangent}
\begin{itemize}
    \item Tangent is defined for all angles except those where $\(\cos(\theta) = 0\)$, which occurs at odd multiples of $\frac{\pi}{2}$.
    \item The tangent of an angle is the slope of the corresponding radius in the unit circle.
    \item Every real number is the slope of some radius in the unit circle. So range of tangent is all real numbers.
    \item The domain of tangent is all real numbers except odd multiples of $\frac{\pi}{2}$:
    \[\text{Domain of } \tan(\theta) = \mathbb{R} \setminus \left\{ \theta \mid \theta = \frac{\pi}{2} + n\pi, n \in \mathbb{Z} \right\}\]
    \item The range of tangent is all real numbers:
    \[\text{Range of } \tan(\theta) = \mathbb{R}\]
\end{itemize}

\subsection{Graphing Tangent: Tangent near $\frac{\pi}{2}$}
\begin{figure}
    \centering
    \includegraphics[scale=0.28]{12.png}
\end{figure}

\section{Trigonometry in Right Triangles}
\subsection{Right Triangle Definitions}
For $0<\theta<\frac{\pi}{2}$ in a right triangle:
\begin{itemize}
    \item In a right triangle, the side opposite the right angle is the \textbf{hypotenuse}.
    \item The other two sides are referred to as the \textbf{adjacent} and \textbf{opposite} sides, depending on the angle of interest.
\end{itemize}
\begin{figure}
    \centering
    \includegraphics[scale=0.3]{13.png}
\end{figure}

\subsection{SOH-CAH-TOA}
\begin{itemize}
    \item \textbf{SOH}: $\sin(\theta) = \frac{\text{opposite}}{\text{hypotenuse}}$
    \item \textbf{CAH}: $\cos(\theta) = \frac{\text{adjacent}}{\text{hypotenuse}}$
    \item \textbf{TOA}: $\tan(\theta) = \frac{\text{opposite}}{\text{adjacent}}$
\end{itemize}

\subsection{Trigonometric Identities}
\begin{itemize}
    \item $\sin^2(\theta) + \cos^2(\theta) = 1$
    \item $\tan(\theta) = \frac{\sin(\theta)}{\cos(\theta)}$
    \item $\cot(\theta) = \frac{1}{\tan(\theta)} = \frac{\cos(\theta)}{\sin(\theta)}$
    \item $\sec(\theta) = \frac{1}{\cos(\theta)}$
    \item $\csc(\theta) = \frac{1}{\sin(\theta)}$
\end{itemize}

\subsection{Trigonometric Identities For Negative Angles}
\begin{figure}
    \centering
    \includegraphics[scale=0.4]{14.png}
    \caption{Trigonometric identities in a right triangle}
\end{figure}
\begin{itemize}
    \item $\sin(-\theta) = -\sin(\theta)$
    \item $\cos(-\theta) = \cos(\theta)$
    \item $\tan(-\theta) = -\tan(\theta)$
\end{itemize}

\subsection{Even and Odd Trigonometric Functions}
\begin{itemize}
    \item \textbf{Even Functions:} $\cos(-\theta) = \cos(\theta)$
    \item \textbf{Odd Functions:} $\sin(-\theta) = -\sin(\theta)$, $\tan(-\theta) = -\tan(\theta) $
\end{itemize}

\subsection{Trigonometric Identities with Right Triangles}
\begin{figure}
    \centering
    \includegraphics[scale=0.4]{15.png}
    \caption{Trigonometric identities in a right triangle}
\end{figure}
\begin{itemize}
    \item  $\sin(\pi/2 - \theta) = \cos(\theta)$
    \item $\cos(\pi/2 - \theta) = \sin(\theta)$
    \item $\tan(\pi/2 - \theta) = \frac{1}{\tan(\theta)} = \cot(\theta)$
\end{itemize}

\subsection{Trigonometric Identities involving multiples of $\pi$}
\begin{figure}
    \centering
    \includegraphics[scale=0.5]{16.png}
    \caption{Trigonometric identities involving multiples of $\pi$}
\end{figure}
\begin{itemize}
    \item $\sin(n\pi + \theta) = (-1)^n \sin(\theta)$
    \item $\cos(n\pi + \theta) = (-1)^n \cos(\theta)$
    \item $\tan(n\pi + \theta) = \tan(\theta)$
\end{itemize}
For example, if $n$ is even, $\sin(n\pi + \theta) = \sin(\theta)$ and if $n$ is odd, $\sin(n\pi + \theta) = -\sin(\theta)$.

\section{Trigonometric Algebra and Geometry}
\subsection{Inverse Trigonometric Functions}
\subsubsection{The Arccosine Function}
\begin{itemize}
    \item A function is called one-to-one if it maps distinct inputs to distinct outputs.
    \item The \textbf{cosine function} whose domain is entire real line $\(\mathbb{R}\)$ is not one-to-one because it takes the same value for different angles. For example, $\cos(0) = \cos(2\pi) = 1$.
    \item Thus the cosine function is not invertible. It fails in horixontal line test
    \item To make it invertible, we restrict the domain to $[0, \pi]$ where it is one-to-one.
\end{itemize}
\begin{figure}
    \centering
    \includegraphics[scale=0.3]{18.png}
    \caption{Graph of the cosine function restricted to $[0, \pi]$}
\end{figure}
The \textbf{arccosine function} is the inverse of the cosine function restricted to the interval $[0, \pi]$:
\[ \arccos(x) = \theta \quad \text{if and only if} \quad x = \cos(\theta) \text{ for } 0 \leq \theta \leq \pi \]
\begin{itemize}
    \item \textbf{Domain:} The domain of the arccosine function is $[-1, 1]$ because the cosine function takes values in this interval.
    \item \textbf{Range:} The range of the arccosine function is $[0, \pi]$ because it outputs angles in this interval.
\end{itemize}

\subsubsection{The Arcsine Function}
\begin{figure}
    \centering
    \includegraphics[scale=0.3]{19.png}
    \caption{Graph of the sine function restricted to $[- \frac{\pi}{2}, \frac{\pi}{2}]$}
\end{figure}
\begin{itemize}
    \item The \textbf{sine function} is also not one-to-one over the entire real line because it takes the same value for different angles. For example, $\sin(0) = \sin(\pi) = 0$.
    \item To make it invertible, we restrict the domain to $[- \frac{\pi}{2}, \frac{\pi}{2}]$ where it is one-to-one.
\end{itemize}
The \textbf{arcsine function} is the inverse of the sine function restricted to the interval $[- \frac{\pi}{2}, \frac{\pi}{2}] $:
\[ \arcsin(x) = \theta \quad \text{if and only if} \quad x = \sin(\theta) \text{ for } -\frac{\pi}{2} \leq \theta \leq \frac{\pi}{2} \]
\begin{itemize}
    \item \textbf{Domain:} The domain of the arcsine function is $[-1, 1]$ because the sine function takes values in this interval.
    \item \textbf{Range:} The range of the arcsine function is $[- \frac{\pi}{2}, \frac{\pi}{2}]$ because it outputs angles in this interval.
\end{itemize}

\subsubsection{The Arctangent Function}
\begin{itemize}
    \item The \textbf{tangent function} is not one-to-one over the entire real line because it takes the same value for different angles. For example, $\tan(0) = \tan(\pi) = 0$.
    \item To make it invertible, we restrict the domain to $(- \frac{\pi}{2}, \frac{\pi}{2})$ where it is one-to-one.
\end{itemize}
\begin{figure}
    \centering
    \includegraphics[scale=0.4]{20.png}
    \caption{Graph of the tangent function restricted to $(- \frac{\pi}{2}, \frac{\pi}{2})$}
\end{figure}
The \textbf{arctangent function} is the inverse of the tangent function restricted to the interval $(- \frac{\pi}{2}, \frac{\pi}{2})$:
\[ \arctan(x) = \theta \quad \text{if and only if} \quad x = \tan(\theta) \text{ for } -\frac{\pi}{2} < \theta < \frac{\pi}{2} \]

\subsection{Inverse Trigonometric Identities}
\begin{itemize}
    \item $\arccos(\cos(x)) = x$ for $0 \leq x \leq \pi$
    \item $\arcsin(\sin(x)) = x$ for $-\frac{\pi}{2} \leq x \leq \frac{\pi}{2}$
    \item $\arctan(\tan(x)) = x$ for $-\frac{\pi}{2} < x < \frac{\pi}{2}$
\end{itemize}
\textbf{Note:}
\begin{itemize}
    \item The inverse trigonometric functions return angles in their respective ranges.
    \item For example, $\arccos(0.5) = \frac{\pi}{3}$ because $\cos(\frac{\pi}{3}) = 0.5$ and $\frac{\pi}{3}$ is in the range of the arccosine function.
    \item Similarly, $\arcsin(0.5) = \frac{\pi}{6}$ because $\sin(\frac{\pi}{6}) = 0.5$ and $\frac{\pi}{6}$ is in the range of the arcsine function.
    \item $\arctan(1) = \frac{\pi}{4}$ because $\tan(\frac{\pi}{4}) = 1$ and $\frac{\pi}{4}$ is in the range of the arctangent function.
\end{itemize}

\subsection{Inverse of $f(x)=3+4\cos x$}
Suppose $f(x)=3+4\cos x$, where the domain of $f$ is $[0,\pi]$.
\begin{enumerate}[]
  \item Find a formula for $f^{-1}(y)$.
  \item What is the domain of $f^{-1}$?
  \item What is the range of $f^{-1}$?
\end{enumerate}
Solution:
\begin{itemize}
  \item $f^{-1}(y)=\arccos\left(\frac{y-3}{4}\right)$.
  \item $\operatorname{dom}(f^{-1})=[-1,7]$.
  \item $\operatorname{range}(f^{-1})=[0,\pi]$.
\end{itemize}

\subsection{Find the $\arccos(-t)$}
\begin{figure}
    \centering
    \includegraphics[scale=0.4]{21.png}
    \caption{Graph of the arccosine function}
\end{figure}
\textbf{Solution:}
\[ x = -t \implies \arccos(-t) = \pi - \arccos(t) \]

\subsection{Find the $\arcsin(t)$}
\begin{figure}
    \centering
    \includegraphics[scale=0.4]{22.png}
    \caption{Graph of the arcsine function}
\end{figure}
\textbf{Solution:}
\[ y = -t \implies \arcsin(-t) = -\sin(\theta) \]

\subsection{Inverse trigonometric identities for $-t$}
\begin{itemize}
    \item $ \cos^{-1}(-t) = \pi - \cos^{-1}(t) $
    \item $ \sin^{-1}(-t) = - \sin^{-1}(t) $
    \item $ \tan^{-1}(-t) = -\tan^{-1}(t) $
\end{itemize}

\subsection{Find the $\arctan(-t)$}
\begin{figure}
    \centering
    \includegraphics[scale=0.4]{23.png}
    \caption{Graph of the arctangent function}
\end{figure}

\subsection{arcsin plus arccosine}
\[ \arcsin(t) + \arccos(t) = \frac{\pi}{2} \]
\begin{itemize}
    \item This identity holds for all $t$ in the domain of both functions, which is $[-1, 1]$.
    \item The identity reflects the complementary nature of the sine and cosine functions.
    \item Geometrically, it represents the relationship between the angles in a right triangle where one angle is $\arcsin(t)$ and the other is $\arccos(t)$.
\end{itemize}
\begin{figure}
    \centering
    \includegraphics[scale=0.4]{24.png}
    \caption{right angle triangle involving arcsine and arccosine functions}
\end{figure}

\section{Trignometry to Compute Area}
\subsection{Area of a Triangle}
The area $A$ of a triangle with base $b$ and height $h$ is given by:
\[ A = \frac{1}{2} b h = \frac{1}{2} b c \sin(\theta) \]
where $\theta$ is the angle between the two sides of length $b$ and $c$.
\begin{figure}
    \centering
    \includegraphics[scale=0.4]{25.png}
    \caption{Area of a triangle using sine function}
\end{figure}

\subsection{Ambiguous Angles}
\begin{figure}
    \centering
    \includegraphics[scale=0.3]{26.png}
    \caption{$\sin(\pi - \theta) = \sin(\theta)$}
\end{figure}
\begin{itemize}
    \item  $A = \frac{1}{2} b c \sin(\theta) \implies \theta = \arcsin\left(\frac{2A}{bc}\right) $
    \item Assume $ \frac{2A}{bc} = \frac{1}{2} $
    \item Then $ \theta = \arcsin\left(\frac{1}{2}\right) = \frac{\pi}{6} $ or $ \frac{5\pi}{6} $
    \item That is given any number $t \in [-1,1]$ there are two possible angles $\theta$ that satisfy the equation.
    \item $\sin (\pi - \theta) = \sin(-(\theta - \pi )) = - \sin(\theta - \pi) = -(-\sin \theta) = \sin \theta $
\end{itemize}

\subsection{Area of a parallelogram}
The area $A$ of a parallelogram with base $b$ and height $h$ is given by:
\[ A = b h \]
Alternatively, if we know the lengths of two sides $a$ and $b$ and the angle $\theta$ between them, we can use:
\[ A =  bc \sin(\theta) \]
\begin{figure}
    \centering
    \includegraphics[scale=0.3]{27.png}
    \caption{Area of a parallelogram using sine function}
\end{figure}

\subsection{Area of Polygon}
\begin{figure}
    \centering
    \includegraphics[scale=0.3]{28.png}
    \caption{Area of a regular polygon using sine function}
\end{figure}
\begin{itemize}
    \item Area of one triangle = $\frac{1}{2} \cdot 1 \cdot 1 \sin(\frac{2\pi}{8})$
    \item Total Area of polygon = $8 \cdot \frac{1}{2} \cdot 1 \cdot 1 \sin(\frac{2\pi}{8})$
    \item Total area of regular polygon with $n$ sides = $\frac{n}{2} \cdot 1 \cdot 1 \sin(\frac{2\pi}{n})$
\end{itemize}

\subsection{Example: Area of a Regular Pentagon}
Each side of the Pentagon that houses the U.S. Department of Defense has a length of 921 feet.
\begin{figure}
    \centering
    \includegraphics[scale=0.3]{29.png}
    \caption{Area of a regular pentagon}
\end{figure}
Solution:
\begin{itemize}
    \item $h = \frac{460.5}{\tan (\frac{\pi}{5})}$
    \item $\frac{1}{2} \cdot 921 \cdot h$
    \item $5 \cdot \frac{1}{2} \cdot 921 \cdot h$
\end{itemize}

\subsection{Trignometric Approximation}
\begin{table}[h]
    \centering
    \begin{tabular}{|c|c|c|}
        \hline
        $q$ & $\sin q$ & $\tan q$ \\
        \hline
        0.5 & 0.47943 & 0.54630 \\
        0.05 & 0.04998 & 0.05004 \\
        0.005 & 0.00499998 & 0.00500004 \\
        0.0005 & 0.00049999998 & 0.00050000004 \\
        0.00005 & 0.00004999999998 & 0.00005000000004 \\
        \hline
    \end{tabular}
    \caption{Values of $\sin q$ and $\tan q$ for small $q$}
\end{table}
If $| \theta | << 1$, then $\sin \theta \approx \theta$ and $\tan \theta \approx \theta$.
This approximation is applicable only for radians. For small angles in degrees, the approximation is not valid.

\subsection{\(\\sin \theta\) for small angles}
\begin{figure}
    \centering
    \includegraphics[scale=0.4]{30.png}
    \caption{Graph of $\sin \theta$ for small angles}
\end{figure}
\begin{itemize}
    \item The blue vertical line and the blue circular arc have nearly the same length if $|\theta| << 1$.
    \item The blue vertical line represents $\sin \theta$, and the blue circular arc represents $\theta$.
    \item From the figure, we see that $\sin \theta < \theta$ for small angles.
\end{itemize}

\subsection{\(\\tan \theta\) for small angles}
\begin{figure}
    \centering
    \includegraphics[scale=0.4]{31.png}
    \caption{Graph of $\tan \theta$ for small angles}
\end{figure}
\begin{itemize}
    \item The yellow region has the area = $\frac{1}{2} \cdot \theta$
    \item The blue vertical line has the length $\tan \theta$
    \item The right triangle which has the base 1 and height $\tan \theta$ has area = $\frac{1}{2} \cdot 1 \cdot \tan \theta$
    \item As the yellow region lies inside the triangle, we have $\frac{1}{2} \cdot \theta < \frac{1}{2} \cdot 1 \cdot \tan \theta$
    \item Thus $\theta < \tan \theta$
\end{itemize}

\subsection{Trignometric Inequality}
If $ 0 < \theta < \frac{\pi}{2} $, then $\sin \theta < \theta < \tan \theta $.
\begin{displaymath}
    \theta < \tan \theta \implies \theta < \frac{\sin \theta}{\cos \theta} \implies \theta \cos \theta < \sin \theta \implies \cos \theta < \frac{\sin \theta}{\theta}
\end{displaymath}
\begin{displaymath}
    \sin \theta < \theta \implies \frac{\sin \theta}{\theta} < 1
\end{displaymath}
If $ 0 < |\theta| < \frac{\pi}{2} $, then
\[ \cos \theta <  \frac{\sin \theta}{\theta} < 1 \]

\subsection{Approximations}
As $ \theta $ approaches 0, $ \frac{\sin \theta}{\theta} $ approaches 1.
As $ |\theta| $ is small, $ \cos \theta $ approaches to  $1 - \frac{\theta^2}{2} $.
\begin{displaymath}
    1 - \cos \theta = 1 - \cos \theta * (\frac{1 + \cos \theta}{1 + \cos \theta}) = \frac{(1 - \cos^2 \theta)}{1 + \cos \theta} = \frac{\sin^2 \theta}{1 + \cos \theta}
\end{displaymath}
If $ \theta$ is small then :
\begin{displaymath}
  \approx  \frac{\theta ^{2}}{1+ \cos \theta}
\end{displaymath}

\subsection{The Law of Sines and Cosines}
\begin{figure}
    \centering
    \includegraphics[scale=0.5]{32.png}
    \caption{The Law of Sines}
\end{figure}
\begin{itemize}
    \item $ \frac{1}{2} ab \sin C = \frac{1}{2}bc\sin A = \frac{1}{2}ac\sin B$
    \item $ \frac{a}{\sin A} = \frac{b}{\sin B} = \frac{c}{\sin C}$
\end{itemize}
The Law of Sines states that in any triangle, the ratio of the sine of the angle to the length of the opposite side is constant:
\[ \frac{\sin A}{a} = \frac{\sin B}{b} = \frac{\sin C}{c} \]

\subsection{Law of Cosines}
\begin{figure}
    \centering
    \includegraphics[scale=0.3]{33.png}
    \includegraphics[scale=0.3]{34.png}
    \caption{The Law of Cosines}
\end{figure}
\begin{align}
    h &= a\sin C\
    t &= a \cos C \
    r &= b - t = b - a \cos C \
    c^2  &= a^2 \sin^2 C + (b - a \cos C)^2 \
    c^2 &= a^2 + b^2 - 2ab \cos C
\end{align}
The Law of Cosines relates the lengths of the sides of a triangle to the cosine of one of its angles:
\[ c^2 = a^2 + b^2 - 2ab \cos C \]
where $a$, $b$, and $c$ are the lengths of the sides opposite to angles $A$, $B$, and $C$ respectively.

\subsection{When to use which law?}
\begin{itemize}
    \item Use the Law of Sines when you have two angles and one side (AAS or ASA) or two sides and a non-included angle (SSA).
    \item Use the Law of Cosines when you have two sides and the included angle (SAS) or all three sides (SSS).
\end{itemize}

\section{Double Angles and Half Angle Formulas}
\subsection{The $\cos 2\theta$}
\begin{figure}
    \centering
    \includegraphics[scale=0.3]{35.png}
    \includegraphics[scale=0.3]{36.png}
    \caption{The cosine of double angle}
\end{figure}
Applying law of cosines:
\begin{align}
    (2\sin \theta)^2 &= 1^2 + 1^2 - 2\cos(2\theta) \\
    4\sin^2 \theta &= 2 - 2\cos(2\theta) \\
    2\cos(2\theta) &= 2 - 4\sin^2 \theta
\end{align}
The double angle formulas for cosine are:
\[ \cos(2\theta)  = 1 - 2\sin^2(\theta) = 2\cos^2(\theta) - 1= \cos^2(\theta) - \sin^2(\theta) \]

\subsection{The $\sin 2\theta$ Formula}
\begin{figure}
    \centering
    \includegraphics[scale=0.3]{36.png}
    \caption{The sine of double angle}
\end{figure}
Applying law of sines:
\begin{align}
    \frac{\sin(2\theta)}{2\sin \theta} &= \sin(\frac{\pi}{2} - \theta) \\
    \frac{\sin(2\theta)}{2\sin \theta} &= \cos(\theta) \\
    \sin(2\theta) &= 2\sin \theta \cos(\theta)
\end{align}
The double angle formula for sine is:
\[ \sin(2\theta) = 2\sin(\theta)\cos(\theta) \]

\subsection{The $\tan 2\theta$ Formula}
The double angle formula for tangent is:
\begin{align}
    \tan(2\theta) &= \frac{\sin(2\theta)}{\cos(2\theta)} \\
    &= \frac{2\sin(\theta)\cos(\theta)}{\cos^2(\theta) - \sin^2(\theta)} \\
    &=\frac{\frac{2\sin(\theta)\cos(\theta)}{\cos^{2}(\theta)}}{\frac{\cos^{2}\theta - \sin^2\theta}{\cos^2 \theta}} \\
    &= \frac{2\tan(\theta)}{1 - \tan^2(\theta)}
\end{align}

\subsection{The Half Angle Formulas}
The half angle formulas are derived from the double angle formulas:
\begin{align}
    \sin\left(\frac{\theta}{2}\right) &= \sqrt{\frac{1 - \cos(\theta)}{2}} \\
    \cos\left(\frac{\theta}{2}\right) &= \sqrt{\frac{1 + \cos(\theta)}{2}} \\
    \tan\left(\frac{\theta}{2}\right) &= \frac{\sin(\theta)}{1 + \cos(\theta)} = \frac{1 - \cos(\theta)}{\sin(\theta)}
\end{align}
\begin{itemize}
    \item These formulas are useful for simplifying trigonometric expressions and solving equations involving half angles.
    \item The choice of the square root sign depends on the quadrant in which the angle lies.
\end{itemize}

\subsection{The Cosine of a Sum and Difference}
\begin{figure}
    \centering
    \includegraphics[scale=0.4]{37.png}
    \caption{The cosine of a sum and difference}
\end{figure}
\begin{itemize}
    \item The idea is to compute $c^2$ via eucledian distance and then equate the same with the law of cosines.
    \item The distance between the points $(\cos(\alpha), \sin(\alpha))$ and $(\cos(\beta), \sin(\beta))$ is given by:
    \[ c = \sqrt{(\cos(\alpha) - \cos(\beta))^2 + (\sin(\alpha) - \sin(\beta))^2} \]
    \item Squaring both sides gives:
    \[ c^2 = (\cos(\alpha) - \cos(\beta))^2 + (\sin(\alpha) - \sin(\beta))^2 \]
    \item Expanding the squares and using the Pythagorean identity $\sin^2(\theta) + \cos^2(\theta) = 1$, we can derive the cosine of a sum and difference formulas.
    \item The distance $d$ can also be expressed in terms of the angle $\theta = \alpha - \beta$:
    \[ d^2 = 2 - 2\cos(\theta) = 2(1 - \cos(\theta)) \]
    \item Thus, we can relate the cosine of the sum and difference to the distance between the points on the unit circle.
\end{itemize}
The cosine of a sum and difference is given by:
\[ \cos( u \pm v) = \cos(u)\cos(v) \mp \sin(u)\sin(v) \]

\subsection{Addtition formula for Tangent}
The addition formula for tangent is:
\[ \tan(u \pm v) = \frac{\tan(u) \pm \tan(v)}{1 \mp \tan(u)\tan(v)} \]
This formula is derived from the sine and cosine addition formulas.

\section{transformation of Trigonometric Functions}
\subsection{Amplitude}
The \textbf{amplitude} of a function is one-half the distance between the maximum and minimum values of the function.

\subsection{Period}
Suppose $f(x)$ is a periodic function with period $T$. This means that for all $x$:
\[ f(x + T) = f(x) \]
The smallest positive value of $T$ is called the \textbf{period} of the function.

\subsection{Phase Shift}
The \textbf{phase shift} of a function is the horizontal shift of the graph of the function. It is determined by the value of $c$ in the function $f(x) = a \sin(b(x - c)) + d$ or $f(x) = a \cos(b(x - c)) + d$.
\begin{itemize}
    \item If $c > 0$, the graph shifts to the right.
    \item If $c < 0$, the graph shifts to the left.
\end{itemize}

\include{chapters/chapter_polar_complex}
\chapter{Sequences and Series}
\label{sec:sequence-series}

\section{Sequence}
A \textbf{sequence} is an ordered list of numbers, typically defined by a function \( a_n \) where \( n \) is a natural number.
\begin{itemize}
    \item For example, \(7,\, \sqrt{3},\, \frac{5}{2}\) is a sequence. The first term of this sequence is 7, the second term is \(\sqrt{3}\), and the third term is \(\frac{5}{2}\).
    \item Sequences differ from sets in that order matters and repetitions are allowed in a sequence.
    \item For example, \(\{2,\, 3,\, 5\}\) and \(\{5,\, 3,\, 2\}\) are the same set, but the sequences \(2,\, 3,\, 5\) and \(5,\, 3,\, 2\) are not the same.
\end{itemize}

\subsection{Finite and Infinite Sequences}
A \textbf{finite sequence} is a sequence that has a specific number of terms, while an \textbf{infinite sequence} continues indefinitely.
\begin{itemize}
    \item For example, the sequence \(1, 2, 3, 4, 5\) is finite because it has 5 terms.
    \item In contrast, the sequence \(1, 2, 3, \ldots\) is infinite because it goes on forever.
\end{itemize}

\subsection{Examples of Sequences}
\begin{enumerate}
    \item \(a_n = n\) (the sequence of natural numbers) : \(1, 2, 3, 4, \ldots\)
    \item \(a_n = 2n - 1\) (the sequence of odd numbers) : \(1, 3, 5, 7, \ldots\)
    \item \(a_n = (-1)^n\) (the alternating sequence) : \(1, -1, 1, -1, \ldots\)
    \item \(a_n = \frac{1}{n}\) (the sequence of reciprocals) : \(1, \frac{1}{2}, \frac{1}{3}, \frac{1}{4}, \ldots\)
    \item \(a_n = n^2\) (the sequence of squares) : \(1, 4, 9, 16, \ldots\)
    \item \(a_n = 2^n\) (the sequence of powers of 2) : \(2, 4, 8, 16, \ldots\)
\end{enumerate}

\subsection{Examples of Sequences (continued)}
\begin{itemize}
    \item What is the fifth term of the sequence \(1, 4, 9, 16, \ldots\)?
\end{itemize}
Solution:
\begin{align*}
    a_{n} &= \frac{n^{4} - 10n^{3}+ 39n^{2} - 50n + 24}{4} \\
    a_{5} &= \frac{5^{4} - 10 \cdot 5^{3} + 39 \cdot 5^{2} - 50 \cdot 5 + 24}{4} = 31 \\
\end{align*}

\subsection{Arithmetic Sequences}
An \textbf{arithmetic sequence} is a sequence of numbers in which the difference between consecutive terms is constant. This difference is called the \textbf{common difference} and is usually denoted by \(d\).
\begin{itemize}
    \item The general form of an arithmetic sequence can be expressed as:
    \[ a_n = a_1 + (n-1)d \]
    where \(a_1\) is the first term and \(d\) is the common difference.
\end{itemize}
Examples of Arithmetic Sequences:
\begin{itemize}
    \item \(2, 5, 8, 11, \ldots\) (common difference \(d = 3\))
    \item \(10, 7, 4, 1, \ldots\) (common difference \(d = -3\))
    \item \(1, 1.5, 2, 2.5, \ldots\) (common difference \(d = 0.5\))
\end{itemize}
The \(n\)-th term of an arithmetic sequence can be found using the formula:
\[ a_n = a_1 + (n-1)d \]
where:
\begin{itemize}
    \item \(a_n\) is the \(n\)-th term,
    \item \(a_1\) is the first term,
    \item \(d\) is the common difference, and
    \item \(n\) is the term number.
\end{itemize}

\subsection{Geometric Sequences}
A \textbf{geometric sequence} is a sequence of numbers in which the ratio between consecutive terms is constant. This ratio is called the \textbf{common ratio} and is usually denoted by \(r\).
\begin{itemize}
    \item The general form of a geometric sequence can be expressed as:
    \[ a_n = a_1 \cdot r^{n-1} \]
    where \(a_1\) is the first term and \(r\) is the common ratio.
\end{itemize}
Examples of Geometric Sequences:
\begin{itemize}
    \item \(2, 6, 18, 54, \ldots\) (common ratio \(r = 3\))
    \item \(100, 50, 25, 12.5, \ldots\) (common ratio \(r = 0.5\))
    \item \(1, -2, 4, -8, \ldots\) (common ratio \(r = -2\))
\end{itemize}
The \(n\)-th term of a geometric sequence can be found using the formula:
\[ a_n = a_1 \cdot r^{n-1} \]
where:
\begin{itemize}
    \item \(a_n\) is the \(n\)-th term,
    \item \(a_1\) is the first term,
    \item \(r\) is the common ratio, and
    \item \(n\) is the term number.
\end{itemize}

\subsection{Compound Interest Sequence Example}
Suppose at the beginning of the year \$1000 is deposited in a bank account that pays 5\% interest per year, compounded once per year at the end of the year. Consider the sequence whose \(n\)-th term is the amount in the bank account at the beginning of the \(n\)-th year.
Solution:
\begin{enumerate}
    \item[(a)] \textbf{First four terms:}
    \begin{itemize}
        \item The sequence is geometric with first term \(a_1 = 1000\) and common ratio \(r = 1.05\).
        \item The terms are:
        \begin{align*}
            a_1 &= 1000 \\
            a_2 &= 1000 \times 1.05 = 1050 \\
            a_3 &= 1000 \times (1.05)^2 = 1102.5 \\
        \end{align*}
    \end{itemize}
    \item[(b)] \textbf{20th term:}
    \begin{itemize}
        \item The formula for the \(n\)-th term is \(a_n = 1000 \times (1.05)^{n-1}\).
        \item The 20th term is:
        \[ a_{20} = 1000 \times (1.05)^{19} \approx 1000 \times 2.527 \approx 2527 \]
        So, at the beginning of the 20th year, there will be approximately \$2527 in the account.
    \end{itemize}
\end{enumerate}

\subsection{Recursively Defined Sequences}
A \textbf{recursively defined sequence} is a sequence in which each term is defined as a function of one or more of the preceding terms.
\begin{itemize}
    \item A common example is the Fibonacci sequence, defined as follows:
    \[ F_n = F_{n-1} + F_{n-2} \quad \text{for } n \geq 3 \]
    with initial conditions \(F_1 = 1\) and \(F_2 = 1\).
    \item All arithmetic and geometric sequences can also be defined recursively
\end{itemize}
Example of a Recursively Defined Sequence:
\begin{itemize}
    \item \textbf{Fibonacci Sequence:}
    \begin{align*}
        F_1 &= 1 \\
        F_2 &= 1 \\
        F_3 &= F_2 + F_1 = 1 + 1 = 2 \\
        F_4 &= F_3 + F_2 = 2 + 1 = 3 \\
        F_5 &= F_4 + F_3 = 3 + 2 = 5 \\
        F_6 &= F_5 + F_4 = 5 + 3 = 8 \\
        F_n &= F_{n-1} + F_{n-2} \quad \text{for } n \geq 3
    \end{align*}
\end{itemize}

\subsubsection{Newtons Method}
\begin{itemize}
    \item Newton's method for finding roots of a function can be defined recursively:
    \[ x_{n+1} = x_n - \frac{f(x_n)}{f'(x_n)} \]
    where \(x_n\) is the current approximation, \(f(x)\) is the function, and \(f'(x)\) is its derivative.
    \item For estimation \(\sqrt{c}\), we can use the recursive formula:
    \[ a_{n} = \frac{1}{2}(a_{n-1} + \frac{c}{a_{n-1}}) \]
\end{itemize}
Example: To find \(\sqrt{5}\) using Newton's method:
\begin{align*}
    a_{2} &= \frac{1}{2}(a_{1} + \frac{5}{a_{1}}) = \frac{9}{4} \\
    a_{3} &= \frac{1}{2}(a_{2} + \frac{5}{a_{2}})  = \frac{161}{72} \\
    a_{4} &= \frac{1}{2}(a_{3} + \frac{5}{a_{3}}) = \frac{51841}{23184} \\
    &= \frac{51841}{23184} \approx 2.236067977
\end{align*}

\section{Series}
A \textbf{series} is the sum of the terms of a sequence. If the sequence is \(a_1, a_2, a_3, \ldots\), then the series is denoted as:
\[ S_n = a_1 + a_2 + a_3 + \ldots + a_n \]
\begin{itemize}
    \item For example, the series corresponding to the sequence \(1, 2, 3, 4\) is:
    \[ S_4 = 1 + 2 + 3 + 4 = 10 \]
    \item The series can be finite or infinite, depending on whether the sequence has a finite or infinite number of terms.
\end{itemize}

\subsection{Arithmetic Series}
An \textbf{arithmetic series} is the sum of the terms of an arithmetic sequence.
The sum of the first \(n\) terms of an arithmetic series can be calculated using the formula:
\[ S_n = \frac{n}{2} (a_1 + a_n) \]
where \(a_1\) is the first term and \(a_n\) is the \(n\)-th term.
Alternatively, if the common difference \(d\) is known, the formula can be expressed as:
\[ S_n = \frac{n}{2} (2a_1 + (n-1)d) \]

\subsubsection{Sum of first \(n\) positive integers}
\begin{itemize}
    \item The sum of the first \(n\) positive integers can be calculated using the formula:
    \[ S_n = \frac{n(n+1)}{2} \]
    \item This is derived from the arithmetic series formula with \(a_1 = 1\), \(a_n = n\), and \(d = 1\).
    \item For example, the sum of the first 5 positive integers is:
    \[ S_5 = \frac{5(5+1)}{2} = \frac{5 \cdot 6}{2} = 15 \]
\end{itemize}

\subsection{Geometric Series}
A \textbf{geometric series} is the sum of the terms of a geometric sequence.
\begin{itemize}
    \item The sum of the first \(n\) terms of a geometric series can be calculated using the formula:
    \[ S_n = a_1 \frac{1 - r^n}{1 - r} \]
    where \(a_1\) is the first term and \(r\) is the common ratio (and \(r \neq 1\)).
    \item If the series is infinite and \(|r| < 1\), the sum can be calculated using:
    \[ S = \frac{a_1}{1 - r} \]
\end{itemize}

\subsection{Pascals Triangle}
\subsubsection{\( {(x+y)}^{n} \)}
\begin{align*}
{(x+y)}^{0} &= 1 \\
{(x+y)}^{1} &= 1x + 1y \\
{(x+y)}^{2} &= 1x^{2} + 2xy + 1y^{2} \\
{(x+y)}^{3} &= 1x^{3} + 3x^{2}y + 3xy^{2} + 1y^{3}\\
{(x+y)}^{4} &= 1x^{4} + 4x^{3}y + 6x^{2}y^{2} + 4xy^{3} + 1y^{4} \\
{(x+y)}^{5} &= 1x^{5} + 5x^{4}y + 10x^{3}y^{2} + 10x^{2}y^{3} + 5xy^{4} + 1y^{5} \\
\end{align*}
The patterns in the expansions of \((x + y)^n\) can be summarized as follows:
\begin{itemize}
    \item Each expansion of \((x + y)^n\) begins with \(x^n\) and ends with \(y^n\).
    \item The second term in each expansion of \((x + y)^n\) is \(n x^{n-1} y\).
    \item The second-to-last term in each expansion is \(n x y^{n-1}\).
    \item Each term in the expansion of \((x + y)^n\) is a coefficient times \(x^j y^k\), where \(j\) and \(k\) are nonnegative integers such that \(j + k = n\).
    \item In the expansion of \((x + y)^n\), the coefficient of \(x^j y^k\) is the same as the coefficient of \(x^k y^j\).
\end{itemize}
\begin{center}
\begin{tabular}{c}
$n=0$: \quad 1 \\
$n=1$: \quad 1 \quad 1 \\
$n=2$: \quad 1 \quad 2 \quad 1 \\
$n=3$: \quad 1 \quad 3 \quad 3 \quad 1 \\
$n=4$: \quad 1 \quad 4 \quad 6 \quad 4 \quad 1 \\
$n=5$: \quad 1 \quad 5 \quad 10 \quad 10 \quad 5 \quad 1 \\
\end{tabular}
\end{center}
\begin{itemize}
    \item Each row gives the coefficients for $(x+y)^n$
    \item Each number is the sum of the two numbers above it or diagonally left and right
\end{itemize}

\subsection{Binomial Coefficients}
A direct way to compute the coefficients in the expansion of \((x + y)^n\) is to use \textbf{binomial coefficients}, which are denoted as \(\binom{n}{k}\) and read as "n choose k".
The \textbf{binomial coefficient} \(\binom{n}{k}\) is defined as the number of ways to choose \(k\) elements from a set of \(n\) elements, and is given by the formula:
\[ \binom{n}{k} = \frac{n!}{k!(n-k)!} \]
The following identity holds for binomial coefficients:
\[ \binom{n}{k} = \binom{n-1}{k-1} + \binom{n-1}{k} \]

\subsection{Binomial Theorem}
The \textbf{Binomial Theorem} states that:
\[ (x + y)^n = \sum_{k=0}^{n} \binom{n}{k} x^{n-k} y^k \]
where \(\binom{n}{k}\) are the binomial coefficients.

\section{Limits of Sequences}
A sequence has a \textbf{limit} \(L\) if the terms of the sequence get arbitrarily close to \(L\) as \(n\) increases. In other words, for every \(\epsilon > 0\), there exists a number \(N\) such that for all \(n > N\), the absolute difference \(|a_n - L| < \epsilon\).
\textit{Note:} Typically, \(N\) is taken to be a natural number, since sequence indices are natural numbers.
The limit of a sequence \((a_n)\) as \(n\) approaches infinity is denoted as:
\[ \lim_{n \to \infty} a_n = L \]
if the terms \(a_n\) get arbitrarily close to \(L\) for sufficiently large \(n\).

\subsection{Example of a Sequence Limit}
Consider the sequence defined by \(a_n = \frac{1}{n}\).
\begin{itemize}
    \item As \(n\) increases, the terms of the sequence get closer and closer to 0.
    \item Therefore, we can say:
    \[ \lim_{n \to \infty} a_n = 0 \]
\end{itemize}

\subsection{Limit of a Geometirc Sequence}
Suppose \(r\) is a real number. Then the geometric sequence \(r, r^2, r^3, \ldots\)
\begin{itemize}
    \item has limit 0 if \(|r| < 1\);
    \item has limit 1 if \(r = 1\);
    \item does not have a limit if \(r \leq -1\) or \(r > 1\).
\end{itemize}

\subsection{Infinite Series}
An \textbf{infinite series} is the sum of the terms of an infinite sequence. If \((a_n)\) is a sequence, then the infinite series is denoted by:
\[ \sum_{n=1}^{\infty} a_n = a_1 + a_2 + a_3 + \ldots \]
The series converges if the sequence of partial sums converges to a limit.


% Part II: Geometry  
\part{Geometry}

\include{chapters/chapter_geometry_introduction}
\include{chapters/chapter_scalars_vectors}
\section{3D Geometry}
\label{sec:3d_geometry}

\subsection{Lines in 3D}
\begin{frame}
    \frametitle{3D coordinates}
    \begin{figure}
        \includegraphics[scale=0.1]{co_ordinate_plane.png}
        \caption{3D Coordinate Plane}
    \end{figure}
\end{frame}

\begin{frame}
    \frametitle{3D Coordinate System}
    \begin{itemize}
        \item The 3D coordinate system is defined by three mutually perpendicular axes: the x-axis, y-axis, and z-axis.
        \item A point in 3D space is represented by its coordinates \(P(x, y, z)\).
        \item The position vector of a point \(P\) in 3D space is given by:
        \[
        \vec{OP} = x\hat{\imath} + y\hat{\jmath} + z\hat{k}
        \]
    \end{itemize}       
\end{frame}

\begin{frame}
    \frametitle{Shifting the Origin}
    \begin{block}{Translation of Axes}
        Shifting the origin  to another point without changing the direction of axes is called \textbf{translation}.
    \end{block}
   Let the origin \(O\) be shifted to a new point \(O'\) with coordinates \(O'(x_0, y_0, z_0)\). The position vector of a point \(P(x,y,z)\) with respect to the new origin \(O'\) given by :
    \[
        \vec{O'P} = (x - x_0)\hat{\imath} + (y - y_0)\hat{\jmath} + (z - z_0)\hat{k}
    \]
\end{frame}

\begin{frame}
\frametitle{Distance Formula in 3D}
\begin{figure}
    \includegraphics[scale=0.1]{distance.png}
    \caption{Distance Formula in 3D}
\end{figure}
Let \(OP = x \hat{\imath} + y \hat{\jmath} + z \hat{k}\)  and \(OQ = x' \hat{\imath} + y' \hat{\jmath} + z' \hat{k}\) be two points in 3D space. The distance \(d\) between points \(P\) and \(Q\) is given by the formula:
\begin{align*}
    \vec{OQ} &= \vec{OP} + \vec{PQ} \\
    \vec{PQ} &= \vec{OQ} - \vec{OP} \\
    d &= |\vec{PQ}| = \sqrt{(x' - x)^2 + (y' - y)^2 + (z' - z)^2}
\end{align*}
\end{frame}

\begin{frame}
    \frametitle{Section Formula}
    \begin{figure}
        \includegraphics[scale=0.3]{section.png}
        \caption{Section Formula in 2D}
    \end{figure} 
\end{frame}


\begin{frame}
    \begin{block}{Section Formula in 3D}
    Let \(P(x_1, y_1, z_1)\) and \(Q(x_2, y_2, z_2)\) be two points in 3D space with position vectors \( \vec{\mathbf{r}_1} \), \( \vec{\mathbf{r}_2} \) respectively. If a point \(R\) divides the line segment \(PQ\) in the ratio \(m:n\), then the position vector of point \(R\) is given by:
    \[
    \vec{\mathbf{r}} = \frac{m\vec{\mathbf{r}_2} + n\vec{\mathbf{r}_1}}{m+n}
    \]
    \end{block}
\end{frame}

\begin{frame}
    \frametitle{Proof of Section Formula}
    \begin{block}{Given}
    Point \(R\) divides line segment \(PQ\) internally in the ratio \(m:n\), where \(P\) and \(Q\) have position vectors \(\vec{\mathbf{a}}\) and \(\vec{\mathbf{b}}\) respectively.
    \end{block}
    
    \begin{block}{To Prove}
    The position vector of \(R\) is \(\vec{\mathbf{r}} = \frac{m\vec{\mathbf{b}} + n\vec{\mathbf{a}}}{m+n}\)
    \end{block}
    
    \begin{block}{Proof}
    Since \(R\) divides \(PQ\) in ratio \(m:n\):
    \begin{align}
        \frac{|\vec{PR}|}{|\vec{RQ}|} &= \frac{m}{n} \\
        \text{Since } \vec{PR} \text{ and } \vec{RQ} \text{ are collinear:} \quad \vec{PR} &= \frac{m}{n}\vec{RQ}
    \end{align}
    \end{block}
\end{frame}

\begin{frame}
    \frametitle{Proof of Section Formula (continued)}
    
    From vector addition: \(\vec{PR} + \vec{RQ} = \vec{PQ}\)
    
    Substituting \(\vec{PR} = \frac{m}{n}\vec{RQ}\):
    \begin{align}
        \frac{m}{n}\vec{RQ} + \vec{RQ} &= \vec{PQ} \\
        \vec{RQ}\left(\frac{m}{n} + 1\right) &= \vec{PQ} \\
        \vec{RQ} &= \frac{n}{m+n}\vec{PQ}
    \end{align}

\end{frame}


\begin{frame}
 
    Since \(\vec{PQ} = \vec{\mathbf{b}} - \vec{\mathbf{a}}\) and \(\vec{RQ} = \vec{\mathbf{b}} - \vec{\mathbf{r}}\):
    \begin{align}
        \vec{\mathbf{b}} - \vec{\mathbf{r}} &= \frac{n}{m+n}(\vec{\mathbf{b}} - \vec{\mathbf{a}}) \\
        \vec{\mathbf{r}} &= \vec{\mathbf{b}} - \frac{n}{m+n}(\vec{\mathbf{b}} - \vec{\mathbf{a}}) \\
        \vec{\mathbf{r}} &= \frac{m\vec{\mathbf{b}} + n\vec{\mathbf{a}}}{m+n}
    \end{align}
  
\end{frame}

\begin{frame}
\frametitle{Direction Ratios}
    The direction ratios of a line are any three numbers that are proportional to the direction cosines of the line. If a line has direction cosines \(l, m, n\), then its direction ratios can be represented as \(a, b, c\) such that:
    \begin{align*}
        l &= \frac{a}{\sqrt{a^2 + b^2 + c^2}} =  k a \\
        m &= \frac{b}{\sqrt{a^2 + b^2 + c^2}}  = k b \\
        n &= \frac{c}{\sqrt{a^2 + b^2 + c^2}} = k c
    \end{align*}
\begin{itemize}
    \item It is evident that the direction ratios are not unique, as they can be scaled by any non-zero constant \(k\).
    \item However, the ratios \(a:b:c\) remain constant regardless of the scaling
\end{itemize}
\end{frame}

\begin{frame}
    \frametitle{Direction Cosines vs. Direction Ratios: The Intuition}
    \begin{itemize}
        \item \textbf{Direction Cosines (DCs):} A unique, standardized recipe. They tell you how much to travel along each axis to move exactly one unit along the line. The sum of their squares is always 1: \(l^2 + m^2 + n^2 = 1\).
        \item \textbf{Direction Ratios (DRs):} A flexible, proportional recipe. They tell you the ratio of movement along the axes. For every 'a' units in x, you move 'b' units in y and 'c' units in z. They are not unique; any scalar multiple represents the same direction.
    \end{itemize}
\end{frame}

\begin{frame}
    \frametitle{Direction Cosines (DCs): Mathematical Definition}
    Let a vector \(\vec{v}\) make angles \(\alpha, \beta, \gamma\) with the positive X, Y, and Z axes.
    \begin{itemize}
        \item \(l = \cos(\alpha)\)
        \item \(m = \cos(\beta)\)
        \item \(n = \cos(\gamma)\)
    \end{itemize}

    For a vector \(\vec{v} = (x, y, z)\) with magnitude \(|\vec{v}| = \sqrt{x^2 + y^2 + z^2}\):
    \[ l = \frac{x}{|\vec{v}|}, \quad m = \frac{y}{|\vec{v}|}, \quad n = \frac{z}{|\vec{v}|} \]
    A key property is that \(l^2 + m^2 + n^2 = 1\).
\end{frame}

\begin{frame}
    \frametitle{Converting Direction Ratios to Direction Cosines}
    If you have direction ratios \((a, b, c)\), you can find the direction cosines \((l, m, n)\) by normalizing them:
    \[ l = \frac{a}{\sqrt{a^2 + b^2 + c^2}} \]
    \[ m = \frac{b}{\sqrt{a^2 + b^2 + c^2}} \]
    \[ n = \frac{c}{\sqrt{a^2 + b^2 + c^2}} \]
    This process creates a unit vector, which is what the direction cosines represent.
\end{frame}


\begin{frame}
    \frametitle{Equation of a Line passing through a given point and parallel to a given vector}
    \begin{figure}
        \includegraphics[scale=0.4]{line_parallel_vector.png}
        \caption{Line in 3D space}
    \end{figure}
\end{frame}


\begin{frame}
\frametitle{Equation of a Line passing through a given point and parallel to a given vector}
    Let \(P_{0}(x_0, y_0, z_0)\) be a point on the line and \(P_{1}(x_1, y_1, z_1)\) be another point on the line. The direction vector of the line can be defined as \(\vec{P_{0}P_{1}} = \vec{r}_1 - \vec{r}_0\), where \(\vec{r}_0\) and \(\vec{r}_1\) are the position vectors of \(P_0\) and \(P_1\) respectively. The vector equation of the line can be expressed as:
    \[
    \vec{r} - \vec{r_0} = t\vec{v}
    \]
    where \(t\) is a scalar parameter. 
\end{frame}
\begin{frame}
    \frametitle{Equation of a Line: Two-Point Form}
    Given two points on the line, \(A(x_1, y_1, z_1)\) and \(B(x_2, y_2, z_2)\).
    \begin{block}{Vector Form}
        Let the position vectors of A and B be \(\vec{a}\) and \(\vec{b}\). The direction vector is \(\vec{d} = \vec{b} - \vec{a}\). The equation of the line is:
        \[ \vec{r} = \vec{a} + \lambda(\vec{b} - \vec{a}) \]
    \end{block}
    \begin{block}{Cartesian Form}
        The direction ratios are \((x_2 - x_1, y_2 - y_1, z_2 - z_1)\). The equation is:
        \[ \frac{x - x_1}{x_2 - x_1} = \frac{y - y_1}{y_2 - y_1} = \frac{z - z_1}{z_2 - z_1} \]
    \end{block}
\end{frame}

\begin{frame}
    \frametitle{Equation of a Line: Symmetric Form}
    If the direction ratios of the line are \(a, b, c\) and it passes through point \(P_0(x_0, y_0, z_0)\), the symmetric form of the line's equation is:
    \[
    \frac{x - x_0}{a} = \frac{y - y_0}{b} = \frac{z - z_0}{c}
    \]
    This form is particularly useful when the direction ratios are known.
\end{frame}

\begin{frame}
    \frametitle{Angle between Two Lines}
    To find the angle \(\theta\) between two lines with direction ratios \((a_1, b_1, c_1)\) and \((a_2, b_2, c_2)\), we use the formula:
    \[
    \cos(\theta) = \frac{a_1 a_2 + b_1 b_2 + c_1 c_2}{\sqrt{a_1^2 + b_1^2 + c_1^2} \sqrt{a_2^2 + b_2^2 + c_2^2}}
    \]
    This formula is derived from the dot product of the direction vectors.
\end{frame}

\begin{frame}
    \frametitle{Angle between Two Lines}
    Let \(\vec{r} = \vec{a+ \lambda b}\) and \(\vec{r'} = vec{a'} + \alpha b'\) be two straight line in 3D space. The angle \(\theta\) between the two lines is given by:
    \[
    \cos(\theta) = \frac{\vec{b} \cdot \vec{b'}}{|\vec{b}| |\vec{b'}|}
    \]
    where \(\vec{b}\) and \(\vec{b'}\) are the direction vectors of the lines.

    This is because the angle between the two lines is equal to the angle between their direction vectors.
\end{frame}

\begin{frame}
    \frametitle{Perpendicular Distance from a Point to a Line in 3D}
    To find the perpendicular distance \(d\) from a point \(P_0(x_0, y_0, z_0)\) to a line defined by a point \(P_1(x_1, y_1, z_1)\) and direction vector \(\vec{v} = (a, b, c)\), we use the formula:
    \[
    d = \frac{|\vec{P_0P_1} \times \vec{v}|}{|\vec{v}|}
    \]
    where \(\vec{P_0P_1} = (x_1 - x_0, y_1 - y_0, z_1 - z_0)\) is the vector from the point to the line.        
\end{frame}


\begin{frame}
\begin{figure}
    \includegraphics[scale=0.9]{perpendicular_line.png}
    \caption{Perpendicular Distance from a Point to a Line in 3D}
\end{figure}
\end{frame}
\begin{frame}
    \frametitle{Perpendicular Distance from a Point to a Line in 3D}
    Let the line AB is defined by 
    \[\frac{x-x_{1}}{a} = \frac{y-y_{1}}{b} = \frac{z-z_{1}}{c}\]
    and \(P(\alpha, \beta, \gamma)\) be the point. The perpendicular distance \(d\) from point \(P\) . The co-ordinates of the point \(L\) can be identified as:
    \[L(x_{1}+at, y_{1}+bt, z_{1}+ct)\]  
    where \(t\) is a scalar.. The vector \(\vec(AL)\) is given by 
    \[\vec(AL) = (x_{1}+at - x_{1}, y_{1}+bt - y_{1}, z_{1}+ct - z_{1}) = (at, bt, ct)\] 
\end{frame} 

\begin{frame}
    
    vector \(\vec{PL} \) is given by 
    \[\vec{PL} = (x_{1}+at - \alpha, y_{1}+bt - \beta, z_{1}+ct - \gamma)\]  
    
    Since \(AL \perp PL\), we can equate following

    \[(x_{1}+at-\alpha)(a) + (y_{1}+bt-\beta)(b) + (z_{1}+ct-\gamma)(c) = 0\]

    This can be solved for \(t\) to find the point \(L\) on the line.

\end{frame}

\begin{frame}
    \frametitle{Shortest Distance between two skew straight lines in 3D}

    Let the two lines are given by :
    \begin{align*}
    l_{1} = a_{1} + \lambda b_{1} \\
    l_{2} = a_{2} + \mu b_{2}   
    \end{align*}
    where \(a_{1}\) and \(a_{2}\) are position vectors of points on the lines, and \(b_{1}\) and \(b_{2}\) are direction vectors of the lines. Let \(\vec{AB}\) be the line through points \(a_{1}\) and \(a_{2}\). The perpendicular to both the lines can be found by taking the cross product of their direction vectors, i.e., \(\vec{b_{1}} \times \vec{b_{2}}\). The shortest distance \(d\) between the two skew lines is given by the projection of the line segment \(\vec{AB}\) onto the direction of the unit perpendicular, which can be calculated using the formula:

    \begin{align*}
    d = \frac{|\vec{AB} \cdot (\vec{b_{1}} \times \vec{b_{2}})|}{|\vec{b_{1}} \times \vec{b_{2}}|}
    \end{align*}

\end{frame}



\subsection{Planes in 3D}

\begin{frame}
    \frametitle{What is a Plane?}
    A plane is a surface such that if any two points on it are joined by a straight line, then every point on that line lies on the surface.
    \begin{block}{General Form}
        The general equation of a plane in 3D space is given by:
        \[
        Ax + By + Cz + D = 0
        \]
        where \(A\), \(B\), \(C\), and \(D\) are constants, and \(A\), \(B\), and \(C\) are not all zero.
    \end{block}
\end{frame}

\begin{frame}
    \frametitle{Proof of the Plane Equation}
    Let \(P(x_{1},x_{2},x_{3}) \) and \(Q(x_{2},y_{2},z_{2})\) be two points on the plane. Then we can write 
    \[
    ax_{1} + by_{1} + cz_{1} + d = 0
    \]
    and
    \[
    ax_{2} + by_{2} + cz_{2} + d = 0
    \]

    Let \(R\) be any point on the line joining \(P\) and \(Q\) with a ratio \(\lambda : 1\). Then the coordinates of \(R\) can be expressed as: 

    \[R\left(\frac{x_{1} + \lambda x_{2}}{1+\lambda}, \frac{y_{1} + \lambda y_{2}}{1+\lambda}, \frac{z_{1} + \lambda z_{2}}{1+\lambda}\right)\]
\end{frame} 

\begin{frame}
   Since \(R\) lies on the plane, it must satisfy the plane equation:

   \[
   a\left(\frac{x_{1} + \lambda x_{2}}{1+\lambda}\right) + b\left(\frac{y_{1} + \lambda y_{2}}{1+\lambda}\right) + c\left(\frac{z_{1} + \lambda z_{2}}{1+\lambda}\right) + d = 0
   \]

   Multiplying through by \(1+\lambda\) gives:

   \[
   a(x_{1} + \lambda x_{2}) + b(y_{1} + \lambda y_{2}) + c(z_{1} + \lambda z_{2}) + d(1+\lambda) = 0
   \]
\end{frame} 

\begin{frame}
   Rearranging terms leads to:

   \[
   (ax_{1} + by_{1} + cz_{1} + d) + \lambda (ax_{2} + by_{2} + cz_{2} + d) = 0
   \]

   Since both terms must equal zero, we have:

   \[
   ax_{1} + by_{1} + cz_{1} + d = 0
   \]
   and
   \[
   ax_{2} + by_{2} + cz_{2} + d = 0
   \]

   Thus, the line joining \(P\) and \(Q\) lies entirely within the plane.

\end{frame}

\begin{frame}
    \frametitle{Equation of a Plane in Normal Form}
    \begin{figure}
        \includegraphics[scale=0.4]{normal_form.png}
        \caption{Plane in 3D space}
    \end{figure}
\end{frame}

\begin{frame}
\frametitle{Equation of Plane in Normal Form}
    Let \(O\) be the origin and \(ON\) be the perpendicular from the origin \(O\) to the given plane \(\pi\) such that \(ON = d \hat{n}\), where \(\hat{n}\) is the unit normal vector to the plane where \( d\) is the perpendicular distance from the origin to the plane. 
    \begin{align*} 
        OP &= \vec{r} \\ 
        NP &\perp ON \\ 
        NP \cdot ON &= 0 \\ 
        (OP - ON) \cdot ON &= 0 \\
        ( \vec{r} - d \hat{n}) \cdot d\hat{n} &= 0 \\
        \vec{r} \cdot d \hat{n} - d^2 \hat{n} \cdot \hat{n} &= 0 \\
        \vec{r} \cdot \hat{n} &= d
    \end{align*}
\end{frame}

\begin{frame}
\frametitle{Normal form Cartesian Form}
 Let \(P(x,y,z)\) be a point on the plane and \( l,m,n\) be the direction cosines of the normal to the plane. The equation of the plane in Cartesian form is given by: 
 \begin{align*}
 r \cdot \hat{n} &= d \\
 x \hat{\imath} + y \hat{\jmath} + z \hat{k} \cdot l \hat{\imath} + m \hat{\jmath} + n \hat{k} &= d \\
 lx + my + nz &= d
 \end{align*}
where \(d\) is the perpendicular distance from the origin to the plane. 
\end{frame} 

\begin{frame}
\frametitle{Normal form vs General form of Plane Equation}
\begin{itemize}
\item The normal form  of the plane given by \( \vec{r} \cdot \hat{n} = d\) . In the cartesian form  
\[ lx + my + nz = d \] 
where \(l,m,n\) are the direction cosines of the normal to the plane as the normal is a unit vector.

\item The general form of the plane is given by \(Ax + By + Cz + D = 0\) where \(A,B,C\) are the direction ratios of the normal to the plane. The coefficents of \(A,B,C\) in the general form are the direction ratios of the normal to the plane.
\end{itemize}
\end{frame}

\begin{frame}
    \frametitle{Vector Equation of a Plane Passing Through
a Given Point and Normal to a Given Vector}

Let \( \vec{r} \) be the position vector of a point \( P(x,y,z) \) on the plane, and let \( \vec{a} \) be the position vector of a given point \( A(x_0,y_0,z_0) \) on the plane. The vector equation of the plane can be expressed as:

\[
(\vec{r} - \vec{a}) \cdot \hat{n} = 0
\]

where \( \hat{n} \) is the unit normal vector to the plane.

\end{frame} 

\begin{frame}
\frametitle{Equation of a Plane Passing Through
a Given Point and Parallel to Two
Given Vectors}
We can infer following observations :
\begin{enumerate}
    \item A vector on the plane is given by \( \vec{r} - \vec{a} \) where \(\vec{r},\vec{a}\) are the position vectors of points \(P\) and \(A\) respectively.
    \item As two vectors \( \vec{b} \) and \( \vec{c} \) are parallel to the plane, the vector \( \vec{r} - \vec{a} \) is coplanar with vectors \( \vec{b} \) and \( \vec{c} \). 
    \item This means \((\vec{r} - \vec{a}) = \lambda_1 \vec{b} + \lambda_2 \vec{c}\) for some scalars \(\lambda_1\) and \(\lambda_2\).
    \item That is \(\vec{r} = \vec{a} + \lambda_1 \vec{b} + \lambda_2 \vec{c}\)
\end{enumerate}     
\end{frame}

\begin{frame}
    \frametitle{Angle between Two Planes}
    The angle bwtween two planes is defined as the angle between their normal vectors. 
    \begin{align*}
        \vec{r} \cdot \hat{n_1} &= d_1 \\
        \vec{r} \cdot \hat{n_2} &= d_2 \\
        \cos(\theta) &= \frac{\hat{n_1} \cdot \hat{n_2}}{|\hat{n_1}| |\hat{n_2}|} \\
    \end{align*}
\end{frame}

\begin{frame}
    \frametitle{Plane parallel to a given Plane}
    Let the equation of the given plane be \( \vec{r} \cdot \hat{n} = d_{1}\). 
    A plane parallel to this plane will have the same normal vector, so its equation can be written as:
    \begin{align*}
    \vec{r} \cdot \hat{n} &= d_{2} \\
    \end{align*}
    
\end{frame}

\begin{frame}
    \frametitle{Equation of a Plane Passing Through the Intersection of Two Given Planes}
    Let the equations of the two given planes be:
    \begin{align*}
    \vec{r} \cdot \hat{n_1} &= d_{1} \\
    \vec{r} \cdot \hat{n_2} &= d_{2} \\
    \end{align*}
    where \(\vec{r}\) is the position vector that lies on the line
    we can take the sum of the equations of the two planes and introduce a parameter (\(\lambda\)) to account for all possible points of Intersection
    \[ \vec{r} \cdot \hat{n_1} + \lambda (\vec{r} \cdot \hat{n_2}) = d_{1} + \lambda d_{2} \]
    represents a family of planes passing through the intersection line of the two given planes, where \(\lambda\) is a scalar parameter.
\end{frame}


\begin{frame}
    \frametitle{The Distance of a point from a Plane}
    The distance $d$ of a point $P_1(x_1, y_1, z_1)$ from a plane with the equation $Ax + By + Cz + D = 0$ is given by the formula:
    
    \[d = \frac{|Ax_1 + By_1 + Cz_1 + D|}{\sqrt{A^2 + B^2 + C^2}}\]
\end{frame}

\begin{frame}
    \frametitle{Key Concepts}
    \begin{itemize}
        \item To find the shortest distance from a point to a plane, we use \textbf{vector projection}.
        \item The shortest distance is always along the line segment that is \textbf{perpendicular} to the plane.
        \item The direction of this line segment is parallel to the plane's \textbf{normal vector}, $\vec{n} = \langle A, B, C \rangle$.
        \item We can find the distance by projecting a vector connecting the point and the plane onto the normal vector.
    \end{itemize}
\end{frame}

\begin{frame}
    \frametitle{Defining the Vectors}
    \begin{itemize}
        \item \textbf{The Normal Vector}: The vector perpendicular to the plane is $\vec{n} = \langle A, B, C \rangle$. The corresponding unit normal vector is $\hat{n} = \frac{\vec{n}}{|\vec{n}|}$.
        \item \textbf{The Point and the Plane}:
        \begin{itemize}
            \item Let $P_1(x_1, y_1, z_1)$ be the given point.
            \item Let $P_0(x_0, y_0, z_0)$ be any arbitrary point on the plane.
        \end{itemize}
        \item \textbf{The Connecting Vector}: The vector from the point on the plane to the given point is $\vec{P_0P_1} = \langle x_1 - x_0, y_1 - y_0, z_1 - z_0 \rangle$.
    \end{itemize}
\end{frame}

\begin{frame}
    \frametitle{The Proof (Part 1)}
    The shortest distance $d$ is the magnitude of the scalar projection of the connecting vector $\vec{P_0P_1}$ onto the normal vector $\vec{n}$.
    
    \[d = |\text{proj}_{\vec{n}} \vec{P_0P_1}| = \left|\vec{P_0P_1} \cdot \frac{\vec{n}}{|\vec{n}|}\right|\]
    
    Substitute the components of the vectors:
    
    \[d = \frac{|\langle x_1 - x_0, y_1 - y_0, z_1 - z_0 \rangle \cdot \langle A, B, C \rangle|}{\sqrt{A^2 + B^2 + C^2}}\]
    
    \[d = \frac{|A(x_1 - x_0) + B(y_1 - y_0) + C(z_1 - z_0)|}{\sqrt{A^2 + B^2 + C^2}}\]
\end{frame}

\begin{frame}
    \frametitle{Distace between two parallel planes}
    Let the equations of the two parallel planes be:
    \begin{align*}
    \vec{r} \cdot \hat{n} &= d_{1} \\
    \vec{r} \cdot \hat{n} &= d_{2} \\
    \end{align*}
    where \(\vec{r}\) is the position vector that lies on the line.
    for normal unit vector \(\hat{n}\) the distance is 
    \[d = |d_{1} - d_{2}|\]

    The distance \(D\) between the two parallel planes is given by:
    \[D = \frac{|d_{1} - d_{2}|}{|{n}|}\]
    where \(|{n}|\) is the magnitude of the normal vector.

\end{frame}

\begin{frame}
    \frametitle{Line of Intersection of Two Planes}
    The line of intersection is formed by the set of points that satisfy the equations of both planes simultaneously.
    
    Let two planes be given by:
    \begin{itemize}
        \item Plane 1: $A_1x + B_1y + C_1z + D_1 = 0$
        \item Plane 2: $A_2x + B_2y + C_2z + D_2 = 0$
    \end{itemize}
    The line of intersection is defined by this system of two equations.
\end{frame}

\begin{frame}
    \frametitle{Proof: Finding the Direction Vector}
    \begin{itemize}
        \item The line of intersection is perpendicular to the normal vector of each plane.
        \item The normal vector of Plane 1 is $\vec{n_1} = \langle A_1, B_1, C_1 \rangle$.
        \item The normal vector of Plane 2 is $\vec{n_2} = \langle A_2, B_2, C_2 \rangle$.
        \item Therefore, the direction vector of the line of intersection, $\vec{v}$, is given by the cross product of the two normal vectors.
    \end{itemize}
    
    $$\vec{v} = \vec{n_1} \times \vec{n_2} = \begin{vmatrix} \hat{i} & \hat{j} & \hat{k} \\ A_1 & B_1 & C_1 \\ A_2 & B_2 & C_2 \end{vmatrix}$$
\end{frame}

\begin{frame}
    \frametitle{Proof: Finding a Point on the Line}
    \begin{itemize}
        \item To define a line, we also need a specific point that lies on it.
        \item We can find a point by setting one of the variables ($x, y,$ or $z$) to a constant value, such as $z = 0$.
        \item This creates a system of two linear equations with two unknowns, which can be solved easily.
        \item For example, setting $z=0$ gives:
        \begin{itemize}
            \item $A_1x + B_1y + D_1 = 0$
            \item $A_2x + B_2y + D_2 = 0$
        \end{itemize}
        \item Solving this system for $x$ and $y$ gives us a point $(x, y, 0)$ on the line of intersection.
    \end{itemize}
\end{frame}

\begin{frame}
    \frametitle{Final Equation of the Line}
    Once we have the direction vector $\vec{v}$ and a point $P_0(x_0, y_0, z_0)$, we can write the equation of the line of intersection in a parametric form.
    
    $$\vec{r} = \vec{r_0} + t\vec{v}$$
    
    Where $\vec{r_0} = \langle x_0, y_0, z_0 \rangle$ and $t$ is a scalar parameter.
\end{frame}

\begin{frame}
    \frametitle{Equation of a Plane Through the Intersection of Two Planes}
    \textbf{The Concept:} Any plane that passes through the line of intersection of two given planes can be represented by a linear combination of their equations.
    
    Let the two given planes be:
    \begin{itemize}
        \item Plane 1: $P_1 \equiv A_1x + B_1y + C_1z + D_1 = 0$
        \item Plane 2: $P_2 \equiv A_2x + B_2y + C_2z + D_2 = 0$
    \end{itemize}
    The equation of a plane passing through their intersection is given by:
    $$P_1 + \lambda P_2 = 0$$
    $$(A_1x + B_1y + C_1z + D_1) + \lambda(A_2x + B_2y + C_2z + D_2) = 0$$
    Where $\lambda$ is a scalar constant.
\end{frame}

\begin{frame}
    \frametitle{Proof}
    \textbf{Step 1: The Locus of Points}
    \begin{itemize}
        \item The equation $P_1 + \lambda P_2 = 0$ is a linear equation in $x, y, z$.
        \item A linear equation in three variables always represents a plane.
        \item Therefore, $P_1 + \lambda P_2 = 0$ represents a family of planes for different values of $\lambda$.
    \end{itemize}

    \textbf{Step 2: Proving the Intersection Property}
    \begin{itemize}
        \item We need to prove that any point on the line of intersection of $P_1$ and $P_2$ will also lie on the plane $P_1 + \lambda P_2 = 0$.
        \item Let $P(x_0, y_0, z_0)$ be any point on the line of intersection of Plane 1 and Plane 2.
        \item Since $P$ lies on Plane 1, it satisfies the equation $P_1 = 0$. So, $A_1x_0 + B_1y_0 + C_1z_0 + D_1 = 0$.
        \item Since $P$ lies on Plane 2, it satisfies the equation $P_2 = 0$. So, $A_2x_0 + B_2y_0 + C_2z_0 + D_2 = 0$.
    \end{itemize}
\end{frame}

\begin{frame}
    \frametitle{Proof (Continued)}
    \textbf{Step 3: Verifying the Equation}
    \begin{itemize}
        \item Now, let's substitute the coordinates of point $P$ into the equation of the new plane, $P_1 + \lambda P_2 = 0$:
        \begin{align*} (A_1x_0 + B_1y_0 + C_1z_0 + D_1) &+ \lambda(A_2x_0 + B_2y_0 + C_2z_0 + D_2) = 0 \\ (0) &+ \lambda(0) = 0 \\ 0 &= 0 \end{align*}
        \item The equation holds true for any point on the line of intersection and for any value of $\lambda$.
        \item This proves that every plane represented by the equation $P_1 + \lambda P_2 = 0$ contains the line of intersection of the planes $P_1$ and $P_2$.
    \end{itemize}
\end{frame}
\begin{frame}
    \frametitle{Proof (Vector Form)}
    \textbf{The Concept:}
    \begin{itemize}
        \item Let the two given planes be in vector form:
        \begin{itemize}
            \item Plane 1: $\vec{r} \cdot \vec{n_1} = d_1$
            \item Plane 2: $\vec{r} \cdot \vec{n_2} = d_2$
        \end{itemize}
        \item The equation of a plane passing through their intersection is given by the linear combination:
        \[(\vec{r} \cdot \vec{n_1} - d_1) + \lambda(\vec{r} \cdot \vec{n_2} - d_2) = 0\]
    \end{itemize}

    \textbf{Step 1: The Locus of Points}
    \begin{itemize}
        \item Rearranging the equation:
        \[\vec{r} \cdot (\vec{n_1} + \lambda\vec{n_2}) = d_1 + \lambda d_2\]
        \item This is of the form $\vec{r} \cdot \vec{N} = D$, which is the equation of a plane, with $\vec{N} = \vec{n_1} + \lambda\vec{n_2}$ and $D = d_1 + \lambda d_2$.
    \end{itemize}
\end{frame}

\begin{frame}
    \frametitle{Proof (Vector Form - Continued)}
    \textbf{Step 2: Proving the Intersection Property}
    \begin{itemize}
        \item Let $\vec{r_0}$ be the position vector of any point on the line of intersection.
        \item By definition, $\vec{r_0}$ satisfies both original plane equations.
        \begin{itemize}
            \item $\vec{r_0} \cdot \vec{n_1} - d_1 = 0$
            \item $\vec{r_0} \cdot \vec{n_2} - d_2 = 0$
        \end{itemize}
        \item Substitute $\vec{r_0}$ into the equation of the new plane:
        \[(\vec{r_0} \cdot \vec{n_1} - d_1) + \lambda(\vec{r_0} \cdot \vec{n_2} - d_2) = 0\]
        \[(0) + \lambda(0) = 0\]
        \[0 = 0\]
        \item Since the equation is satisfied for any point on the line of intersection, the new plane contains this line. This completes the proof in vector form.
    \end{itemize}
\end{frame}

\begin{frame}
    \frametitle{Angle between a Line and a Plane}
    \begin{figure}
        \includegraphics[scale=0.7]{angle_line_plane.png}
        \caption{Angle between a Line and a Plane}
    \end{figure}
\end{frame}

\begin{frame}
\frametitle{Angle between a Line and a Plane}
\begin{itemize}
    \item Let the line be represented in vector form as:
    \[\vec{r} = \vec{a} + \mu \vec{b}\]
    where $\vec{a}$ is a point on the line, $\vec{b}$ is the direction vector of the line, and $\mu$ is a scalar parameter.
    \item Let the plane be represented as:
    \[\vec{r} \cdot \vec{n} = d\]
    where $\vec{n}$ is the normal vector of the plane and $d$ is the distance from the origin to the plane.
    \item To find the angle $\theta$ between the line and the plane, we can use the fact that the angle between the line and the normal to the plane is complementary to the angle between the line and the plane itself.
\end{itemize}                       
\end{frame}


% Part III: Calculus
\part{Calculus}

\chapter{Introduction to Calculus}

\section{Chapter Overview}
\begin{itemize}
    \item \textbf{Historical Context:} Mathematicians of the 17th century were keenly interested in studying motion
    \begin{itemize}
        \item Objects on or near Earth
        \item Motion of planets and stars
    \end{itemize}
    
    \item \textbf{Key Insights:} Motion study involved both speed and direction
    \begin{itemize}
        \item Direction at any instant: along the tangent line to the path
        \item Need for precise mathematical tools
    \end{itemize}
    
    \item \textbf{The Limit Concept:} Fundamental to finding
    \begin{itemize}
        \item Velocity of moving objects
        \item Tangent lines to curves
    \end{itemize}
    
    \item \textbf{Function Behavior:} Limits help us distinguish between
    \begin{itemize}
        \item Continuous variation (small changes in $x$ → small changes in $f(x)$)
        \item Discontinuous behavior (jumps, erratic variation)
        \item Unbounded growth or decay
    \end{itemize}
    
    \item \textbf{Our Approach:} Develop limits intuitively, then formally
\end{itemize}
\section{Slopes and Average Rate of Change}

% Slide: Motivation and Overview
\section{Motivation: Why Study Change?}
  \begin{itemize}
    \item Calculus explores how quantities change : how change in one quantity is related to a change in anotherand provides tools for modeling these changes.
    \item Functions link inputs ($x$) to outputs ($y=f(x)$); we investigate how $y$ varies as $x$ moves over an interval.
    \item Real-world example: Predicting economic indicators, modeling speeds, and more.
  \end{itemize}

% Slide: Defining Average Rate of Change
\section{Average Rate of Change and Secant Lines}
  \begin{block}{Definition}
    For a function $f(x)$ on interval $[x_{1},x_{2}]$, the \emph{average rate of change} over that interval is
    \[ 
      = \frac{\Delta y}{\Delta x} = \frac{f(x_{2})-f(x_{1})}{x_{2}-x_{1}} = \frac{f(x_{1}+h) - f(x_{1})}{h}, h \neq 0
    \]
    which geometrically represents the slope of the secant line between $(x_{1},f(x_{1}))$ and $(x_{2},f(x_{2}))$.
  \end{block}
  \begin{itemize}
    \item Rise: $f(x_{2})-f(x_{1})$
    \item Run: $x_{2}-x_{1}$
    \item The secant line smooths out fluctuations; its slope reports the net change per unit input over the interval.
    \item Approximating the curve over the interval with a straight line. 
  \end{itemize}

% Slide: Graphical Illustration
\section{Graphical Illustration}
  \begin{columns}
    \column{0.6\textwidth}
      \includegraphics[width=\textwidth]{avg_rate_of_change.png}
    \column{0.4\textwidth}
      \begin{itemize}
        \item Focus on interval $[1,3]$ on the curve $y=f(x)$.
        \item Secant line (in orange) joins $(1,f(1))$ and $(3,f(3))$.
        \item Its slope measures the average change in $y$ per unit change in $x$.
      \end{itemize}
  \end{columns}

% Slide: Problem – Trivandrum to Chennai Average Speed
\section{Example: Trivandrum to Chennai Journey}
  \begin{columns}
    \column{0.5\textwidth}
      \includegraphics[width=\textwidth]{trivandrum_chennai_map.png}
    \column{0.5\textwidth}
      \includegraphics[width=\textwidth]{trivandrum_chennai_journey.png}
  \end{columns}
  \vspace{1ex}


  \begin{itemize}
    \item Total distance: 762 km
    \item Total time (including stops): 15 hours
    \item Average speed: $\dfrac{762}{15}\approx50.8$ km/h
  \end{itemize}
  \begin{solutionblock}
    Average speed $=\dfrac{h}$.  
  This illustrates how the secant slope over an interval gives overall performance even with rest periods.
  \end{solutionblock}

% Slide: Sensitivity to Rounding
\section{Rounding and Accuracy}
  \begin{columns}
    \column{0.58\textwidth}
      \includegraphics[width=\textwidth]{rounding_accuracy_demo.png}
    \column{0.42\textwidth}
      \begin{itemize}
        \item Distance measurements rounded to nearest kilometer introduce potential error
        \item For our journey: $\dfrac{762\text{ km}}{15\text{ h}} \approx 50.8$ km/h
        \item A mere 3 km error in measurement can push calculated speed above/below the 50 km/h limit
        \item Data precision is important.
      \end{itemize}
  \end{columns}

% Slide: Toward Instantaneous Rate of Change
\section{Instantaneous Rate of Change}
  \begin{itemize}
    \item Instantaneous speed corresponds to slope of tangent line at a point.
    \item As secant interval shrinks ($b\to a$), average rate approaches derivative $f'(x)$.
    \item Next: Formalize tangent lines and derivatives.
  \end{itemize}

\frametitle{Why Instantaneous Rate of Change?}
\begin{itemize}
  \item Average rate of change provides overall performance but lacks detail.
  \item Instantaneous rate of change captures behavior at a specific moment while average rate is over an interval
  \item Essential for understanding dynamic systems (e.g., velocity, acceleration).
  \item Instantaneous rate of change = calculate average rathe of change over smaller and smaller intervals.
\end{itemize}
\section{Secants to Tangents}
  \begin{block}{Problem}
    Find the slope of the tangent line to the parabola $y = x^{2}$ at $P(2,4)$ by looking at slopes of secant lines through $P$ and a nearby point $Q(2+h,(2+h)^{2})$ and letting $h \to 0$.
  \end{block}

  \begin{block}{Average rate of change (secant slope)}
    Between $x=2$ and $x=2+h$ (with $h \ne 0$):
    \[
      \text{slope}(h)=\frac{(2+h)^2 - 4}{h}
    \]
  \end{block}

  \begin{itemize}
    \item Expand: $(2+h)^2 = 4 + 4h + h^{2}$
      \[
        \text{slope}(h)=\frac{4+4h+h^{2}-4}{h}= \frac{4h+h^{2}}{h}
      \]
    \item Simplify (since $h \ne 0$): $\text{slope}(h)=4 + h$
    \item Sample values:
      \[
        h=1\!\to\!5,\; h=0.5\!\to\!4.5,\; h=0.1\!\to\!4.1,\; h=0.01\!\to\!4.01,\; h=-0.01\!\to\!3.99
      \]
    \item The secant slopes get closer to $4$ as $h$ gets closer to $0$.
    \item Limit viewpoint:
      \[
        \lim_{h\to 0} \frac{(2+h)^2 - 4}{h} = \lim_{h\to 0} (4 + h)=4
      \]
      So the instantaneous rate of change (slope of the tangent line) at $x=2$ is $4$.
  \end{itemize}



  \frametitle{The Mean Value Theorem}
  \begin{figure}
    \includegraphics[width=0.75\textwidth]{mean_value.png}
    \caption{Mean Value Theorem: There exists a point $c$ where the tangent equals the secant.}
  \end{figure}
% Slide: Mean Value Theorem (informal, no derivatives)
\section{The Mean Value Theorem (informal, no derivatives)}
  \begin{block}{Statement}
  For a smooth curve $y=f(x)$ on an interval $[a,b]$, the \emph{average rate of change}
  \[ \dfrac{f(b)-f(a)}{b-a} \]
  is realized as the slope of a \emph{tangent line} at some point $x=c$ with $a<c<b$.
  In other words, the secant line joining $(a,f(a))$ and $(b,f(b))$ has the same slope as
  the tangent line to the curve at some point inside the interval.
  \end{block}
  \vspace{0.5em}
\section{Displacement, Velocity, and Acceleration}
\section{Displacement}
  \begin{block}{Definition}
    The displacement $x(t)$ at time $t$ measures position on the real line relative to an origin $0$, so positive values indicate one direction and negative values the opposite.
  \end{block}
  \begin{itemize}
    \item Independent variable: time $t$ (typically seconds).
    \item Dependent variable: displacement $x$.
    \item Choice of origin and direction provides a signed measure of position.
  \end{itemize}

\section{Velocity}
  \begin{block}{Definition}
    The velocity $v(t)$ is the instantaneous rate of change of displacement $x(t)$.
  \end{block}
  \begin{itemize}
    \item $v(t)>0$ (motion in the positive direction), $v(t)<0$ (negative direction), or $v(t)=0$.
    \item Speed is the magnitude of velocity: $\text{speed}=\lvert v(t)\rvert$.
    \item A speedometer displays instantaneous speed.
  \end{itemize}

\section{Acceleration}
  \begin{block}{Definition}
    The acceleration $a(t)$ is the instantaneous rate of change of velocity $v(t)$.
  \end{block}
  \begin{itemize}
    \item $a(t)$ may be positive, negative, or zero.
    \item The term “deceleration” is often used when $a(t)$ is negative and quoted as a positive number by convention.
  \end{itemize}

\section{Example: Cannonball Projectile}
  \begin{figure}
    \centering
    \includegraphics[width=0.5\textwidth]{cannonball.png}
  \end{figure}






% Part IV: Probability and Statistics
\part{Probability and Statistics}

% \include{chapters/chapter_probability_statistics_introduction}
\chapter{Combinatorial Analysis}

\section{Why combinatorial analysis?}

A communication system consist of \(n\) seemingly identical antennas that are lined up in a linear ordered array. The resulting system will then be able to receive all incoming signals and will be called functional as long as no two consecutive antennas are defective. If \(m\) of the \(n\) antennas are defective, how many different states of the system are possible?

If \(n =4\) and \(m=2\) there are \(6\) states possible : 

\[ 
\begin{matrix}
    0 & 1 & 1 & 0 \\
    1 & 0 & 1 & 0 \\
    1 & 0 & 0 & 1 \\
    0 & 1 & 0 & 1 \\
    0 & 0 & 1 & 1 \\
    1 & 1 & 0 & 0 \\
\end{matrix}
\]

In \(6\) of these \(3\) of them are functional. 

\begin{definition}
    The mathematical theory of counting is known as \textbf{combinatorial analysis}  
\end{definition}

\section{The basic principle of counting}

\begin{definitionbox}[title=The Basic Principle of Counting]
    Suppose that two experiments are to be performed. The first experiment has \(m\) possible outcomes and for each of these outcomes the second experiment has \(n\) possible outcomes. Then the two experiments together have \(m \times n\) possible outcomes.
\end{definitionbox}
\begin{examplebox}[title=Example: Coin and Die]
    A coin is tossed and a die is rolled. How many possible outcomes are there?
    
    \textbf{Solution:} The coin has \(2\) possible outcomes and the die has \(6\) possible outcomes. Therefore the two experiments together have \(2 \times 6 = 12\) possible outcomes.
\end{examplebox}


\begin{definitionbox}[title=The Generalized Basic Principle of Counting]
    If \(r\) experiments are to be performed are to be such that the first one may result in any of \(n_1\) possible outcomes, and for each of these \(n_{1}\) possible outcomes there are \(n_2\) possible outcomes for the second experiment, and so on up to the \(r\)-th experiment which may result in any of \(n_r\) possible outcomes, then the \(r\) experiments together have \(n_1 \times n_2 \times \ldots \times n_r\) possible outcomes.
\end{definitionbox}


\begin{figure}[H]
    \centering
    \includegraphics[width=0.8\textwidth]{counting.png}
    \caption{Mermaid diagram illustrating the Generalized Basic Principle of Counting}
    \label{fig:counting}
\end{figure}



\begin{examplebox}[title=Example: License Plates with Repetition]
    How many different \(7\) place license plates are possible if the first two places are to be occupied by letters and the remaining five by digits?
    
    \textbf{Solution:} The first two places can be filled with any of the \(26\) letters of the alphabet. Assuming letter can be repeated, the first place has \(26\) options and the second place has \(26\) options. The remaining five places can be filled with any of the \(10\) digits (from \(0\) to \(9\)), repeatedly. Therefore, the total number of different license plates is:

    \[
    26 \times 26 \times 10^5
    \]
\end{examplebox}

\begin{examplebox}[title=Example: License Plates without Repetition]
    If the letters and numbers in the previous example cannot be repeated, how many different license plates are possible?
    
    \textbf{Solution:} The first place has \(26\) options and the second place has \(25\) options (since letters cannot be repeated). The remaining five places can be filled with any of the \(10\) digits (from \(0\) to \(9\)). Therefore, the total number of different license plates is:

    \[
    26 \times 25 \times 10 \times 9 \times 8 \times 7 \times 6
    \]
\end{examplebox}

\section{Permutations}

\textbf{How many different ordered arrangements of letters \(a,b,c\) are possible?}

By enumeration we can see that : 

\[
abc, acb, bac, bca, cab, cba
\]
Each arrangement is called a \textbf{permutation} of the letters \(a,b,c\). There are \(6\) possible permutations for \(3\) distinct letters or a set of \(3\) distinct objects.

\begin{definitionbox}
    A permutation of \(n\) distinct objects is an arrangement of the objects in a specific order. The number of permutations of \(n\) distinct objects is given by \(n!\) (n factorial), which is the product of all positive integers up to \(n\):
    
    \[
    n! = n \times (n-1) \times (n-2) \times \ldots \times 2 \times 1
    \]
\end{definitionbox} 


\begin{examplebox}
    \textbf{Example 3:} How many different letter arrangments can be made from the letters PEPPER?
        \textbf{solution:} We have \(6\) letters in total where \(3\) letters are \(P\), \(2\) letters are \(E\) and \(1\) letter is \(R\). Therefore the number of different letter arrangements is given by:
        \[ \frac{6!}{3! \times 2! \times 1!} = 60 \]
        The idea is illustrated in Figure~\ref{fig:permute_pepper}.
\end{examplebox}

\begin{figure}[H!]
    \centering
    \includegraphics[width=1.1\textwidth]{permute_pepper_hd.png}
    \caption{The permutations of the letters in the word "PEPPER" by taking one permutation "PEPPER" and showing how many arrangements of the letters lead to the same word. This will repeat for other permutations as well}
    \label{fig:permute_pepper}
\end{figure}


\begin{keyconceptbox}
Let us start with \(n\) different objects and \(k\) be a positive integer such that \(k \leq n\). We want to determine the number of ways in which we can pick \(k\) objects from the total of \(n\) objects and arrange them in a sequence. i.e. the number of distinct \(k\) object sequence. We can do it in the following way 

\begin{itemize}
    \item We can pick the first object in \(n\) ways
    \item We can pick the second object in \(n-1\) ways (since we cannot pick the first object again)
    \item We can pick the third object in \(n-2\) ways (since we cannot pick the first and second objects again)
    \item ...
    \item We can pick the \(k\)-th object in \(n-(k-1)\) ways (since we cannot pick the first \(k-1\) objects again)
\end{itemize}

Therefore, by the generalized basic principle of counting, the total number of ways in which we can pick \(k\) objects from the total of \(n\) objects and arrange them in a sequence is given by
\[ 
    n \times (n-1) \times (n-2) \times \ldots \times (n-(k-1)) 
\]

\[
    = \frac{n(n-1)\cdots(n-(k-1))\times (n-k) \times (n-(k+1)) \times \cdots \times 2 \times 1}{(n-k) \times (n-(k+1)) \times \cdots \times 2 \times 1} 
\]

\[
= \frac{n!}{(n-k)!}
\]

In a special case when \(k=n\), we have
\[ \frac{n!}{(n-n)!} = \frac{n!}{0!} = n! \]
which is the number of ways in which we can arrange \(n\) distinct objects in a sequence.
\end{keyconceptbox}
\begin{exercisebox}
    \textbf{Exercise:} A chess tournament is to be held among \(10\) players. \(4\) are Russian, \(3\) are American, \(2\) are French and \(1\) is a German. How many different ways can the players be ranked from first to tenth place if the tournament just list the nationality of the players?
\end{exercisebox}


\begin{solutionbox}
    \textbf{Solution:} We have \(10\) players in total where \(4\) players are Russian, \(3\) players are American, \(2\) players are French and \(1\) player is German. Therefore the number of different ways the players can be ranked from first to tenth place is given by:
    \[ \frac{10!}{4! \times 3! \times 2! \times 1!} = 12600 \]
\end{solutionbox}

\section{Combinations}
\subsection{Inspiration}

Let's consider a problem: How many different groups of 3 objects could be formed from a total of 5 distinct objects A, B, C, D, and E?

If we think about selecting these objects in order, we'd have:
\begin{itemize}
    \item 5 ways to select the first object
    \item 4 ways to select the second object
    \item 3 ways to select the third object
\end{itemize}

Using the generalized basic principle of counting, we get $5 \times 4 \times 3 = 60$ ways of making ordered selections.

However, if we're only interested in which objects are in the group, not their order, then each distinct group is counted multiple times. For instance, the group \{A, B, C\} appears as all possible permutations: ABC, ACB, BAC, BCA, CAB, and CBA.

Since each group of 3 objects can be arranged in $3! = 6$ ways, the actual number of different groups is:
\[ \frac{5 \times 4 \times 3}{3!} = \frac{60}{6} = 10 \]

This illustrates the fundamental relationship between permutations and combinations: when we don't care about order, we divide the number of permutations by the number of ways to arrange the selected items.
\subsection{Definition}
\begin{definitionbox}
    A combination of \(n\) distinct objects taken \(r\) at a time is a selection of \(r\) objects from the total of \(n\) objects without regard to the order of selection. The number of combinations of \(n\) distinct objects taken \(r\) at a time is denoted by \(C(n,r)\) or \(\binom{n}{r}\) and is given by:
    
    \[
    C(n,r) = \binom{n}{r} = \frac{n!}{r!(n-r)!}
    \]
\end{definitionbox}

\begin{examplebox}[title=Example: Antennas with No Consecutive Defectives]
    \textbf{Problem:} Consider a set of $n$ antennas of which $m$ are defective and $n-m$ are functional. Assume that all defective antennas are indistinguishable from each other, and all functional antennas are indistinguishable from each other. How many linear orderings are there in which no two defective antennas are consecutive?
    
    \textbf{Solution:} This is a problem about combinations rather than permutations since we consider all defective antennas as identical, and likewise for functional antennas.
    
    To ensure no two defective antennas are consecutive, we need to place the $m$ defective antennas such that they are always separated by at least one functional antenna.
    
    Let's approach this by first placing the $n-m$ functional antennas in a row, creating $n-m+1$ potential positions for defective antennas (including before the first functional antenna, between functional antennas, and after the last functional antenna):
    
    \[
    \_ \circ \_ \circ \_ \circ \_ \cdots \circ \_
    \]
    
    where $\circ$ represents a functional antenna and $\_$ represents a potential position for defective antennas.
    
    Now, we need to choose $m$ positions out of these $n-m+1$ potential positions to place our defective antennas. This is a combination problem:
    
    \[
    \binom{n-m+1}{m} = \frac{(n-m+1)!}{m!(n-m+1-m)!} = \frac{(n-m+1)!}{m!(n-2m+1)!}
    \]
    
    Therefore, the number of linear orderings with no consecutive defective antennas is $\binom{n-m+1}{m}$.
    
    For example, if $n=7$ and $m=3$, then we have $\binom{7-3+1}{3} = \binom{5}{3} = 10$ possible arrangements with no consecutive defective antennas.
\end{examplebox}

\subsection{Pascal's Identity}
\begin{theorembox}[title=Pascal's Identity]
    For any integers \(n \geq 1\) and \(1 \leq k \leq n\),
    \[
    \binom{n}{k} = \binom{n-1}{k-1} + \binom{n-1}{k}
    \]
\end{theorembox}
\paragraph{Proof:}
We can prove Pascal's Identity using a combinatorial argument. Consider a set of \(n\) elements. We want to choose a subset of \(k\) elements from this set. We can do this in two ways:

1. **Case 1:** The first element is included in the subset. In this case, we need to choose \(k-1\) elements from the remaining \(n-1\) elements. The number of ways to do this is \(\binom{n-1}{k-1}\).

2. **Case 2:** The first element is not included in the subset. In this case, we need to choose \(k\) elements from the remaining \(n-1\) elements. The number of ways to do this is \(\binom{n-1}{k}\).

By considering both cases, we can express the total number of ways to choose \(k\) elements from \(n\) elements as the sum of the two cases:
\[
\binom{n}{k} = \binom{n-1}{k-1} + \binom{n-1}{k}
\]
This completes the proof of Pascal's Identity.

\subsubsection{Pascal's Triangle}
Pascal's Triangle is a triangular array of numbers where each number is the sum of the two directly above it. The \(n\)-th row corresponds to the coefficients in the expansion of \((x + y)^n\) and also represents the binomial coefficients \(\binom{n}{k}\).

\begin{figure}[h]
    \centering
    \begin{tabular}{ccccccccccc}
        & & & & & 1 & & & & & \\
        & & & & 1 & & 1 & & & & \\
        & & & 1 & & 2 & & 1 & & & \\
        & & 1 & & 3 & & 3 & & 1 & & \\
        & 1 & & 4 & & 6 & & 4 & & 1 & \\
        1 & & 5 & & 10 & & 10 & & 5 & & 1 \\
    \end{tabular}
    \caption{Pascal's Triangle showing the first 6 rows}
    \label{fig:pascal}
\end{figure}

\noindent
Each number in Pascal's Triangle is the sum of the two numbers above it. The value at position $(n,k)$ is $\binom{n}{k}$, where $n$ is the row number (starting from 0) and $k$ is the position within the row (also starting from 0).
\

\paragraph{The Connection:}
Pascal's Triangle is a direct visual representation of Pascal's Identity. The structure of the triangle is built using this identity as a fundamental rule. When constructing the triangle row by row, we start with 1 at the top and place 1's at both edges of each row (since $\binom{n}{0} = \binom{n}{n} = 1$). Every inner entry is then obtained by adding the two numbers directly above it—precisely following Pascal's Identity.

For example, examining row 4 ($n = 4$), the middle entry is $\binom{4}{2}$. By Pascal's Identity, this can be calculated as:
\[ \binom{4}{2} = \binom{3}{1} + \binom{3}{2} \]

Looking at the triangle, we can verify this: $3 + 3 = 6$, which is exactly the middle number in row 4. This pattern continues throughout the entire triangle, making it an elegant visual proof of Pascal's Identity and a useful tool for quickly calculating binomial coefficients.


\subsection{Binomial Theorm}
\begin{theorembox}[title=Binomial Theorem]
    For any integer \(n \geq 0\),
    \[
    (x + y)^n = \sum_{k=0}^{n} \binom{n}{k} x^{n-k} y^k
    \]
\end{theorembox}
\paragraph{Proof:}

\subsection{Binomial Theorem}
\begin{theorembox}[title=Binomial Theorem]
    For any positive integer \(n\),
    \[
    (x+y)^n = \sum_{k=0}^{n} \binom{n}{k} x^{n-k} y^k
    \]
\end{theorembox}
\paragraph{Proof:} Consider the product $(x+y)^n = \underbrace{(x+y)\,(x+y)\,\cdots\,(x+y)}_{\text{$n$ factors}}$.

When expanding this product, we must choose either $x$ or $y$ from each factor. To generate a term $x^{n-k}y^k$, we must select $y$ from exactly $k$ of the $n$ factors, and $x$ from the remaining $n-k$ factors.

The number of ways to select which $k$ factors contribute a $y$ is given by the binomial coefficient $\binom{n}{k}$. Thus, in the expansion, the coefficient of $x^{n-k}y^k$ is $\binom{n}{k}$.

Therefore:
\[
(x+y)^n = \sum_{k=0}^{n} \binom{n}{k}x^{n-k}y^k
\]

\begin{example}
For $n=3$:
\begin{itemize}[leftmargin=1.25em]
    \item $x^3$: choose $x$ from all factors (1 way)
    \item $x^2y$: choose $y$ from exactly 1 factor ($\binom{3}{1}=3$ ways)
    \item $xy^2$: choose $y$ from exactly 2 factors ($\binom{3}{2}=3$ ways)
    \item $y^3$: choose $y$ from all factors (1 way)
\end{itemize}

Thus $(x+y)^3 = x^3 + 3x^2y + 3xy^2 + y^3$, confirming the theorem.
\end{example}

\paragraph{Counting Principle Extension for Choosing Multiple Subsets}
We can extend our combinatorial principles to solve more complex problems, such as dividing a set into multiple groups. This directly connects to the concept of multinomial coefficients, which we'll explore next.

\subsection{Multinomial Coefficients}
\begin{definitionbox}[title=Multinomial Coefficient]
    Let $n$ and $r$ be positive integers, and let $n_1, n_2, \ldots, n_r$ be non-negative integers such that $n_1 + n_2 + \ldots + n_r = n$. The multinomial coefficient is defined as:
    \[
    \binom{n}{n_1, n_2, \ldots, n_r} = \frac{n!}{n_1! n_2! \cdots n_r!}
    \]
    
    This represents the number of ways to partition $n$ distinct objects into $r$ distinct groups, where group $i$ contains exactly $n_i$ objects.
\end{definitionbox}

\begin{examplebox}[title=Example: Committee Formation]
    A class of 20 students needs to form an executive committee with a president, vice president, secretary, and treasurer, and also needs to select 5 students for a planning committee and 6 students for a social committee. The remaining 5 students will not be on any committee. How many different ways can the committees be formed?
    
    \textbf{Solution:} This is a problem of partitioning 20 distinct students into 4 groups:
    \begin{itemize}
        \item 4 students for the executive committee
        \item 5 students for the planning committee
        \item 6 students for the social committee
        \item 5 students not on any committee
    \end{itemize}
    
    The number of ways to partition the students is:
    \[
    \binom{20}{4, 5, 6, 5} = \frac{20!}{4!5!6!5!} = 1,646,492,110,720
    \]
    
    However, within the executive committee, we must also assign specific roles. The 4 students can be arranged in $4! = 24$ ways. Therefore, the total number of different committee formations is:
    \[
    \frac{20!}{4!5!6!5!} \times 4! = \frac{20!}{5!6!5!} = 39,515,810,657,280
    \]
\end{examplebox}





\subsection{Multinomial Theorem}
\begin{theorembox}[title=Multinomial Theorem]
    For any positive integer \(n\) and any integer \(k \geq 2\),
    \[
    (x_1 + x_2 + \ldots + x_k)^n = \sum_{n_1+n_2+\ldots+n_k=n} \frac{n!}{n_1! n_2! \ldots n_k!} x_1^{n_1} x_2^{n_2} \ldots x_k^{n_k}
    \]
\end{theorembox}




\section{Applications of Combinatorial Analysis}
Combinatorial analysis has applications in various fields, including computer science, statistics, and operations research. Some common applications include:
\begin{itemize}
    \item Cryptography: Designing secure communication systems. For example, in RSA encryption, large prime numbers are used to create keys where the security relies on the computational difficulty of finding the prime factorization of large numbers.
    
    \item Network Design: Optimizing the layout of networks. For instance, determining the minimum number of routers needed to connect all computers in an office while minimizing cable length is a combinatorial optimization problem.
    
    \item Game Theory: Analyzing strategies in competitive situations. For example, in poker, calculating the probability of certain card combinations helps players make optimal betting decisions based on possible opponent hands.
    
    \item Probability Theory: Calculating probabilities in complex scenarios. For instance, in quality control, determining the probability of finding defective items in a batch using sampling techniques relies on combinatorial calculations.
\end{itemize}

\chapter{Probability}

Luck. Coincidence. Randomness. Uncertainty. Risk. Doubt. Fortune. Chance. You’ve probably heard these words countless times, but chances are that they were
used in a vague, casual way. Unfortunately, despite its ubiquity in science and everyday life, probability can be deeply counterintuitive.

\section{Inspiration and Overview}
\begin{exampleboxbreak}{Probability in Healthcare}
\begin{dialogue}
\speak{RELATIVE} Nurse, what is the probability that the drug will work? \\
\speak{NURSE} I hope it works, we'll know tomorrow. \\
\speak{RELATIVE} Yes, but what is the probability that it will? \\ 
\speak{NURSE} Each case is different, we have to wait. \\
\speak{RELATIVE} But let's see, out of a hundred patients that are treated under similar conditions, how many times would you expect it to work? \\
\speak{NURSE} \emph{(somewhat annoyed)} I told you, every person is different, for some it works, for some it doesn't. \\
\speak{RELATIVE} \emph{(insisting)} Then tell me, if you had to bet whether it will work or not, which side of the bet would you take? \\
\speak{NURSE} \emph{(cheering up for a moment)} I'd bet it will work. \\
\speak{RELATIVE} \emph{(somewhat relieved)} OK, now, would you be willing to lose two dollars if it doesn't work, and gain one dollar if it does? \\
\speak{NURSE} \emph{(exasperated)} What a sick thought! You are wasting my time!
\end{dialogue}
\end{exampleboxbreak}

In this conversation, the relative attempts to use the concept of probability to discuss an uncertain situation—whether the drug will work for their loved one. The nurse's initial responses suggest that the meaning of ``probability'' is not uniformly understood, as they focus on individual differences rather than likelihood. 

The relative tries multiple approaches to make the concept more concrete:
\begin{itemize}
    \item First, requesting a frequency interpretation (``out of a hundred patients\ldots'')
    \item Then, asking for a binary prediction (which side of a bet would the nurse take?)
    \item Finally, attempting to establish specific odds (the two-to-one bet)
\end{itemize}

The nurse may not be entirely wrong in refusing to discuss in such terms—what if this were an experimental drug administered for the very first time in this hospital or in the nurse's experience? 

This dialogue exemplifies how probability can be viewed through different lenses: as statistical frequencies based on past events, or as subjective beliefs about unique circumstances. The relative's final attempt to establish a bet based on these probabilities reveals an important application of probability theory—decision-making under uncertainty and risk assessment. The nurse's discomfort with this proposition highlights how probability concepts, while mathematically sound, can create social friction when applied to medical contexts where uncertainty is typically managed through standardized protocols rather than explicit probability assessments.

\headingA{Frequency vs. Subjective Interpretations}

While there are many situations involving uncertainty in which the frequency interpretation is appropriate, there are other situations where it is not applicable. Consider, for example, a scholar who asserts that the Iliad and the Odyssey were composed by the same person, with probability 90\%. Such an assertion conveys meaningful information, but not in terms of frequencies, since the subject is a one-time historical event. Rather, it is an expression of the scholar's subjective belief based on evidence and expertise.

\headingA{The Importance of Subjective Probability}

One might initially dismiss subjective beliefs as uninteresting from a mathematical or scientific perspective. However, people frequently make choices in the presence of uncertainty, and a systematic framework for utilizing these beliefs is essential for successful—or at least consistent—decision making.

\headingA{Probability in Action}

This distinction between frequency-based and subjective interpretations highlights the versatility of probability theory. In the opening dialogue, we see both interpretations at work:
\begin{itemize}
    \item The relative first seeks a frequency-based answer (``out of a hundred patients'')
    \item Then pivots to eliciting the nurse's subjective belief about this particular treatment's effectiveness
\end{itemize}

Had the nurse been willing to accept a one-for-one bet that the drug would work, we could infer that the nurse judged the probability of success to be at least \(50\%\). Had the nurse accepted the final proposed bet (two-for-one), this would have indicated a success probability of at least \(2/3\).

\section{Set Theory Review}

\begin{definitionboxbreak}{Set}
A set is a collection of distinct objects, considered as an object in its own right.
\begin{itemize}
    \item Sets are typically denoted by curly braces, e.g., \( S = \{1, 2, 3\} \).
    \item The objects within a set are called elements or members. For example, in the set \( S \), the number \( 1 \) is an element of \( S \), denoted as \( 1 \in S \) and if \( 4 \) is not in \( S \), we write \( 4 \notin S \).
    \item A set can have any number of elements, including none (the empty set, denoted \( \emptyset \)).
    \item If \(S\) contains a finite number of elements, we say \(S = \{1,2,3\}\) is a finite set.
    \item If \(S\) contains infinite elements, such as the set of all natural numbers \( \mathbb{N} = \{1, 2, 3, \ldots\} \), we say \(S\) is countably infinite.
\end{itemize}

\end{definitionboxbreak}

\newcommand{\circled}[1]{\tikz[baseline=(char.base)]{
            \node[shape=circle,draw,inner sep=1pt] (char) {#1};}}
            
\begin{keyconceptboxbreak}{Countable Infinity vs. Uncountably Infinite Sets}

    \headingB{Countably Infinite Sets:}
    \begin{definitionboxbreak}{Countably Infinite}
    A set $S$ is \textbf{countably infinite} if its elements can be put into a one-to-one correspondence (bijection) with the natural numbers $\mathbb{N} = \{1, 2, 3, \ldots\}$. In other words, the elements of $S$ can be arranged in a sequence: $s_1, s_2, s_3, \ldots$ where every element of $S$ appears exactly once in the sequence.
    \end{definitionboxbreak}
    
    The term ``countable'' refers to both finite sets and countably infinite sets, meaning we can count their elements one by one (even if the counting process never ends).

    \headingB{Examples of Countably Infinite Sets:}
    \begin{itemize}
        \item The set of integers $\mathbb{Z} = \{\ldots, -2, -1, 0, 1, 2, \ldots\}$ is countably infinite.
        \item The set of rational numbers $\mathbb{Q} = \{\frac{p}{q} : p, q \in \mathbb{Z}, q \neq 0\}$ is countably infinite.
    \end{itemize}

    \headingB{Uncountably Infinite Sets:}
    \begin{definitionboxbreak}{Uncountably Infinite}
    A set $S$ is \textbf{uncountably infinite} if it is infinite but cannot be put into a one-to-one correspondence with the natural numbers $\mathbb{N}$. In other words, the elements of $S$ cannot be arranged in a sequence where each element appears exactly once.
    \end{definitionboxbreak}
    
    Uncountable sets represent a ``larger'' kind of infinity than countable sets. The existence of different sizes of infinity was one of Cantor's most revolutionary discoveries.

    \headingB{Examples of Uncountable Sets:}
    \begin{itemize}
        \item The set of real numbers $\mathbb{R}$ is uncountably infinite.
        \item The set of all functions $f: \mathbb{N} \rightarrow \{0,1\}$ is uncountably infinite.
        \item The set of irrational numbers is uncountably infinite.
    \end{itemize}
\end{keyconceptboxbreak}

\begin{funfactsbreak}{Cantor's Infinite Hotel}

\headingB{Did you know?} There are different sizes of infinity! Georg Cantor proved this mind-bending fact in the 1870s, shocking the mathematical world.

\headingB{Imagine a Hotel with Infinite Rooms:}

\infoheading{Scenario 1:} The infinite hotel is full, but one new guest arrives. Can you accommodate them?

\infoheading{Solution:} Yes! Ask each guest to move from room $n$ to room $n+1$. This frees up room 1 for the new guest!

\infoheading{Scenario 2:} The infinite hotel is full, and a bus with infinite new guests arrives. Can you fit them all?

\infoheading{Solution:} Yes! Ask each current guest to move from room $n$ to room $2n$. This puts all current guests in even-numbered rooms, leaving all odd-numbered rooms free for the infinitely many new guests!

\headingB{Counting Rational Numbers}

Surprisingly, all rational numbers can be accommodated in our Infinite Hotel! They are what mathematicians call ``countably infinite''.

\begin{figure}[H]
    \centering
    \includegraphics[width=0.6\textwidth]{cantor_diag.png}
    \caption{Cantor's diagonal zigzag method for counting rational numbers}
    \label{fig:cantor_diag}
\end{figure}

\infoheading{Key Insight:} Cantor's diagonal zigzag pattern (shown in Figure~\ref{fig:cantor_diag}) creates a one-to-one correspondence between natural numbers and rational numbers. This clever counting method ensures we'll eventually reach any specific fraction after a finite number of steps, proving that rational numbers are countable—they can all fit in our Infinite Hotel!

\headingB{But Irrational Numbers Won't Fit!}

Cantor discovered that the real numbers (which include both rational and irrational numbers) form an \textit{uncountable} infinity. Here's why:


\headingB{The Diagonal Argument: Why Real Numbers Won't Fit}
To fully appreciate what ``countable'' means, it's crucial to see Cantor's proof that some infinite sets are uncountable. The most famous example is the set of real numbers ($\mathbb{R}$) between 0 and 1. An uncountable set is an infinite set for which no one-to-one correspondence with the natural numbers can ever be made.

\headingB{The Proof by Contradiction}

\headingB{Assume the Opposite}
Let's assume the set of real numbers between 0 and 1 is countable. If it's countable, we can write them all down in an infinite list, just like we did for the rationals. Let this hypothetical list be:
\begin{align*}
r_1 &= 0.\mathbf{d_{11}}d_{12}d_{13}d_{14}\dots \\
r_2 &= 0.d_{21}\mathbf{d_{22}}d_{23}d_{24}\dots \\
r_3 &= 0.d_{31}d_{32}\mathbf{d_{33}}d_{34}\dots \\
r_4 &= 0.d_{41}d_{42}d_{43}\mathbf{d_{44}}\dots \\
&\vdots
\end{align*}
..and so on, where $d_{ij}$ is the $j$-th decimal digit of the $i$-th number.

\headingB{Construct a New Number}
Now, we'll construct a new real number, let's call it $X$, which is \emph{not} on this list. We do this by going down the diagonal of our list (the bold digits) and changing each digit.
\begin{itemize}
\item The first decimal digit of $X$ will be different from the first digit of $r_1$ ($d_{11}$).
\item The second decimal digit of $X$ will be different from the second digit of $r_2$ ($d_{22}$).
\item The third decimal digit of $X$ will be different from the third digit of $r_3$ ($d_{33}$).
\item In general, the $n$-th digit of $X$ is different from the $n$-th digit of $r_n$.
\end{itemize}
Let's define a rule: If the diagonal digit $d_{nn}$ is 1, we make our new digit 2. If $d_{nn}$ is anything else, we make our new digit 1.

\paragraph{Example:}
Suppose our list starts like this:
\begin{align*}
r_1 &= 0.\mathbf{7}182\dots \\
r_2 &= 0.3\mathbf{1}41\dots \\
r_3 &= 0.88\mathbf{2}3\dots \\
r_4 &= 0.413\mathbf{5}\dots
\end{align*}
Our new number $X$ is constructed as follows:
\begin{itemize}
\item 1st digit of $X$ is not 7 (let's make it 1).
\item 2nd digit of $X$ is not 1 (let's make it 2).
\item 3rd digit of $X$ is not 2 (let's make it 1).
\item 4th digit of $X$ is not 5 (let's make it 1).
\end{itemize}
So, $X = 0.1211\dots$

\headingB{The Contradiction}
This new number $X$ is a real number between 0 and 1. But where is it on our list?
\begin{itemize}
\item It can't be $r_1$ because its first digit is different.
\item It can't be $r_2$ because its second digit is different.
\item It can't be $r_n$ for any $n$, because its $n$-th digit is different from $r_n$'s $n$-th digit.
\end{itemize}
So, $X$ is \textbf{not on the list}. But we started by assuming our list contained \emph{all} real numbers between 0 and 1. This is a contradiction.

\headingB{Conclusion}
Our initial assumption must be false. It is impossible to list all the real numbers between 0 and 1. Therefore, the set of real numbers is \textbf{uncountably infinite}. It represents a ``larger'' infinity than the infinity of integers or rational numbers.

\end{funfactsbreak}

\subsection{Set Operations}
We will denote the \textbf{universal set} by \( \Omega \), which contains all possible outcomes or elements under consideration. Subsets of \( \Omega \) are denoted by capital letters such as \( A \), \( B \), and \( C \).
The \textbf{complement} of a set \( S \), denoted \( S^c \) or \( \overline{S} \), is the set of all elements in the universal set \( \Omega \) that are not in \( S \):

\[
S^c = \{ x \in \Omega \mid x \notin S \}
\]
\[ \Omega^c = \emptyset   \]

The \textbf{union} of two sets \( A \) and \( B \), denoted \( A \cup B \), is the set of elements that are in \( A \), in \( B \), or in both:

\[
A \cup B = \{ x \mid x \in A \text{ or } x \in B \}
\]

The \textbf{intersection} of two sets \( A \) and \( B \), denoted \( A \cap B \), is the set of elements that are in both \( A \) and \( B \):

\[
A \cap B = \{ x \mid x \in A \text{ and } x \in B \}
\]

The \textbf{set difference} of \( A \) and \( B \), denoted \( A \setminus B \), is the set of elements that are in \( A \) but not in \( B \):

\[
A \setminus B = \{ x \mid x \in A \text{ and } x \notin B \} = A \cap B^c
\]

The \textbf{symmetric difference} of \( A \) and \( B \), denoted \( A \triangle B \), is the set of elements that are in exactly one of the sets \( A \) or \( B \):

\[
A \triangle B = (A \setminus B) \cup (B \setminus A) = (A \cup B) \setminus (A \cap B)
\]


% Include the combined set operations diagram generated with matplotlib
\begin{figure}[h]
    \centering
    \includegraphics[width=\textwidth]{figures/set_operations/all_operations.png}
    \caption{Visualizations of fundamental set operations}\label{fig:set_operations}
\end{figure}

The union or intersection of more or infinitely many sets is defined similarly. For example, the union of a collection of sets \( \{A_i\}_{i \in I} \) is:

\[
\bigcup_{i \in I} A_i = \{ x \mid x \in A_i \text{ for some } i \in I \}
\]

And the intersection is:

\[
\bigcap_{i \in I} A_i = \{ x \mid x \in A_i \text{ for all } i \in I \}
\]

If two sets \( A \) and \( B \) have no elements in common, they are called \textbf{disjoint}, and their intersection is the empty set:
\[ A \cap B = \emptyset \]  

A collection of sets \( \{A_i\}_{i \in I} \) is called a \textbf{partition} of a set \( S \) if:
\begin{itemize}
    \item The sets \( A_i \) are pairwise disjoint: \( A_i \cap A_j = \emptyset \) for all \( i \neq j \)
    \item The union of all sets in the collection equals \( S \): \( \bigcup_{i \in I} A_i = S \)
\end{itemize}   

% \begin{figure}[h]
%     \centering
%     \includegraphics[width=0.6\textwidth]{figures/set_operations/partition.png}
%     \caption{A partition of a set: disjoint subsets whose union equals the whole set}\label{fig:set-partition}
% \end{figure}

\subsubsection{Set Algebra}

\begin{identitiesboxbreak}{Set Theory}
For any sets \( A \), \( B \), and \( C \) within a universal set \( \Omega \), the following identities hold:
\begin{itemize}
    \item \textbf{Commutative Laws:}
    \[ A \cup B = B \cup A \]
    \[ A \cap B = B \cap A \]
    \item \textbf{Associative Laws:}
    \[ (A \cup B) \cup C = A \cup (B \cup C) \]
    \[ (A \cap B) \cap C = A \cap (B \cap C) \]
    \item \textbf{Distributive Laws:}
    \[ A \cup (B \cap C) = (A \cup B        ) \cap (A \cup C) \]
    \[ A \cap (B \cup C) = (A \cap B) \cup (A \cap C) \]
    \item \textbf{Identity Laws:}
    \[ A \cup \emptyset = A \]
    \[ A \cap \Omega = A \]
    \item \textbf{Domination Laws:}
    \[ A \cup \Omega = \Omega \]        
    \[ A \cap \emptyset = \emptyset \]      
    \item \textbf{Idempotent Laws:}
    \[ A \cup A = A \]
    \[ A \cap A = A \]
    \item \textbf{Complement Laws:}
    \[ A \cup A^c = \Omega \]
    \[ A \cap A^c = \emptyset \]
    \item \textbf{Double Complement Law:}
    \[ (A^c)^c = A \]
    \item \textbf{De Morgan's Laws:}
    \[ (A \cup B)^c = A^c \cap B^c \]
    \[ (A \cap B)^c = A^c \cup B^c \]
\end{itemize}
\end{identitiesboxbreak}

\section{Probablistic Models}

Probability is a mathematical descrption of an unknown phenomenon. It is used to predict the likelihood of various outcomes. A probablistic model consists of three main components:
\begin{itemize}
    \item A \textbf{sample space} \( \Omega \), which is the set of all possible outcomes.
    \item A set of \textbf{events}, which are subsets of the sample space.
    \item A \textbf{probability measure/probability law} \( P \), which assigns to a set \(A\) of possible outcomes called an event , a non negative number \( P(A) \) called the probability of \(A\) that encodes our knowledge or belief about the collective likelihood of the elements of \(A\). 

\subsection{Sample Space and Events}
\begin{itemize}
    \item Every probabilistic model involves an underlying process, called the \textbf{experiment}, that will produce exactly one out of several possible \textbf{outcomes}.
    \item The set of all possible outcomes is called the \textbf{sample space}, denoted by \( \Omega \).
    \item An \textbf{event} is any subset of the sample space, including the empty set and the sample space itself.
    \item Events can be combined using set operations such as union, intersection, and complement.
\end{itemize}

\subsubsection{Choosing a Sample Space}
When defining a probabilistic model, the choice of sample space is crucial. The sample space should:
\begin{itemize}
    \item Include all possible outcomes of the experiment.
    \item Be as specific as necessary for the problem at hand.
    \item Allow for the definition of all relevant events.
\end{itemize}

\begin{exampleboxbreak}{Choosing a Sample Space}
\headingB{Example 1: Rolling a Die}
\begin{itemize}
    \item Experiment: Rolling a fair six-sided die.
    \item Sample Space: \( \Omega = \{1, 2, 3, 4, 5, 6\} \).
    \item Events:
    \begin{itemize}
        \item \( A = \{2, 4, 6\} \) (rolling an even number)
        \item \( B = \{1, 3, 5\} \) (rolling an odd number)
        \item \( C = \{1, 2\} \) (rolling a number less than 3)
    \end{itemize}
\end{itemize}   
\headingB{Example 2: Flipping a Coin}
\begin{itemize}
    \item Experiment: Flipping a fair coin.
    \item Sample Space: \( \Omega = \{\text{Heads}, \text{Tails}\} \).
    \item Events:
    \begin{itemize}
        \item \( A = \{\text{Heads}\} \) (getting heads)
        \item \( B = \{\text{Tails}\} \) (getting tails)
    \end{itemize}
\end{itemize}
\headingB{Example 3: Drawing a Card from a Deck}
\begin{itemize} 
    \item Experiment: Drawing a card from a standard deck of 52 playing cards.
    \item Sample Space: \( \Omega = \{\text{2H}, \text{3H}, \ldots, \text{AH}, \text{2D}, \ldots, \text{AD}, \text{2C}, \ldots, \text{AC}, \text{2S}, \ldots, \text{AS}\} \) (where H, D, C, S represent hearts, diamonds, clubs, and spades respectively).
    \item Events:
    \begin{itemize}
        \item \( A = \{\text{All Hearts}\} = \{\text{2H}, \text{3H}, \ldots, \text{AH}\} \)
        \item \( B = \{\text{All Face Cards}\} = \{\text{JH}, \text{QH}, \text{KH}, \text{JD}, \text{QD}, \text{KD}, \text{JC}, \text{QC}, \text{KC}, \text{JS}, \text{QS}, \text{KS}\} \)
        \item \( C = \{\text{All Aces}\} = \{\text{AH}, \text{AD}, \text{AC}, \text{AS}\} \)
    \end{itemize}
\end{itemize}
\end{exampleboxbreak}

\subsubsection{Properties of a Well-Defined Sample Space}

A properly defined sample space must satisfy certain requirements to be useful in probabilistic modeling:

\begin{itemize}
    \item \textbf{Mutually Exclusive Outcomes}: Each outcome in the sample space must be distinct and non-overlapping. The outcomes must be defined so that when the experiment is conducted, exactly one outcome occurs.
    
    \item \textbf{Collectively Exhaustive Outcomes}: The sample space must include all possible outcomes of the experiment.
    
    \item \textbf{Appropriate Level of Detail}: The sample space should be defined with enough detail to answer the questions of interest, but not so detailed as to make the analysis unnecessarily complex.
\end{itemize}

\begin{exampleboxbreak}{Finding the Right Level of Detail in a Sample Space}
Consider a clinical trial testing the effectiveness of a new drug. Three possible approaches to defining the sample space illustrate the importance of choosing an appropriate level of detail:

\headingB{Approach 1: Too Little Detail}
\[ \Omega_1 = \{\text{``Drug works''}, \text{``Drug doesn't work''}\} \]

This sample space lacks sufficient detail because:
\begin{itemize}
    \item It doesn't capture varying degrees of effectiveness
    \item It ignores potential side effects
    \item It doesn't account for different patient responses
    \item It can't answer questions about the magnitude of improvement
\end{itemize}

\headingB{Approach 2: Too Much Detail}
\begin{align*}
\Omega_2 = \{\text{``Every possible molecular interaction in each patient''}\}
\end{align*}

This sample space has excessive detail because:
\begin{itemize}
    \item It includes information irrelevant to the drug's effectiveness
    \item It makes probability calculations practically impossible
    \item It obscures the clinically relevant outcomes
    \item It introduces unnecessary complexity that doesn't help answer the core question
\end{itemize}

\headingB{Approach 3: Appropriate Level of Detail}
\begin{align*}
\Omega_3 = \{\text{``No effect''}, \text{``Mild improvement''}, \text{``Moderate improvement''}, \\ 
\text{``Major improvement''}, \text{``Adverse reaction''}\}
\end{align*}

This sample space balances detail with practicality:
\begin{itemize}
    \item It captures clinically meaningful outcome categories
    \item It includes sufficient detail to evaluate effectiveness
    \item It allows for statistical analysis of results
    \item It directly addresses the question of interest without unnecessary complexity
\end{itemize}
\end{exampleboxbreak}

\begin{exampleboxbreak}{Improper Sample Space Definition}
Consider the roll of a standard six-sided die. The following would be an improper sample space:
\begin{align*}
\Omega = \{\text{``1 or 3''}, \text{``1 or 4''}, \text{``2''}, \text{``3''}, \text{``5''}, \text{``6''}\}
\end{align*}

This is improper because:
\begin{itemize}
    \item If the roll results in 1, we wouldn't know whether the outcome is ``1 or 3'' or ``1 or 4''
    \item The outcomes are not mutually exclusive (they overlap)
    \item When a specific experimental result occurs (e.g., rolling a 1), the outcome wouldn't be uniquely determined
\end{itemize}

A proper sample space for this experiment would be:
\[ \Omega = \{1, 2, 3, 4, 5, 6\} \]

With this definition, each possible roll corresponds to exactly one outcome in the sample space.
\end{exampleboxbreak}
\begin{exampleboxbreak}{Sample Space Design: Two Coin Flipping Games}
Consider two alternative games, both involving ten successive coin tosses:

\headingB{Game 1:} We receive \$1 each time a head comes up.

\headingB{Game 2:} We receive \$1 for every coin toss, up to and including the first time a head comes up. Then, we receive \$2 for every coin toss, up to the second time a head comes up. More generally, the dollar amount per toss is doubled each time a head comes up.

These games illustrate how the same experiment can be modeled with different levels of detail depending on what information we need:

\headingB{The Underlying Sample Space}
Both games involve the same physical experiment: ten successive coin tosses. The complete sample space for this experiment is:
\[ \Omega = \{\text{all possible sequences of H and T of length 10}\} \]
This sample space has $2^{10} = 1024$ possible outcomes.

\headingB{Sample Space for Game 1}
In Game 1, only the total number of heads matters, not their positions in the sequence. We can use a \textbf{simplified sample space}:
\[ \Omega_1 = \{0, 1, 2, \ldots, 10\} \]
where each outcome represents the number of heads obtained in the ten tosses. This simplified space has only 11 outcomes and is sufficient for calculating payouts in Game 1, since all sequences with the same number of heads yield the same payout.

\headingB{Sample Space for Game 2}
For Game 2, the order of heads and tails is crucial because payouts depend on when heads appear. We must use the complete sample space:
\[ \Omega_2 = \{\text{all possible sequences of H and T of length 10}\} \]

For example, the sequences HHTTTHTHTT and HTHTTTTHTH would yield different payouts in Game 2, even though both contain exactly 4 heads. According to the rules, we receive a payment for every coin toss, and the amount per toss doubles each time a head appears. 

For the sequence HHTTTHTHTT:
\begin{itemize}
    \item Toss 1 (H): \$1 (first head appears, rate becomes \$2 for subsequent tosses)
    \item Toss 2 (H): \$2 (second head appears, rate becomes \$4 for subsequent tosses)
    \item Toss 3 (T): \$4
    \item Toss 4 (T): \$4
    \item Toss 5 (T): \$4
    \item Toss 6 (H): \$4 (third head appears, rate becomes \$8 for subsequent tosses)
    \item Toss 7 (T): \$8
    \item Toss 8 (H): \$8 (fourth head appears, rate becomes \$16 for subsequent tosses)
    \item Toss 9 (T): \$16
    \item Toss 10 (T): \$16
\end{itemize}
Total: \$1 + \$2 + \$4 + \$4 + \$4 + \$4 + \$8 + \$8 + \$16 + \$16 = \$67.

For the sequence HTHTTTTHTH:
\begin{itemize}
    \item Toss 1 (H): \$1
    \item Toss 2 (T): \$2
    \item Toss 3 (H): \$2
    \item Toss 4 (T): \$4
    \item Toss 5 (T): \$4
    \item Toss 6 (T): \$4
    \item Toss 7 (T): \$4
    \item Toss 8 (H): \$4
    \item Toss 9 (T): \$8
    \item Toss 10 (H): \$8
\end{itemize}
Total: \$1 + \$2 + \$2 + \$4 + \$4 + \$4 + \$4 + \$4 + \$8 + \$8 = \$41.

Even though both sequences have the same number of heads, they yield different payouts (\$67 vs \$41) because the positions of the heads differ.

\headingB{Key Insight: Choosing the Right Level of Detail}

Both games share the same underlying sample space (all 1024 possible sequences), but we model them differently:
\begin{itemize}
    \item \textbf{Game 1:} We can use the simplified sample space $\Omega_1$ with only 11 outcomes because multiple sequences (those with the same number of heads) produce identical payouts. This simplification makes calculations much easier.

    \item \textbf{Game 2:} We must use the complete sample space $\Omega_2$ with all 1024 sequences because each distinct sequence may produce a different payout.
\end{itemize}
\end{exampleboxbreak}
\section{Probability Laws}
The probability law assigns to every event \(A\), a number \(P(A)\), called the \textbf{probability} of \(A\), satisfying the following axioms 

\begin{axiomboxbreak}{Probability Axioms}
    \begin{itemize}
        \item \textbf{Non-negativity:} For any event \(A\), \(P(A) \geq 0\).
        \item \textbf{Normalization:} \(P(\Omega) = 1\), where \(\Omega\) is the sample space.
        \item \textbf{Additivity:} If \(A\) and \(B\) are disjoint events (i.e., \(A \cap B = \emptyset\)), then 
        \[ P(A \cup B) = P(A) + P(B)\]
        If the sample space has an infinite number of elements
and \(A_1, A_2, \ldots\) is a sequence of disjoint events, then the probability of
their union satisfies
\[ P(A_1 \cup A_2 \cup \cdots) = P(A_1) + P(A_2) + \cdots \]
    \end{itemize}
\end{axiomboxbreak}

\begin{proof}
    
\end{proof}

\section{Sequential Models}

Many experiments have an inherently sequential character, such as for example tossing a coin three times, or observing the value of a stock on five successive days, or receiving eight successive digits at a communication receiver. It is then often useful to describe the experiment and the associated sample space by means of a tree-based sequential description.

\subsection{Tree-Based Sequential Description}

A \textbf{tree diagram} provides a visual representation of sequential experiments, where:
\begin{itemize}
    \item Each branch represents a possible outcome at a particular stage
    \item The path from the root to a leaf represents a complete sequence of outcomes
    \item The set of all paths (leaves) forms the sample space
\end{itemize}


Consider the experiment of tossing a fair coin three times. Each toss can result in either Heads (H) or Tails (T). The sample space can be represented as a tree:

\begin{figure}[h]
    \centering
    \includegraphics[width=0.9\textwidth]{sequential_tree.png}
    \caption{Tree diagram for three successive coin tosses}
    \label{fig:sequential_tree}
\end{figure}

\headingB{Reading the Tree:}
\begin{itemize}
    \item \textbf{Stage 1 (First Toss):} Two branches from the start—one for H and one for T
    \item \textbf{Stage 2 (Second Toss):} From each outcome of the first toss, two more branches emerge
    \item \textbf{Stage 3 (Third Toss):} From each outcome of the second toss, two final branches complete the sequences
\end{itemize}

\headingB{Sample Space:}
The complete sample space consists of all possible paths from the root to the leaves:
\[ \Omega = \{\text{HHH, HHT, HTH, HTT, THH, THT, TTH, TTT}\} \]

This gives us $2^3 = 8$ possible outcomes, where each path through the tree represents one complete sequence of three tosses.

\headingB{Key Advantages of Sequential Models:}
\begin{itemize}
    \item \textbf{Visual clarity:} Easy to see all possible outcomes and their relationships
    \item \textbf{Systematic enumeration:} Ensures no outcomes are missed or double-counted
    \item \textbf{Natural structure:} Matches the temporal order of events in the experiment
    \item \textbf{Probability calculation:} Facilitates computation of probabilities by multiplying along branches
\end{itemize}





\section{Discrete Models}

When the sample space \(\Omega\) contains a finite or countably infinite number of outcomes, we have a discrete probability model.

\begin{exampleboxbreak}{Coin Toss Experiment}
Consider an experiment involving a single coin toss. There are two possible outcomes, heads (H) and tails (T). The sample space is $\Omega = \{H, T\}$, and the events are:
\begin{itemize}
    \item The empty set: $\emptyset$
    \item The event of getting heads: $\{H\}$
    \item The event of getting tails: $\{T\}$
    \item The entire sample space: $\Omega = \{H, T\}$
\end{itemize}

A probability law for this experiment must assign probabilities to each of these events. If the coin is fair, we would assign:
\begin{align}
P(\{H\}) &= 0.5\\
P(\{T\}) &= 0.5
\end{align}

From the axioms, we must also have:
\begin{align}
P(\emptyset) &= 0\\
P(\Omega) &= P(\{H\} \cup \{T\}) = P(\{H\}) + P(\{T\}) = 0.5 + 0.5 = 1
\end{align}

If the coin is biased, we might have different probabilities, such as:
\begin{align}
P(\{H\}) &= 0.6\\
P(\{T\}) &= 0.4
\end{align}
but we still must have $P(\{H\}) + P(\{T\}) = 1$. 

\headingB{Three Coin Tosses}

Consider now an experiment involving three coin tosses. The outcome will be a 3-character string of heads or tails. The sample space is:
\[\Omega = \{\text{HHH}, \text{HHT}, \text{HTH}, \text{HTT}, \text{THH}, \text{THT}, \text{TTH}, \text{TTT}\}\]

For a fair coin, each outcome has equal probability:
\[P(\{\text{HHH}\}) = P(\{\text{HHT}\}) = \cdots = P(\{\text{TTT}\}) = \frac{1}{8} = 0.125\]

We can now calculate probabilities for various events:
\begin{itemize}
    \item The event of getting exactly 2 heads: $A = \{\text{HHT}, \text{HTH}, \text{THH}\}$
    \[P(A) = P(\{\text{HHT}\}) + P(\{\text{HTH}\}) + P(\{\text{THH}\}) = 3 \cdot \frac{1}{8} = \frac{3}{8}\]
    
    \item The event of getting at least 1 head: $B = \{\text{HHH}, \text{HHT}, \text{HTH}, \text{HTT}, \text{THH}, \text{THT}, \text{TTH}\}$
    \[P(B) = 7 \cdot \frac{1}{8} = \frac{7}{8}\]
    
    \item The event of getting the same outcome on all three tosses: $C = \{\text{HHH}, \text{TTT}\}$
    \[P(C) = P(\{\text{HHH}\}) + P(\{\text{TTT}\}) = \frac{1}{8} + \frac{1}{8} = \frac{1}{4}\]
\end{itemize}

\end{exampleboxbreak}


\subsection{Discrete Probability Law}

If the sample space consists of a finite number of possible outcomes, then the probability law is specified by the probabilities of the events that consist of a single element. In particular, the probability of any event is the sum of the probabilities of its elements.

\begin{definitionboxbreak}{Discrete Probability Law}
For a discrete sample space $\Omega = \{s_1, s_2, \ldots, s_n\}$, the probability law is specified by assigning probabilities to the singleton events $\{s_1\}, \{s_2\}, \ldots, \{s_n\}$ such that:
\begin{itemize}
    \item $P(\{s_i\}) \geq 0$ for all $i$ (non-negativity)
    \item $\sum_{i=1}^{n} P(\{s_i\}) = 1$ (normalization)
\end{itemize}
Then, for any event $A = \{s_1, s_2, \ldots, s_m\} \subseteq \Omega$, the probability of $A$ is given by:
\[P(\{s_1, s_2, \ldots, s_m\}) = P(\{s_1\}) + P(\{s_2\}) + \cdots + P(\{s_m\})\]
\end{definitionboxbreak}

\begin{definitionboxbreak}{Discrete Uniform Probability Law}
    If the sample space consists of $n$ possible outcomes which are equally likely (i.e., all single-element events have the same probability), then the probability of any event $A$ is given by
    \[P(A) = \frac{\text{Number of elements in }A}{n}\]

    This is often referred to as the classical probability formula, and it applies to situations where all outcomes are equally likely to occur.
\end{definitionboxbreak}

\begin{exampleboxbreak}{Dice Experiment}
Consider the experiment of rolling a pair of 4-sided dice. We assume the dice are fair, and we interpret this assumption to mean that all 16 possible outcomes are equally likely. The sample space is:
\[\Omega = \{(1,1), (1,2), (1,3), (1,4), (2,1), (2,2), (2,3), (2,4), (3,1), (3,2), (3,3), (3,4), (4,1), (4,2), (4,3), (4,4)\}\]
Since there are 16 equally likely outcomes, the probability of any single outcome is:
\[P(\{(i,j)\}) = \frac{1}{16} \quad \text{for all } (i,j) \in \Omega\]
\end{exampleboxbreak}


\subsection{Continuous Models}
Probabilistic models with continuous sample spaces differ from their discrete counterparts in that the probabilities of the single-element events may not be sufficient to characterize the probability law. In fact, in continuous sample spaces, single-element events typically have zero probability.



% filepath: book/chapters/chapter_probability.tex
\begin{exampleboxbreak}{Spinning a Wheel — Understanding Continuous Probability}
Imagine you have a game show wheel with numbers marked from $0$ to $1$. When you spin it, the pointer can land on any number in that range. Our sample space is $\Omega = [0,1]$.

\headingB{The Strange Case of Single Points}

Here's a puzzling question: What's the probability that the pointer lands on exactly $0.5$?

Your first instinct might be to say it has some small positive probability. But let's see why that can't be right:
\begin{itemize}
    \item Suppose landing on exactly $0.5$ has probability $c > 0$ (some positive number).
    \item By the same logic, landing on $0.1$, $0.2$, $0.3$, etc., each has the same probability $c$.
    \item If you pick enough distinct points (say $n$ points), their combined probability would be $n \times c$.
    \item For large enough $n$, this sum exceeds $1$, which violates the basic rule that total probability cannot exceed $1$.
\end{itemize}

Therefore, the only possibility is:
\[
P(\text{landing on exactly } x) = 0 \quad \text{for any specific number } x
\]

\headingB{How Do We Calculate Probabilities Then?}

Instead of asking about single points, we ask about \textbf{ranges}. For a fair wheel, the probability of landing in an interval equals the \textbf{length} of that interval:
\[
P(\text{landing between } a \text{ and } b) = b - a
\]

\textbf{Examples:}
\begin{itemize}
    \item $P(\text{landing between } 0.2 \text{ and } 0.5) = 0.5 - 0.2 = 0.3$ (or 30\%)
    \item $P(\text{landing between } 0 \text{ and } 0.25) = 0.25 - 0 = 0.25$ (or 25\%)
    \item $P(\text{landing between } 0.7 \text{ and } 1) = 1 - 0.7 = 0.3$ (or 30\%)
\end{itemize}

\headingB{Key Takeaways}

\begin{itemize}
  \item \textbf{Single points have zero probability} — even though they're possible outcomes! This seems strange, but it's how continuous probability works.

  \item \textbf{Ranges have positive probability} — the wider the range, the higher the probability (proportional to its length).

  \item \textbf{We can't just "add up" single points} — unlike with dice or coins where we add probabilities of individual outcomes, continuous models require us to think about intervals and ranges.
\end{itemize}

This is the fundamental difference between discrete probability (like coin flips) and continuous probability (like spinning a wheel).
\end{exampleboxbreak}

\begin{exampleboxbreak}{Romeo and Juliet's Date — A Two-Dimensional Continuous Problem}
Romeo and Juliet plan to meet at a certain location. Each will arrive with a delay between 0 and 1 hour, and all possible combinations of delays are equally likely. The first to arrive will wait for 15 minutes (0.25 hours) and then leave if the other hasn't shown up. What's the probability they actually meet?

\headingB{Setting Up the Problem}

Let's denote:
\begin{itemize}
    \item $x$ = Romeo's delay (in hours), where $0 \leq x \leq 1$
    \item $y$ = Juliet's delay (in hours), where $0 \leq y \leq 1$
\end{itemize}

Our sample space is the unit square: $\Omega = [0,1] \times [0,1]$ (all possible pairs of delays).

\headingB{When Do They Meet?}

They meet if and only if they arrive within 15 minutes (0.25 hours) of each other. This means:
\[
|x - y| \leq 0.25
\]

This inequality can be rewritten as two conditions:
\[
-0.25 \leq x - y \leq 0.25 \quad \Longleftrightarrow \quad y - 0.25 \leq x \leq y + 0.25
\]

\headingB{Visualizing the Solution}

In the $xy$-plane (where $x$ is Romeo's delay and $y$ is Juliet's delay), the sample space is a $1 \times 1$ square. The favorable region where they meet is the band between the lines $y = x - 0.25$ and $y = x + 0.25$.

\textbf{The total area of the sample space:} $1 \times 1 = 1$

\textbf{The area where they meet:} This is the area of the unit square minus two corner triangles.

Each corner triangle has:
\begin{itemize}
    \item Base and height = $1 - 0.25 = 0.75$
    \item Area = $\frac{1}{2} \times 0.75 \times 0.75 = \frac{0.5625}{2} = 0.28125$
\end{itemize}

Total area where they DON'T meet: $2 \times 0.28125 = 0.5625$

Area where they DO meet: $1 - 0.5625 = 0.4375$

\headingB{The Answer}

Since all points in the unit square are equally likely (uniform distribution), the probability equals the ratio of areas:
\[
P(\text{they meet}) = \frac{\text{Area where they meet}}{\text{Total area}} = \frac{0.4375}{1} = 0.4375 = \frac{7}{16}
\]

\textbf{Therefore, Romeo and Juliet have a 43.75\% chance of meeting!}

\headingB{Key Insights}

\begin{itemize}
    \item This is a \textbf{two-dimensional} continuous probability problem (unlike the wheel, which was one-dimensional).
    \item We calculate probability as the \textbf{ratio of areas} in the sample space.
    \item The geometric approach makes the solution visual and intuitive.
    \item See Figure~\ref{fig:romeo_juliet} for a visualization generated by the Python code in the supplementary materials.
\end{itemize}
\end{exampleboxbreak}

\begin{figure}[H]
    \centering
    \includegraphics[width=0.8\textwidth]{romeo_juliet_probability.png}
    \caption{Romeo and Juliet meeting probability visualization showing the sample space (unit square), the meeting region (green band where $|x-y| \leq 0.25$), and miss regions (red triangles). The colored dots represent a Monte Carlo simulation with 200 random scenarios: green dots show cases where they meet, red dots show cases where they miss each other.}
    \label{fig:romeo_juliet}
\end{figure}

\begin{figure}[H]
    \centering
    \includegraphics[width=\textwidth]{romeo_juliet_wait_times.png}
    \caption{Effect of different wait times on the probability of Romeo and Juliet meeting. The visualization shows how the meeting probability increases as the wait time increases from 6 minutes (10\%) to 45 minutes (75\%).}
    \label{fig:romeo_juliet_wait_times}
\end{figure}

\begin{keyconceptboxbreak}{Monte Carlo Simulation: Computing Probabilities Through Random Sampling}

\headingB{What is Monte Carlo Simulation?}

Monte Carlo simulation is a powerful computational technique that uses \textbf{random sampling} to solve problems and estimate probabilities—especially when exact mathematical solutions are difficult or impossible to find. The name comes from the famous Monte Carlo Casino in Monaco, reflecting the role of randomness in the method.

\headingB{The Core Idea}

Instead of calculating probability through mathematical formulas, we:
\begin{enumerate}
    \item \textbf{Simulate} the random experiment many times (thousands or millions)
    \item \textbf{Count} how often the event of interest occurs
    \item \textbf{Estimate} the probability as:
    \[
    P(\text{event}) \approx \frac{\text{Number of times event occurs}}{\text{Total number of simulations}}
    \]
\end{enumerate}

\headingB{Why Does This Work? The Law of Large Numbers}

The mathematical foundation is the \textbf{Law of Large Numbers}, which states:

\textit{As the number of independent trials of a random experiment increases, the observed relative frequency of an event approaches its true probability.}

In simple terms: If you flip a fair coin 10 times, you might get 7 heads. But if you flip it 10,000 times, you'll get very close to 5,000 heads (50\%). The more trials, the more accurate your estimate becomes.

\headingB{Monte Carlo for Romeo and Juliet Problem}

Let's see how Monte Carlo simulation works for our Romeo and Juliet meeting problem, where we calculated the theoretical probability as $\frac{7}{16} = 0.4375$ (43.75\%).

\textbf{Step-by-Step Process:}

\begin{enumerate}
    \item \textbf{Generate Random Arrival Times}
    \begin{itemize}
        \item For each simulation trial, randomly pick Romeo's delay: $x \sim \text{Uniform}(0, 1)$
        \item Randomly pick Juliet's delay: $y \sim \text{Uniform}(0, 1)$
        \item Each number is equally likely anywhere between 0 and 1 hour
    \end{itemize}

    \item \textbf{Check the Meeting Condition}
    \begin{itemize}
        \item Calculate the difference: $|x - y|$
        \item If $|x - y| \leq 0.25$ (within 15 minutes), they meet → count this as a success
        \item Otherwise, they miss each other
    \end{itemize}

    \item \textbf{Repeat Many Times}
    \begin{itemize}
        \item Run steps 1-2 for $N$ trials (e.g., $N = 10{,}000$)
        \item Keep track of successes (meetings) and failures (misses)
    \end{itemize}

    \item \textbf{Estimate the Probability}
    \[
    \hat{P}(\text{meet}) = \frac{\text{Number of meetings}}{N}
    \]
\end{enumerate}

\textbf{Example Results:}
\begin{itemize}
    \item With 100 simulations: might get 39 meetings → $\hat{P} = 0.39$ (39\%)
    \item With 1,000 simulations: might get 441 meetings → $\hat{P} = 0.441$ (44.1\%)
    \item With 10,000 simulations: might get 4,368 meetings → $\hat{P} = 0.4368$ (43.68\%)
    \item With 100,000 simulations: might get 43,742 meetings → $\hat{P} = 0.43742$ (43.742\%)
\end{itemize}

Notice how the estimate gets closer to the true value (0.4375) as we increase the number of simulations!

\headingB{Python Implementation}

Here's a complete Python program that implements Monte Carlo simulation for this problem:

\end{keyconceptboxbreak}

\begin{codeblock}{Python: Monte Carlo Simulation for Romeo and Juliet}
\begin{lstlisting}
import numpy as np
import matplotlib.pyplot as plt

def monte_carlo_romeo_juliet(n_simulations=10000, wait_time=0.25):
    """
    Monte Carlo simulation for Romeo and Juliet meeting problem.

    Parameters:
    -----------
    n_simulations : int
        Number of random trials to run
    wait_time : float
        How long they wait (in hours). Default: 0.25 (15 minutes)

    Returns:
    --------
    estimated_probability : float
        Estimated probability of meeting
    """
    # Step 1: Generate random arrival times
    # np.random.uniform(low, high, size) generates random numbers
    # uniformly distributed between 'low' and 'high'
    romeo_arrivals = np.random.uniform(0, 1, n_simulations)
    juliet_arrivals = np.random.uniform(0, 1, n_simulations)

    # Step 2: Check meeting condition for each pair
    # They meet if |romeo_time - juliet_time| <= wait_time
    time_differences = np.abs(romeo_arrivals - juliet_arrivals)
    they_meet = time_differences <= wait_time

    # Step 3: Count successes
    number_of_meetings = np.sum(they_meet)

    # Step 4: Calculate probability estimate
    estimated_prob = number_of_meetings / n_simulations

    return estimated_prob, romeo_arrivals, juliet_arrivals, they_meet

# Run the simulation
np.random.seed(42)  # For reproducible results
prob, romeo, juliet, meet = monte_carlo_romeo_juliet(10000)

# Display results
print(f"Monte Carlo Simulation Results:")
print(f"Number of simulations: 10,000")
print(f"Number of meetings: {np.sum(meet)}")
print(f"Estimated probability: {prob:.4f} ({prob*100:.2f}%)")
print(f"\nTheoretical probability: 7/16 = {7/16:.4f} ({7/16*100:.2f}%)")
print(f"Estimation error: {abs(prob - 7/16):.4f}")

# Visualize convergence: How estimate improves with more trials
def show_convergence():
    """Show how Monte Carlo estimate converges to true value."""
    sample_sizes = [10, 50, 100, 500, 1000, 5000, 10000, 50000]
    estimates = []

    for n in sample_sizes:
        prob, _, _, _ = monte_carlo_romeo_juliet(n)
        estimates.append(prob)

    # Plot
    plt.figure(figsize=(10, 6))
    plt.semilogx(sample_sizes, estimates, 'bo-', label='MC Estimate')
    plt.axhline(y=7/16, color='r', linestyle='--',
                label='True Probability (7/16)')
    plt.xlabel('Number of Simulations')
    plt.ylabel('Estimated Probability')
    plt.title('Monte Carlo Convergence')
    plt.grid(True, alpha=0.3)
    plt.legend()
    plt.show()

show_convergence()
\end{lstlisting}
\end{codeblock}

\begin{keyconceptboxbreak}{Understanding the Code}

\textbf{Key Python Concepts Used:}

\begin{enumerate}
    \item \texttt{np.random.uniform(0, 1, n)}: Generates $n$ random numbers uniformly distributed between 0 and 1. This simulates arrival times.

    \item \texttt{np.abs(array)}: Computes absolute value element-wise. We use this to find $|x - y|$ for all pairs.

    \item \texttt{array <= value}: Creates a boolean array (True/False values). Each element is True if the condition holds, False otherwise.

    \item \texttt{np.sum(boolean\_array)}: Counts how many True values exist. This gives us the number of meetings.
\end{enumerate}

\textbf{Why Monte Carlo is Powerful:}

\begin{itemize}
    \item \textbf{Versatility}: Works for problems where exact solutions are unknown or very complex
    \item \textbf{Intuitive}: Mirrors how we might test something in real life (run experiments)
    \item \textbf{Parallelizable}: Can run simulations simultaneously on multiple computers
    \item \textbf{Visualization}: Provides actual sample points we can plot and analyze
    \item \textbf{Validation}: Confirms theoretical results or finds errors in calculations
\end{itemize}

\textbf{Real-World Applications:}

\begin{itemize}
    \item \textbf{Finance}: Stock price modeling, option pricing, risk assessment
    \item \textbf{Physics}: Particle simulations, quantum mechanics computations
    \item \textbf{Engineering}: Reliability analysis, stress testing systems
    \item \textbf{Medicine}: Clinical trial design, drug dosage optimization
    \item \textbf{Machine Learning}: Training neural networks, reinforcement learning
    \item \textbf{Climate Science}: Weather prediction, climate change modeling
\end{itemize}

\textbf{Trade-offs:}

\begin{itemize}
    \item[$+$] Can solve very complex problems
    \item[$+$] Easy to implement and understand
    \item[$+$] Accuracy improves with more simulations
    \item[$-$] Requires many trials for high accuracy (computational cost)
    \item[$-$] Only gives approximate answers, not exact solutions
    \item[$-$] Need good random number generators
\end{itemize}

\end{keyconceptboxbreak}

\begin{definitionboxbreak}{Continuous Probability Models}
A continuous probability model involves a sample space $\Omega$ that contains uncountably many outcomes, typically corresponding to one or more continuous variables. In such models:
\begin{itemize}
    \item The probability of any single point is typically zero: $P(\{x\}) = 0$ for any $x \in \Omega$
    \item Probabilities are assigned to intervals or regions rather than individual points
    \item The probability law is often specified through a probability density function (PDF)
\end{itemize}
\end{definitionboxbreak}

\subsubsection{Properties of Probability Laws}

From the axioms of probability, we can derive several important properties that hold for any probability law:
\begin{itemize}
    \item \textbf{Complement Rule:} For any event \(A\), the probability of its complement \(A^c\) is given by:
    \[ P(A^c) = 1 - P(A) \]
    \item \textbf{Monotonicity:} If \(A \subseteq B\), then:
    \[ P(A) \leq P(B) \]
    \item \textbf{Inclusion-Exclusion Principle:} For any two events \(A\) and \(B\):
    \[ P(A \cup B) = P(A) + P(B) - P(A \cap B) \]
    \item \textbf{Union Bound:} For any events \(A_1, A_2, \ldots, A_n\):
    \[ P\left(\bigcup_{i=1}^{n} A_i\right) \leq \sum_{i=1}^{n} P(A_i) \]
\end{itemize}

\subsection{The Two-Stage Framework of Probability}

Applying probability theory to real-world uncertainty involves two distinct stages:

\headingB{Stage 1: Model Construction (Art)}
\begin{itemize}
    \item Build a probabilistic model by specifying a sample space and probability law
    \item No rigid rules—must satisfy only the three axioms
    \item Reasonable people may disagree on the "best" model
    \item Often involves tradeoffs: accuracy vs. simplicity vs. tractability
    \item May intentionally use "incorrect" but useful models
\end{itemize}

\headingB{Stage 2: Mathematical Analysis (Science)}
\begin{itemize}
    \item Work within the fully specified model
    \item Calculate probabilities of events or derive properties
    \item Governed strictly by logic and probability axioms
    \item All questions have precise answers—only skill required
    \item May be complex, but never ambiguous
\end{itemize}

\begin{exampleboxbreak}{Two-Stage Framework in Action}

\headingB{Scenario:} A tech company wants to model user clicks on their website.

\headingB{Stage 1 — Model Construction:}

Three data scientists propose different models:

\textbf{Model A (Simple):}
\begin{itemize}
    \item Sample space: $\Omega = \{\text{Click}, \text{No Click}\}$
    \item Probability law: $P(\text{Click}) = 0.15$ (based on historical average)
    \item \textit{Justification:} Simple, easy calculations, captures overall behavior
\end{itemize}

\textbf{Model B (Time-Aware):}
\begin{itemize}
    \item Sample space: $\Omega = \{\text{Click}, \text{No Click}\} \times \{\text{Morning}, \text{Afternoon}, \text{Evening}\}$
    \item Different click probabilities for different times of day
    \item \textit{Justification:} More accurate, accounts for daily patterns
\end{itemize}

\textbf{Model C (Complex):}
\begin{itemize}
    \item Sample space includes user demographics, previous behavior, device type, etc.
    \item Probability law based on machine learning model
    \item \textit{Justification:} Most accurate, but computationally expensive
\end{itemize}

\textit{Who is right?} All three! The choice depends on:
\begin{itemize}
    \item What questions need answering
    \item Available computational resources
    \item Required accuracy level
    \item Time constraints
\end{itemize}

\headingB{Stage 2 — Mathematical Analysis:}

Once Model A is chosen, questions have definite answers:
\begin{itemize}
    \item Q: ``What's the probability of 3 clicks out of 10 users?''
    \item A: Using binomial formula: $P(X=3) = \binom{10}{3}(0.15)^3(0.85)^7 = 0.1298$
\end{itemize}

No ambiguity here—the model determines the answer uniquely.

\end{exampleboxbreak}

\begin{exampleboxbreak}{The Danger of Ambiguous Models}

\headingB{The Birthday Paradox "Paradox"}

\textbf{Claim:} ``In a room of 30 people, what's the probability two share a birthday?''

\textbf{Two analysts disagree:}

\textit{Analyst 1:} ``The probability is about 70\%—I calculated it using the standard birthday problem formula.''

\textit{Analyst 2:} ``No, it's much lower! I got 7.7\%.''

\headingB{What went wrong?}

\textbf{Hidden Model Assumptions:}

\textit{Analyst 1's Model:}
\begin{itemize}
    \item Event: ``At least two people share some birthday''
    \item Assumes: All birthdays equally likely, no twins, 365 days/year
    \item Answer: $P \approx 0.706$ (70.6\%)
\end{itemize}

\textit{Analyst 2's Model:}
\begin{itemize}
    \item Event: ``At least one person shares \textit{my specific} birthday''
    \item This is a different question!
    \item Answer: $P = 1 - (364/365)^{29} \approx 0.077$ (7.7\%)
\end{itemize}

\headingB{The Lesson:}

There's no "paradox"—just \textbf{poorly specified models}. Once we clarify:
\begin{itemize}
    \item What exactly is the sample space?
    \item What precisely is the event of interest?
    \item What are our assumptions?
\end{itemize}

The apparent contradiction disappears. Stage 1 (model specification) must be unambiguous before Stage 2 (calculation) can proceed.

\end{exampleboxbreak}

\begin{keyconceptboxbreak}{Avoiding Probability "Paradoxes"}

\textbf{Key Insight:} Probability theory contains many apparent "paradoxes" where different methods seem to give different answers to the same question.

\textbf{The Truth:} These paradoxes invariably reflect:
\begin{itemize}
    \item Poorly specified probabilistic models
    \item Ambiguous problem statements
    \item Hidden or unstated assumptions
    \item Different interpretations of the same verbal description
\end{itemize}

\textbf{Resolution Strategy:}
\begin{enumerate}
    \item \textbf{Explicitly state} the sample space $\Omega$
    \item \textbf{Precisely define} all events of interest
    \item \textbf{Clearly specify} the probability law and its assumptions
    \item \textbf{Verify} that all parties agree on the model before calculating
\end{enumerate}

Once Stage 1 is unambiguous, Stage 2 has no room for disagreement—mathematics is deterministic.

\end{keyconceptboxbreak}

\begin{funfactsbreak}{Markov Chains: When the Past Doesn't Matter}

\headingB{The Memoryless Property}

Imagine a drunk person wandering on a street, taking random steps left or right. Where they go next depends only on where they are \textit{right now}—not on how they got there. This is the essence of a \textbf{Markov chain}!

\headingB{What is a Markov Chain?}

A Markov chain is a mathematical system that transitions from one state to another according to certain probabilistic rules. The key property:
\begin{center}
\textit{``The future depends only on the present, not on the past.''}
\end{center}

Formally: $P(\text{Next state} \mid \text{Current state, Past states}) = P(\text{Next state} \mid \text{Current state})$

\headingB{Real-World Examples:}

\begin{itemize}
    \item \textbf{Weather Prediction:} If it's sunny today, there's a 70\% chance it's sunny tomorrow and 30\% chance of rain—regardless of whether it rained last week.

    \item \textbf{Google's PageRank:} The algorithm that ranks web pages models a random web surfer clicking links. The probability of visiting a page depends only on the current page, not the browsing history.

    \item \textbf{Stock Prices (Efficient Market Hypothesis):} Some models assume tomorrow's price depends only on today's price, not on the entire price history.

    \item \textbf{Board Games:} In Monopoly, your next position depends only on where you are now and the dice roll—not on your previous moves.

    \item \textbf{Text Generation (AI):} Simple text prediction models (like early autocomplete) choose the next word based only on the current word.
\end{itemize}

\headingB{A Simple Example: The Weather Model}

Suppose we have two states: Sunny (S) and Rainy (R). The transition probabilities are:
\begin{itemize}
    \item If today is Sunny: 80\% chance tomorrow is Sunny, 20\% chance of Rain
    \item If today is Rainy: 50\% chance tomorrow is Sunny, 50\% chance of Rain
\end{itemize}

We can represent this as a \textbf{transition matrix}:
\[
P = \begin{pmatrix}
0.8 & 0.2 \\
0.5 & 0.5
\end{pmatrix}
\]
where rows represent current state and columns represent next state.

\begin{figure}[H]
    \centering
    \begin{tikzpicture}[->, >=stealth, auto, node distance=3cm, thick]
        % States
        \node[circle, draw, fill=yellow!30, minimum size=1.2cm] (S) {Sunny};
        \node[circle, draw, fill=blue!30, minimum size=1.2cm, right of=S, node distance=5cm] (R) {Rainy};

        % Transitions
        \path (S) edge [loop above] node {0.8} (S)
              (S) edge [bend left] node {0.2} (R)
              (R) edge [bend left] node {0.5} (S)
              (R) edge [loop above] node {0.5} (R);
    \end{tikzpicture}
    \caption{Markov chain state diagram for the weather model}
    \label{fig:markov_weather}
\end{figure}

\headingB{Interesting Fact:}

Many Markov chains reach a \textbf{steady state} (equilibrium) where the long-term probabilities stabilize. For our weather example, no matter what the weather is today, after many days the probability converges to approximately 71.4\% Sunny and 28.6\% Rainy!

\headingB{Modern Applications:}

\begin{itemize}
    \item \textbf{Machine Learning:} Hidden Markov Models (HMMs) for speech recognition
    \item \textbf{Finance:} Modeling credit rating transitions
    \item \textbf{Biology:} DNA sequence analysis
    \item \textbf{Queueing Theory:} Modeling customer service systems
    \item \textbf{Physics:} Statistical mechanics and thermodynamics
\end{itemize}

Named after Russian mathematician Andrey Markov (1856-1922), who first studied these chains in the early 1900s while analyzing the patterns of vowels and consonants in Russian literature!

\end{funfactsbreak}

\section{Exercises}


% \include{chapters/chapter_statistics}

\end{document}