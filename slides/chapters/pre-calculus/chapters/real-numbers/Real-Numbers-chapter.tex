\chapter{Algebra of Real Numbers}
\author{Nithin}

\section{Properties of Real Numbers}
\begin{itemize}
    \item \textbf{Commutative Properties}
    \begin{itemize}
        \item Addition: $a + b = b + a$
        \item Multiplication: $a \cdot b = b \cdot a$
    \end{itemize}
    \vspace{5pt}

    \item \textbf{Associative Properties}
    \begin{itemize}
        \item Addition: $(a + b) + c = a + (b + c)$
        \item Multiplication: $(a \cdot b) \cdot c = a \cdot (b \cdot c)$
    \end{itemize}
    \vspace{5pt}
    \item \textbf{Distributive Property}
    \begin{itemize}
        \item $a \cdot (b + c) = (a \cdot b) + (a \cdot c)$
    \end{itemize}
    \vspace{5pt}
    \item \textbf{Identity Elements}
    \begin{itemize}
        \item Additive Identity: $a + 0 = a$
        \item Multiplicative Identity: $a \cdot 1 = a$
    \end{itemize}
    \vspace{5pt}
    \item \textbf{Inverse Elements}
    \begin{itemize}
        \item Additive Inverse: $a + (-a) = 0$
        \item Multiplicative Inverse (if $a \neq 0$): $a \cdot \frac{1}{a} = 1$
    \end{itemize}
\end{itemize}

\begin{itemize}
    \vspace{5pt}
    \item \textbf{Closure Property}
    \begin{itemize}
        \item Real numbers are closed under addition, subtraction, multiplication, and division (except division by zero).
    \end{itemize}
\end{itemize}

\section{Inequalities,Intervals and Absolute Value}
\subsection{Inequalities}
\subsubsection{Transitivity}
\begin{itemize}
    \item If $a < b$ and $b < c$, then $a < c$
    \end{itemize}
\subsubsection{Multiplication}
Suppose $a < b$
\begin{itemize}
    \item If $c > 0$, then $ac < bc$
    \item If $c <  0$, then $ ac > bc$
\end{itemize}

\subsection{Exercise}
Find all number \( x \) such that 
\[\frac{x-8}{x-4} < 3 \]
Our first step  is to multiply by \(x-4\)
Here there are two conditions:
\begin{enumerate}
    \item \( x-4 > 0 \)
    \[ x -8 < 3(x-4)  \implies x-8 < 3x-12 \implies 2x > 4  \implies x > 2\]
    But our initial assumption is \(x-4 > 0 \implies x > 4 \). As \( 4 > 2 \), original inequality holds if \( x > 4\)
    \item \( x-4 < 0 \)
    \[ x-8 > 3(x-4) \implies  x < 2 \]
    Initial assumption is \( x<4 \). As \( 2 < 4\), inequality holds for \( x < 2 \)
\end{enumerate}
The original inequality holds true for \( [x < 2 ,  x > 4 ] \)
or 
\[  (-\infty, 2) \cup (4,\infty) \]

\subsection{Inequalities}
\subsubsection{Additive Inverse}
If \( a < b \) then \( -a > -b \)
Direction of inequalities has to be reversed when taking additive inverses on both sides 
\subsubsection{Multiplicative Inverse}
If \( a < b \)
\begin{itemize}
    \item If \(a > 0, b > 0\), then \(\frac{1}{a} > \frac{1}{b} \)
    \item If \(a < 0 < b\), then \(\frac{1}{a} < \frac{1}{b} \)
\end{itemize} 

\subsection{What is a Set?}
A \textbf{set} is a well-defined collection of distinct objects, called \textbf{elements} or \textbf{members} of the set.
\textbf{Representation of a Set:}
\begin{itemize}
    \item \textbf{Roster Form:} List elements inside curly braces:
    \[
    A = \{1, 2, 3, 4\}
    \]
    \item \textbf{Set-Builder Notation:} Describe properties of elements:
    \[
    A = \{x \mid x \text{ is a positive integer less than 5}\}
    \]
\end{itemize}
\subsubsection{Membership}
\begin{itemize}
    \item If \(x\) belongs to \(A\), write \(x \in A\).
    \item If \(x\) does not belong to \(A\), write \(x \notin A\).
\end{itemize}

\subsection{Types of Sets}
\begin{itemize}
    \item \textbf{Finite Set:} A set with a countable number of elements. \\
    Example: \(A = \{1, 2, 3, 4\} \)
    
    \item \textbf{Infinite Set:} A set with an uncountable or infinite number of elements. \\
    Example: \(\mathbb{N} = \{1, 2, 3, \dots\} \)

    \item \textbf{Empty/Null Set:} A set with no elements, denoted as \(\emptyset\) or \(\{\}\).

    \item \textbf{Subset:} \(A \subseteq B\) if every element of \(A\) is in \(B\).

    \item \textbf{Universal Set:} A set containing all objects under consideration, usually denoted by \(U\).

    \item \textbf{Power Set:} The set of all subsets of \(A\), denoted as \(P(A)\). \\
    Example: If \(A = \{1, 2\}\), then \(P(A) = \{\emptyset, \{1\}, \{2\}, \{1, 2\}\} \)
\end{itemize}

\subsection{Set Operations}
\subsubsection{Union (\(\cup\))}
Combines elements of two sets:
\[
    A \cup B = \{x \mid x \in A \text{ or } x \in B\}
\]
\subsubsection{Intersection (\(\cap\))}
Elements common to both sets:
\[
    A \cap B = \{x \mid x \in A \text{ and } x \in B\}
\]
\subsubsection{Difference (\(A - B\))}
Elements in \(A\) but not in \(B\):
\[
    A - B = \{x \mid x \in A \text{ and } x \notin B\}
\]
\subsubsection{Complement (\(A^c\))}
Elements not in the set \(A\):
\[
    A^c = \{x \mid x \notin A\}
\]
\subsubsection{Examples}
\begin{itemize}
    \item The set of natural numbers: \(\mathbb{N} = \{1, 2, 3, \dots\} \)
    \item The set of integers: \(\mathbb{Z} = \{\dots, -3, -2, -1, 0, 1, 2, 3, \dots\} \)
    \item The set of even numbers: \(\{2, 4, 6, \dots\} \)
\end{itemize}

\subsection{What is an Interval?}
An \textbf{interval} is a set of real numbers that includes all the numbers between two given endpoints.
Intervals describe ranges of values on the real number line and are widely used in mathematics.

\subsection{Types of Intervals}
\begin{itemize}
    \item \textbf{Closed Interval (\([a, b]\))}: Includes both endpoints \(a\) and \(b\).
    \[
    [a, b] = \{x \in \mathbb{R} \mid a \leq x \leq b\}
    \]
    Example: \([2, 5] = \{x \mid 2 \leq x \leq 5\} \).
    \vspace{5pt}

    \item \textbf{Open Interval (\((a, b)\))}: Excludes both endpoints \(a\) and \(b\).
    \[
    (a, b) = \{x \in \mathbb{R} \mid a < x < b\}
    \]
    Example: \((2, 5) = \{x \mid 2 < x < 5\} \).
\end{itemize}

\subsection{Half-Open or Half-Closed Intervals}
\begin{itemize}
    \item \textbf{Left-Closed, Right-Open (\([a, b)\))}:
    \[
    [a, b) = \{x \in \mathbb{R} \mid a \leq x < b\}
    \]
    Example: \([2, 5) = \{x \mid 2 \leq x < 5\} \).
    \vspace{5pt}

    \item \textbf{Left-Open, Right-Closed (\((a, b]\))}:
    \[
    (a, b] = \{x \in \mathbb{R} \mid a < x \leq b\}
    \]
    Example: \((2, 5] = \{x \mid 2 < x \leq 5\} \).
\end{itemize}

\subsection{Infinite Intervals}
\begin{itemize}
    \item \((a, \infty)\): All numbers greater than \(a\).
    \[
    (a, \infty) = \{x \in \mathbb{R} \mid x > a\}
    \]
    Example: \((3, \infty)\) includes all numbers greater than 3.

    \item \((-\infty, b)\): All numbers less than \(b\).
    \[
    (-\infty, b) = \{x \in \mathbb{R} \mid x < b\}
    \]
    Example: \((-\infty, 4)\) includes all numbers less than 4.

    \item \((-\infty, \infty)\): The entire real number line.
    \[
    (-\infty, \infty) = \mathbb{R}
    \]
\end{itemize}

\subsection{Summary of Interval Types}
\begin{tabular}{|c|c|c|}
    \hline
    \textbf{Type} & \textbf{Interval Notation} & \textbf{Description} \\
    \hline
    Closed         & \([a, b]\)      & Includes both endpoints \(a, b\) \\
    Open           & \((a, b)\)      & Excludes both endpoints \(a, b\) \\
    Half-Open Left & \([a, b)\)      & Includes \(a\), excludes \(b\) \\
    Half-Open Right & \((a, b]\)     & Excludes \(a\), includes \(b\) \\
    Infinite Left  & \((-\infty, b)\)& All \(x < b\) \\
    Infinite Right & \((a, \infty)\) & All \(x > a\) \\
    Entire Line    & \((-\infty, \infty)\) & All real numbers \\
    \hline
\end{tabular}

\subsection{What is Absolute Value?}
The \textbf{absolute value} of a number is its distance from zero on the number line, regardless of direction. It is always non-negative.
For a real number \(x\), the absolute value, denoted as \(|x|\), is defined as:
\[
|x| =
\begin{cases} 
x, & \text{if } x \geq 0, \\
-x, & \text{if } x < 0.
\end{cases}
\]
Breaking the absolute value: 
\begin{itemize}
    \item \( |f(x)| \leq c \quad \implies \quad -c \leq f(x) \leq c \)
    \item \( |f(x)| \geq c \quad \implies \quad f(x) \leq -c \quad \text{or} \quad f(x) \geq c \)
\end{itemize}

\subsection{Examples of Absolute Value}
\begin{itemize}
    \item \(|3| = 3\) \quad (because \(3 \geq 0\))
    \item \(|-5| = -(-5) = 5\) \quad (because \(-5 < 0\))
    \item \(|0| = 0\) \quad ( because \(0\) is neither positive nor negative)
\end{itemize}

\subsection{Properties of Absolute Value}
\begin{itemize}
    \item \textbf{Non-Negativity:} \(|x| \geq 0\) for all \(x\).
    \item \textbf{Identity Property:} \(|x| = 0 \quad \text{if and only if } x = 0\)
    \item \textbf{Multiplicative Property:} \(|x \cdot y| = |x| \cdot |y|\).
    \item \textbf{Triangle Inequality:} \( |x + y| \leq |x| + |y| \).
    \item \textbf{Distance Interpretation:} \(|x - y|\) represents the distance between \(x\) and \(y\).
\end{itemize}

\subsection{Exercises}
\subsubsection{Ball Bearings}
Ball bearings need to have extremely accurate sizes to work correctly. The ideal diameter of a particular ball bearing is 0.8 cm, but a ball bearing is declared acceptable if the error in the diameter size is less than 0.001 cm. Write the inequality for acceptance criteria
\subsubsection{Solution}
The ball bearings are acceptable if the diameter \(d\) satisfies:
\[
|d - 0.8| < 0.001
\]

\subsection{Exercises}
Find all numbers \(t \) such that \( |3t-4| = 10\)
Solution :
\[ 3t -4  = 10  \; or \; 3t-4 = -10  \implies t = \frac{14}{3}, t = -2 \]

\subsection{Exercise}
Find all numbers \(x\) such that \( \left| \frac{3x-5}{x-1} \right|< 2 \)
Solution :
\[ |3x-5| < 2 |x-1| \]
\begin{enumerate}
    \item \(x-1 > 0 \)
\end{enumerate}
Breaking the absolute value:
\begin{flalign}
    &\implies -2(x-1) < 3x-5 < 2(x-1) = -2x + 7 < 3x < 2x + 3  \ 
    &\implies 3x > -2x+7   \; \& \; 3x < 2x + 3  
\end{flalign}

\subsection{Exercise}
Solving for \( 3x < 2x + 3\)
\begin{flalign}
    &\implies x < 3 \\
\end{flalign}
Solving for \(3x > -2x+7 \)
\begin{flalign}
    &\implies 3x > -2x+7 \implies 5x > 7 \implies x > 7/5 \\
    &\implies x \in (7/5,3) 
\end{flalign}

\subsection{Exercise} 
\begin{enumerate}
    \setcounter{enumi}{1}
    \item \(x-1 < 0 \implies x < 1 \)
\end{enumerate}
\begin{flalign}
    &\implies |3x-5| < 2 |x-1| == |3x-5| < -2(x-1) \\
    &\implies         3x-5 < -2(x-1) \; \text{ and } \; -(3x-5) < -2(x-1) \\
    &\implies  3x-5 < -2(x-1)  \; \text{ and } \; 3x-5 > 2(x-1) 
\end{flalign}
\begin{flalign}
    &3x-5 < -2(x-1) \implies  3x < -2x+7 \implies 5x < 7 \implies x < 7/5 \\
    &\implies 3x-5 > 2(x-1) \implies 3x > 2x + 3 \implies x > 3 
\end{flalign}
Here \( x > 3\) is inconsistent with our assumption \(x < 1\). So for \(x<1\) there are no values of \(x\) satisfying the inequality