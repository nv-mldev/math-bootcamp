\documentclass{beamer}
\usetheme{AnnArbor}
\usepackage{graphicx}
\graphicspath{{pics/}}
\usepackage{amsmath}
\usepackage{amssymb}
\usepackage{framed}
\usepackage{tikz}
\usepackage[]{xcolor}
\usepackage[most]{tcolorbox}
\usepackage{pgfplots}
\pgfplotsset{compat=1.18}
\usepackage{blkarray}
\setbeamercolor{mycolorbox}{%
  bg=yellow!20,   % background color (20% yellow)
  fg=black      % foreground (text) color
}
\usefonttheme[onlymath]{serif}

\newtcolorbox{solutionblock}{
  colback=yellow!5,        % Background color: light yellow
  colframe=orange!80!black, % Frame color: dark orange
  title=Solution,
  fonttitle=\bfseries
}

\title{Introduction to Functions}
\author{Nithin}
\institute{}
\date{\today}
\begin{document}
\begin{frame}
  \titlepage
\end{frame}
\begin{frame}
  \tableofcontents
\end{frame}
\section{Slopes and Average Rate of Change}

% Slide: Motivation and Overview
\begin{frame}{Motivation: Why Study Change?}
  \begin{itemize}
    \item Calculus explores how quantities change and provides tools for modeling these changes.
    \item Functions link inputs ($x$) to outputs ($y=f(x)$); we investigate how $y$ varies as $x$ moves over an interval.
    \item Real-world example: Predicting economic indicators, modeling speeds, and more.
  \end{itemize}
\end{frame}

% Slide: Defining Average Rate of Change
\begin{frame}{Average Rate of Change}
  \begin{block}{Definition}
    For a function $f(x)$ on interval $[a,b]$, the \emph{average rate of change} is
    \[ 
      \frac{f(b)-f(a)}{b-a},
    \]
    which geometrically represents the slope of the secant line between $(a,f(a))$ and $(b,f(b))$.
  \end{block}
  \begin{itemize}
    \item Rise: $f(b)-f(a)$
    \item Run: $b-a$
    \item Secant line smooths out fluctuations; reports overall trend.
  \end{itemize}
\end{frame}

% Slide: Graphical Illustration
\begin{frame}{Graphical Illustration}
  \begin{columns}
    \column{0.6\textwidth}
      \includegraphics[width=\textwidth]{figures/secant_line_example.pdf}
    \column{0.4\textwidth}
      \begin{itemize}
        \item Focus on interval $[a,b]$ on the curve $y=f(x)$.
        \item Secant line (in blue) joins $(a,f(a))$ and $(b,f(b))$.
        \item Its slope measures the average change in $y$ per unit change in $x$.
      \end{itemize}
  \end{columns}
\end{frame}

% Slide: Real-World Example (Car Trip)
\begin{frame}{Example: Average Speed of a Car Trip}
  \begin{itemize}
    \item Distance from Sydney to Melbourne: 873 km over about 11.5 hours.
    \item Average speed: $\frac{873}{11.5}\approx75.4$ km/h.
    \item Even with stops (flat segments), the secant slope gives overall performance.
  \end{itemize}
\end{frame}

% Slide: Sensitivity to Rounding
\begin{frame}{Rounding and Accuracy}
  \begin{itemize}
    \item Distance measurements rounded to nearest kilometer introduce potential error.
    \item Small changes in endpoints can shift computed average speed above or below speed limit.
    \item Highlights importance of data precision in applications.
  \end{itemize}
\end{frame}

% Slide: Toward Instantaneous Rate of Change
\begin{frame}{Instantaneous Rate of Change}
  \begin{itemize}
    \item Instantaneous speed corresponds to slope of tangent line at a point.
    \item As secant interval shrinks ($b\to a$), average rate approaches derivative $f'(x)$.
    \item Next: Formalize tangent lines and derivatives.
  \end{itemize}
\end{frame}
\end{document}